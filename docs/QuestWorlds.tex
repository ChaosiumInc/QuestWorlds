% Options for packages loaded elsewhere
\PassOptionsToPackage{unicode}{hyperref}
\PassOptionsToPackage{hyphens}{url}
%
\documentclass[
]{article}
\usepackage{lmodern}
\usepackage{amssymb,amsmath}
\usepackage{ifxetex,ifluatex}
\ifnum 0\ifxetex 1\fi\ifluatex 1\fi=0 % if pdftex
  \usepackage[T1]{fontenc}
  \usepackage[utf8]{inputenc}
  \usepackage{textcomp} % provide euro and other symbols
\else % if luatex or xetex
  \usepackage{unicode-math}
  \defaultfontfeatures{Scale=MatchLowercase}
  \defaultfontfeatures[\rmfamily]{Ligatures=TeX,Scale=1}
\fi
% Use upquote if available, for straight quotes in verbatim environments
\IfFileExists{upquote.sty}{\usepackage{upquote}}{}
\IfFileExists{microtype.sty}{% use microtype if available
  \usepackage[]{microtype}
  \UseMicrotypeSet[protrusion]{basicmath} % disable protrusion for tt fonts
}{}
\makeatletter
\@ifundefined{KOMAClassName}{% if non-KOMA class
  \IfFileExists{parskip.sty}{%
    \usepackage{parskip}
  }{% else
    \setlength{\parindent}{0pt}
    \setlength{\parskip}{6pt plus 2pt minus 1pt}}
}{% if KOMA class
  \KOMAoptions{parskip=half}}
\makeatother
\usepackage{xcolor}
\IfFileExists{xurl.sty}{\usepackage{xurl}}{} % add URL line breaks if available
\IfFileExists{bookmark.sty}{\usepackage{bookmark}}{\usepackage{hyperref}}
\hypersetup{
  hidelinks,
  pdfcreator={LaTeX via pandoc}}
\urlstyle{same} % disable monospaced font for URLs
\usepackage{longtable,booktabs}
% Correct order of tables after \paragraph or \subparagraph
\usepackage{etoolbox}
\makeatletter
\patchcmd\longtable{\par}{\if@noskipsec\mbox{}\fi\par}{}{}
\makeatother
% Allow footnotes in longtable head/foot
\IfFileExists{footnotehyper.sty}{\usepackage{footnotehyper}}{\usepackage{footnote}}
\makesavenoteenv{longtable}
\usepackage{graphicx}
\makeatletter
\def\maxwidth{\ifdim\Gin@nat@width>\linewidth\linewidth\else\Gin@nat@width\fi}
\def\maxheight{\ifdim\Gin@nat@height>\textheight\textheight\else\Gin@nat@height\fi}
\makeatother
% Scale images if necessary, so that they will not overflow the page
% margins by default, and it is still possible to overwrite the defaults
% using explicit options in \includegraphics[width, height, ...]{}
\setkeys{Gin}{width=\maxwidth,height=\maxheight,keepaspectratio}
% Set default figure placement to htbp
\makeatletter
\def\fps@figure{htbp}
\makeatother
\setlength{\emergencystretch}{3em} % prevent overfull lines
\providecommand{\tightlist}{%
  \setlength{\itemsep}{0pt}\setlength{\parskip}{0pt}}
\setcounter{secnumdepth}{-\maxdimen} % remove section numbering

\author{}
\date{}

\begin{document}

{
\setcounter{tocdepth}{3}
\tableofcontents
}
\hypertarget{credits-legal-information}{%
\section{0.0 Credits \& Legal
Information}\label{credits-legal-information}}

\hypertarget{legal-information}{%
\subsection{0.1 Legal Information}\label{legal-information}}

The \emph{QuestWorlds} System Reference Document 0.1 (``QWSRD0.1'')
describes the rules of \emph{QuestWorlds}. You may incorporate the rules
as they appear in QWSRD0.1, wholly or in part, into a derivative work,
through the use of the \emph{QuestWorlds} Open Game License, Version
1.0. You should read and understand the terms of that License before
creating a derivative work from QWSRD0.1.

Thanks to Wizards of the Coast, the Open Source Initiative, and Creative
Commons for their work in creating the framework behind Open Source (and
in this case Open Game) licenses. You should be aware that the
\emph{QuestWorlds} Open Game License for use of the \emph{QuestWorlds}
system differs from the Wizards Open Game License and has different
terms and conditions.

\hypertarget{using-this-license}{%
\subsubsection{0.1.1 Using This License}\label{using-this-license}}

You should note that this is version of 0.1 of the \emph{QuestWorlds}
System Reference Document. We expect to release revised versions of this
SRD, especially after development of Chaosium's upcoming
\emph{QuestWorlds Core Book}. When we release the \emph{QuestWorlds Core
Book} we will update the version designation to 1.0, indicating that the
SRD reflects the text published in that book. If you are developing
materials for \emph{QuestWorlds} projects you may want to bear this in
mind. We will track any changes to the SRD at
\emph{https://github.com/ChaosiumInc/QuestWorlds}.

Once we release SRD version 1.0 we expect that to be stable for some
time.

If you have questions about this license, please reach out to Moon
Design at licensing@chaosium.com.

\hypertarget{questworlds-open-game-license-version-1.0}{%
\subsubsection{\texorpdfstring{0.1.2 \emph{QuestWorlds} Open Game
License, Version
1.0}{0.1.2 QuestWorlds Open Game License, Version 1.0}}\label{questworlds-open-game-license-version-1.0}}

All Rights Reserved.

\begin{enumerate}
\def\labelenumi{\arabic{enumi}.}
\tightlist
\item
  Definitions:
\end{enumerate}

\begin{enumerate}
\def\labelenumi{(\alph{enumi})}
\item
  ``Contributors'' means the copyright and/or trademark owners who have
  contributed Open Game Content;
\item
  ``Derivative Material'' means copyrighted material including
  derivative works and translations (including into computer languages),
  potation, modification, correction, addition, extension, upgrade,
  improvement, compilation, abridgment, or other forms in which an
  existing work may be recast, transformed, or adapted;
\item
  ``Distribute'' means to reproduce, license, rent, lease, sell,
  broadcast, publicly display, transmit, or otherwise distribute;
\item
  ``Open Game Content'' means the \emph{QuestWorlds} game, including the
  game mechanics and the methods, procedures, processes, and routines to
  the extent such content does not embody Prohibited Content and is an
  enhancement over the prior art and any additional content clearly
  identified as Open Game Content by the Contributor, and means any work
  covered by this License, including translations and derivative works
  under copyright law, but specifically excludes Prohibited Content;
\item
  The following items are hereby identified as ``Prohibited Content'':
  All trademarks, registered trademarks, proper names (characters,
  deities, place names, etc.), plots, story elements, locations,
  characters, artwork, or trade dress from any of the following: any
  releases from the product lines of \emph{Call of Cthulhu},
  \emph{Dragon Lords of Melniboné}, \emph{ElfQuest}, \emph{Elric!},
  \emph{Hawkmoon}, \emph{HeroQuest}, \emph{Hero Wars}, \emph{King Arthur
  Pendragon}, \emph{Magic World}, \emph{Nephilim}, \emph{Prince
  Valiant}, \emph{Ringworld}, \emph{RuneQuest}, \emph{7th Sea},
  \emph{Stormbringer}, \emph{Superworld}, \emph{Thieves' World},
  \emph{Worlds of Wonder}, and any related sublines; the world and
  mythology of Glorantha; all works related to the Cthulhu Mythos,
  including those that are otherwise public domain; and all works
  related to \emph{Le Morte d'Arthur}. This list may be updated in
  future versions of the License.
\item
  ``Trademark'' means the logos, names, marks, signs, mottos, and
  designs that are used by a Contributor to identify itself or its
  products or the associated products contributed to the
  \emph{QuestWorlds} Open Game License by the Contributor;
\item
  ``Use,'' ``Used,'' or ``Using'' means to use, distribute, copy, edit,
  format, modify, translate, and otherwise create Derivative Material of
  Open Game Content;
\item
  ``You'' or ``Your'' means the licensee in terms of this agreement.
\end{enumerate}

\begin{enumerate}
\def\labelenumi{\arabic{enumi}.}
\setcounter{enumi}{1}
\item
  Grant: Except for material designated as Prohibited Content (see
  Section 1(e) above), the \emph{QuestWorlds} System Reference Document
  is Open Game Content, as defined in the \emph{QuestWorlds} Open Game
  License version 1.0, Section 1(d). No portion of this work other than
  the material designated as Open Game Content may be reproduced in any
  form without permission from Moon Design.
\item
  The License: This License applies to any work Using \emph{QuestWorlds}
  Open Game Content published by Moon Design. You must affix a complete
  copy of this License to any \emph{QuestWorlds} Open Game Content that
  You Use and include the Copyright Notice detailed in Section 7 in all
  appropriate locations. No terms may be added to or subtracted from
  this License except as described by the License itself. No other terms
  or conditions may be applied to any \emph{QuestWorlds} Open Game
  Content distributed Using this License.
\item
  Offer and Acceptance: By Using the \emph{QuestWorlds} Open Game
  Content You indicate Your acceptance of the terms of the
  \emph{QuestWorlds Open Game License}.
\item
  Grant of License: Subject to the terms and conditions of this License,
  the Contributors grant You a perpetual, worldwide, royalty-free,
  non-exclusive license to Use the Open Game Content.
\item
  Representation of Authority to Contribute: If You are contributing
  original material as Open Game Content, You represent that Your
  contributions are Your original creation and/or You have sufficient
  rights to grant the rights conveyed by this License.
\item
  Copyright Notice: You must update the Copyright Notice portion of this
  License to include the current version of the text of the Copyright
  Notice of any \emph{QuestWorlds} Open Game Content You are copying,
  modifying, or distributing.
\end{enumerate}

This work created using the \emph{QuestWorlds} Open Game License.

\emph{QuestWorlds} Open Game License v 1.0 © copyright 2020 Moon Design
Publications LLC.

\emph{QuestWorlds} © copyright xxxx--2020 Moon Design Publications LLC;
Author, original rules: Robin D. Laws; developed by Greg Stafford, Ian
Cooper, David Dunham, Mark Galeotti, Stephen Martin, Jeff Richard, Neil
Robinson, Roderick Robinson, David Scott, and Lawrence Whitaker.

Chained Contests and Plot Edits from \emph{Mythic Russia} © copyright
2006, 2010 Mark Galeotti; developed by Graham Robinson (for ``Chained
Contests'') and added as Open Game Content here with permission.

QuestWorlds and the QuestWorlds logo are trademarks of Moon Design
Publications LLC. Used with permission.

\begin{enumerate}
\def\labelenumi{\arabic{enumi}.}
\setcounter{enumi}{7}
\item
  Limitations on Grant: You agree not to Use any Prohibited Content,
  except as expressly licensed in another, independent Agreement with
  Moon Design. You agree not to indicate compatibility or
  co-adaptability with any Trademark or Registered Trademark in
  conjunction with a work containing \emph{QuestWorlds} Open Game
  Content except as expressly licensed in another, independent Agreement
  with the owner of such Trademark or Registered Trademark.
\item
  Identification: If you distribute Open Game Content You must clearly
  indicate which portions of the work that you are distributing are Open
  Game Content.
\item
  Updating the License: Moon Design or its designated Agents may publish
  updated versions of the \emph{QuestWorlds} Open Game License,
  including updates to the Prohibited Content list. Material published
  under any version of the License can continue to be published Using
  the terms of that version, but You agree to Use the most recent
  authorized version of this License for any new Open Game Content You
  publish or for revised or updated works with thirty percent (30\%) or
  more revised or new content.
\item
  Use of Contributor Credits: You may not market or advertise the Open
  Game Content using the name of any Contributor unless You have written
  permission from the Contributor to do so.
\item
  Reputation: You must not copy, modify, or distribute Open Game Content
  connected to this License in a way that would be prejudicial or
  harmful to the honor or reputation of the Contributors.
\item
  Inability to Comply: If it is impossible for You to comply with any of
  the terms of this License with respect to some or all of the
  \emph{QuestWorlds} Open Game Content due to statute, judicial order,
  or governmental regulation then You may not Use any Open Game Material
  so affected.
\end{enumerate}

14. Termination: This License will terminate automatically if You fail
to comply with all terms herein and fail to cure such breach within
thirty (30) days of becoming aware of the breach.

\begin{enumerate}
\def\labelenumi{\arabic{enumi}.}
\setcounter{enumi}{14}
\tightlist
\item
  Labeling: You must prominently display one of the following
  \emph{QuestWorlds} logos on the front and back exterior and in the
  interior package, on the title page or its equivalent, of your Use of
  the Open Content. You are granted permission to reproduce the logo
  only for that purpose.
\end{enumerate}

\begin{figure}
\centering
\includegraphics{Logos/QuestWorlds-Logo-TM-black.png}
\caption{BW QuestWorlds Logo}
\end{figure}

\begin{figure}
\centering
\includegraphics{Logos/QuestWorlds-Logo-TM.png}
\caption{Color QuestWorlds Logo}
\end{figure}

\begin{enumerate}
\def\labelenumi{\arabic{enumi}.}
\setcounter{enumi}{15}
\item
  Severability: If any provision of this License is held to be
  unenforceable, such provision shall be severed only to the extent
  necessary to make it enforceable.
\item
  Governing Law and Venue: The governing law for any disputes arising
  under this License shall be the laws of the State of Michigan, without
  reference to its choice of laws provisions. Venue is exclusively
  vested in the United States Federal District Court for the Eastern
  District of Michigan.
\end{enumerate}

\hypertarget{credits}{%
\subsection{0.2 Credits}\label{credits}}

Original Rules: Robin D. Laws

Further Development: Greg Stafford, Ian Cooper, David Dunham, Mark
Galeotti, Stephen Martin, Jeff Richard, Neil Robinson, Roderick
Robinson, David Scott, Lawrence Whitaker

Additional Contributions: Shannon Appelcline, Simon Bray, David Cake,
Dave Camoirano, Melissa Camoirano, John Carnahan, Charles Corrigan,
David Dunham, Alex Ferguson, James Frusetta, Phil Hibbs, Simon D. Hibbs,
Jeff Kyer, Martin Laurie, Mark Leymaster, Julian Lord, Rick Meints,
Peter Metcalfe, Peter Nordstrand, Wesley Quadros, Mikael Raaterova,
Jamie Revell, Graham Robinson, Jonas Schiött, Gary Sturgess, Ian
Thomson, Nils Weinander

Development of this version: Ian Cooper

Development Assistance for this version: Jonathan Laufersweiler, James
Lowder, Michael O'Brien, Jeff Richard

Proofreading of this version: Martin Helsdon

\hypertarget{introduction}{%
\section{1.0 Introduction}\label{introduction}}

\emph{QuestWorlds} is a roleplaying rules engine suitable for you to
play in any genre.

It is a traditional roleplaying game in that there is a GM and players.
The players play characters, each guided by the internal thoughts of
their character as to what decisions they make, and the GM plays the
world, including non-player characters (NPCs) and abstract threats.

It features an abstract, conflict-based, resolution method and scalable,
customizable, character descriptions. Designed to emulate the way
characters in fiction face and overcome challenges, it is suitable for a
wide variety of genres and play styles.

It is a rules-light system that facilitates beginning play easily, and
resolving conflicts in play quickly.

We refer to a rules-light but traditional roleplaying game as a
storytelling game, after Greg Stafford's definition in \emph{Prince
Valiant}.

\hypertarget{why-questworlds}{%
\subsection{1.1 Why QuestWorlds?}\label{why-questworlds}}

\emph{QuestWorlds} is meant to facilitate your creativity, and then to
get out of your way.

It is well suited to a collaborative, friendly group with a high degree
of trust in each other's creativity. Characters in \emph{QuestWorlds}
are described more in terms of their place in your imagination and the
game setting than by game mechanics.

If your group are often at odds and rely on their chosen rules kit as an
arbiter between competing visions of how the game ought to develop, or
use mechanical options to decide ``what action to take,''
\emph{QuestWorlds} is not a rules set that provides that structure. Make
sure to discuss with your group whether you are collectively on board
with trying a new play style dynamic, or if you would rather stick to
more structured systems.

\hypertarget{version}{%
\subsection{1.2 Version}\label{version}}

The first version of these rules \emph{Hero Wars} was published in 2000
(ISBN 978-1-929052-01-1)

The second version \emph{HeroQuest} was published in 2003 (ISBN
978-1-929052-12-7). We refer to this as \emph{HeroQuest} 1e to
disambiguate.

The third version \emph{HeroQuest}: Core Rules was published in 2009
(ISBN 978-0-977785-32-2). We refer to this as \emph{HeroQuest} 2e.

\emph{HeroQuest Glorantha} was published in 2015 (ISBN
978-1-943223-01-5). It is the version of the rules in \emph{HeroQuest}
2e, presented for playing in Glorantha. We refer to this as
\emph{HeroQuest} 2.1e.

\emph{QuestWorlds} was published as a System Reference Document (SRD)
(this document) in 2019. The version of the rules here is slightly
updated, mainly to clarify ambiguities, from the version presented in
\emph{HeroQuest} 2e and \emph{HeroQuest} 2.1e. This makes this ruleset
\emph{HeroQuest} 2.2e, despite the name change. However, to simplify we
identify this version as \emph{QuestWorlds} 1e.

An Appendix lists changes in this version. As the SRD is updated we will
continue to track version changes there.

\hypertarget{basic-advanced-mechanics}{%
\subsection{1.3 Basic \& Advanced
Mechanics}\label{basic-advanced-mechanics}}

As the \emph{QuestWorlds} engine is oriented towards rules-light play,
we designate some mechanics as ``Basic'' and others as ``Advanced.''

Basic rules represent the minimum mechanical structure you need to play
a game with the feel of \emph{QuestWorlds}, while advanced rules add
mechanical depth to areas of play which you might wish to emphasize,
though possibly at a cost in speed of play.

In this document, we put advanced mechanics in their own section from
the relevant basic mechanic section.

The advanced mechanics can be added to play individually; you do not
have to bring them in together. In cases where an optional mechanic
depends upon or interrelates with another optional mechanic, we note
that in the text.

If this is your first time reading these rules, you may wish to skip
over these sections. Return to them once you understand the basic
mechanics. Similarly, if this is your first time playing these rules,
you may wish to omit the advanced rules at first, getting a feel for the
game in its simplest form, and then add them in as you want more depth.
Your GM may even decide to employ certain advanced rules on a case by
case basis, providing more detail or nuance to particularly interesting
conflicts where beneficial, while otherwise taking a lighter approach.

\hypertarget{who-is-this-document-for}{%
\subsubsection{1.3.1 Who Is This Document
For}\label{who-is-this-document-for}}

The primary audience for this document is game-designers who wish to
utilize the \emph{QuestWorlds} rules framework to implement their own
game.

We also recognize that some people will use this document to learn about
the \emph{QuestWorlds} system before purchasing it, and some players in
games were the GM has a rule book, may use this as a reference to help
understand the rules.

For that latter reason, we address the rules here to a player.

However, this remains a technical document with few examples, advice, or
other non-rules text to help you play your game, as such are beyond the
scope of this System Reference Document.

It is expected that the designers of games you play based on these rules
will include such guidance and context as is relevant to their game's
particular genre or setting, presented in a format better suited for
learning how to play.

\hypertarget{numbering}{%
\subsection{1.4 Numbering}\label{numbering}}

Sections within this document are numbered. This is for the benefit of
game designers and reviewers.

This does not imply that game designers need number the rules in their
own games.

Numbering however makes it easy to refer to rules in this document when
page numbers may vary by presentation format for the purposes of error
trapping or tracking changes. If you need to give us feedback about this
document, that will assist us.

\hypertarget{participants}{%
\subsection{1.5 Participants}\label{participants}}

\hypertarget{players}{%
\subsubsection{1.5.1 Players}\label{players}}

You and your fellow players each create a Player Character (PC) to be
the ``avatar'' or ``persona'' whose role you will play in the game. The
PCs pursue various goals in an imaginary world, using their
\textbf{abilities}, motivations, connections, and more to solve problems
and overcome \textbf{story obstacles} that stand in their way.

When we say `you' in this document we may mean the player or their PC.
Which should be clear from context, or explicitly noted.

\hypertarget{game-master}{%
\subsubsection{1.5.2 Game Master}\label{game-master}}

Your Game Master (GM) is the interface between your imagination and the
game-world in which the PCs have their adventures; describing the
people, places, creatures, objects, and events therein. Your GM also
plays the role of any Non-Player Characters (NPCs) with whom your PC
interacts in the course of your adventures.

We generally refer to the GM as `your GM' in this document's
player-facing language. However, if you are the GM for a given game,
this naturally refers to you.

\hypertarget{mechanics}{%
\section{2.0 Mechanics}\label{mechanics}}

In a \emph{QuestWorlds} game, stories develop dynamically as you and
your GM work together to role-play the dramatic conflict between your
group's PCs in pursuit of their goals and the challenges, or threats
that your GM presents to stand in your way. Stories advance by two
methods: conflict, where your PC is prevented from achieving their goals
because there is something that must be overcome, a \textbf{story
obstacle}, to gain a desired person, thing, or even status: the
\textbf{prize}; or revelation, where something must be overcome, a
\textbf{story obstacle}, to learn a secret, uncover the past, or reach
understanding: the \textbf{prize}.

Over the course of play, your GM will present various \textbf{story
obstacles} as conflicts to the PCs, resulting in either \textbf{victory}
or \textbf{defeat} for your character, which determines whether or not
you gain the \textbf{prize} you sought. These conflicts can represent
any sort of challenge you might face: fighting, a trial or debate,
survival in a harsh environment, out-wooing rival suitors, and so on.

Rather than mechanically addressing the individual tasks that make up
these conflicts, \emph{QuestWorlds} usually assesses your overall
\textbf{victory} or \textbf{defeat} in a single \textbf{contest} where
you and your GM make an opposed roll pitting your characters
\textbf{ability} vs the \textbf{resistance} the \textbf{story obstacle}
presents to you achieving the \textbf{prize}.

Whenever the GM presents a \textbf{story obstacle} for you to overcome,
you should \textbf{frame the contest} by describing what you are trying
to accomplish, the \textbf{prize}, and which of your \textbf{abilities}
(see below) you want to use to achieve that \textbf{prize}, and how.

Based on that \textbf{framing} and other factors, your GM will assess
what \textbf{resistance} the characters face.

You roll a twenty-sided die (D20) against your PC's \textbf{ability},
and your GM rolls a D20 against the \textbf{resistance}. Your GM will
assess your overall \textbf{victory} or \textbf{defeat} in the contest
based on the \textbf{success} or \textbf{failure} of both rolls, and
narrates the results of your attempt to overcome the \textbf{story
obstacle} and gain the \textbf{prize} accordingly. The direction of the
story changes, in either a big or small way, depending on whether you
gain the \textbf{prize} or not.

We encourage your GM to work with your suggestions when narrating the
\textbf{victory} or \textbf{defeat}, but the final decision rests with
them.

\hypertarget{abilities}{%
\subsection{2.1 Abilities}\label{abilities}}

Characters in \emph{QuestWorlds} are defined by the \textbf{abilities}
they use to face the challenges that arise in the course of their story.
Rather than having a standard list of attributes, skills, powers, etc.
for all characters, anything that you can apply to solve a problem or
overcome a \textbf{story obstacle} could be one of your
\textbf{abilities}. While your GM may provide some example
\textbf{abilities} to choose from that connect your PC to a particular
story or game world (whether created by your GM or by the designer of a
particular game), you get to make up and describe most or all of your
\textbf{abilities}.

Some \textbf{abilities} might be broad descriptions of your background
or expertise, like ``Dwarf of the Chalk Hills'' or ``Private Detective''
- implying a variety of related capabilities. Others might represent
specific capabilities or assets such as ``Lore of the Ancients,''
``Captain of the Fencing Team,'' or ``The Jade Eye Medallion.''

Ultimately, \textbf{abilities} are just names for the interesting things
your character can do.

\hypertarget{ratings-and-masteries}{%
\subsubsection{2.1.2 Ratings and
Masteries}\label{ratings-and-masteries}}

\emph{QuestWorlds} \textbf{abilities} are \textbf{rated} on a range of
1--20, representing the \textbf{target number (TN)} you need to roll or
less to succeed on your roll during a \textbf{contest} (see §2.3 for
more details). \textbf{Ratings} are also scalable beyond that range
using tiers of capability we refer to as \textbf{Mastery}. When you
raise a \textbf{rating} of 20 by one point, either permanently through
character advancement or a temporarily with \textbf{modifier} to a
contest roll, the \textbf{rating} increases not to 21, but to 1M.

The ``M'' after the \textbf{rating} signifies \textbf{mastery}. You have
now reached a new order of excellence in that \textbf{ability}, such
that your die rolls will almost always succeed. Unless opposed by
similarly exalted \textbf{resistance} (see §2.3.5.2 for more details)
the number in front of the M is now the \textbf{target number} you seek
to roll or less to achieve a \textbf{critical}!

As a \textbf{rating} climbs, you may even gain multiple
\textbf{masteries} in it. \textbf{Mastery} tiers above one (representing
an overall \textbf{rating} or 41 or more) are marked with a number to
the right of the M symbol. Each successive \textbf{rating} increase over
20 becomes a new \textbf{mastery} tier. Thus, if you have 10M2, you have
two \textbf{masteries} (representing a total \textbf{rating} of 50,
10+20+20). 10M3 means that you have three \textbf{masteries}, and so on.

Specific \emph{QuestWorlds} games or genre packs may use other symbols
relevant to their setting or genre to denote \textbf{mastery} instead of
M. If so, this should be clearly noted by their designers.

In summary, to reflect abilities (or \textbf{resistance}s), higher than
20, you divide the \textbf{ability} by 20, noting a \textbf{mastery} for
each multiple of 20 the \textbf{ability} exceeds, and treating the
remainder as the \textbf{target number}. So, for an \textbf{ability} of
27, 27 ÷ 20 = 1 remainder 7, which translates into a \textbf{rating} of
seven with one \textbf{mastery} written as ``7M''; while for an
\textbf{ability} of 43, 43 ÷ 20 = 2 remainder 3, which translates into a
\textbf{rating} of three with two \textbf{masteries}, written as ``3M2''
on the character sheet.

\hypertarget{no-relevant-ability}{%
\paragraph{\texorpdfstring{2.1.2.1 No Relevant
\textbf{ability}}{2.1.2.1 No Relevant ability}}\label{no-relevant-ability}}

You may sometimes be faced with a \textbf{story obstacle} for which you
have no relevant \textbf{ability} whatsoever. In such cases, you may
still enter into conflict with the \textbf{story obstacle} using a
minimum base \textbf{target number} of 6 for your \textbf{contest} roll.
Like \textbf{ratings}, it may also be subject to \textbf{modifiers}.

\hypertarget{making-ratings-quantitative}{%
\paragraph{2.1.2.2 Making Ratings
Quantitative}\label{making-ratings-quantitative}}

While \emph{QuestWorlds} generally treats \textbf{ratings} as abstract
measures of problem solving power rather than quantitive measures of
in-fiction traits, some games may also depart from this practice in
order to more closely couple key fictional elements to the mechanics.
For example, a magic system might classify certain supernatural effects
as Apprentice, Journeyman, or Master level, and require \textbf{ratings}
of 15, 5M, or 1M2 (respectively) in a relevant \textbf{ability} to even
attempt them.

Such departures from abstraction should generally only be made where the
increased complexity they bring leads to rewarding choices in a key area
of interest to the setting or genre at hand. In most cases, you and your
GM can simply follow the fiction surrounding your \textbf{ability} and
its context within the setting for guidance as to what applications of
the \textbf{ability} are credible.

\hypertarget{possessions-and-equipment}{%
\subsection{2.2 Possessions and
Equipment}\label{possessions-and-equipment}}

Your character will generally be considered to have whatever equipment
is reasonably implied by your abilities. Having an ``Athenian Hoplite''
\textbf{ability} will mean that your character possesses bronze armor, a
shield, a spear, and a short-sword; while a ``Country Doctor'' would be
expected to have a well-stocked medical-bag and possibly a horse \&
buggy in the right setting.

However, if you wish your character to possess something that is
particularly special, interesting, or unusual, you may also enumerate it
as a rated \textbf{ability} in its own right, just like any other
\textbf{ability} your character might use to solve a problem.

In play, the degree to which you can overcome \textbf{story obstacles}
with your possessions depends not on any qualities inherent to the
objects themselves, but to the \textbf{rating} of your relevant
\textbf{ability}. However the significance of various sorts of gear lies
in the types of actions you can credibly propose, and what their impact
might reasonably be. An ``Invisibility Cloak'' \textbf{ability} implies
very different fictional capabilities than ``Souped-up Muscle Car''
does.

Conversely, if in the course of play you find your character in a
situation without equipment essential to utilize an ability effectively,
or where your character's gear is poorly suited to the task at hand,
your GM may take into account in assessing credibility-based
\textbf{modifiers}.

\hypertarget{wealth}{%
\subsubsection{2.2.1 Wealth}\label{wealth}}

In \emph{QuestWorlds}, wealth is treated as just another way to overcome
\textbf{story obstacles}. Many characters may not even have an explicit
wealth \textbf{ability}, with their wealth or assets instead implied by
\textbf{abilities} representing their background, profession, or status.
Whether explicit or implied, the relevant \textbf{rating} is not an
objective measure of the size of your fortune, but instead indicates how
well you solve problems with money and resources.

Similarly, treasures and other windfalls that arise during play may be
represented in the rules via \textbf{plot augments} (see § 2.4.3) in
\textbf{contests} where using the resource is relevant.

\hypertarget{contest-procedure}{%
\subsection{2.3 Contest Procedure}\label{contest-procedure}}

You choose an \textbf{ability} relevant to the conflict at hand,
describe exactly what you are trying to accomplish, and how. Your GM may
modify these suggested actions to better fit the fictional
circumstances, and describe the actions of the NPCs or forces on the
other side of the conflict.

\hypertarget{resolution-methods}{%
\subsubsection{2.3.1 Resolution Methods}\label{resolution-methods}}

The Basic resolution methods are as follows:

\hypertarget{automatic-victory}{%
\paragraph{2.3.1.1 Automatic Victory}\label{automatic-victory}}

Sometimes, your GM may not call for a \textbf{contest} at all, in which
case you are simply victorious in overcoming the \textbf{story obstacle}
at hand. This may be because \textbf{defeat} in the conflict to overcome
the \textbf{story obstacle} would lead to uninteresting results or a
narrative dead-end, such as when finding an important clue is essential
to the progress of an adventure.

Your GM might also do this in cases where, within the fictional context,
the particular \textbf{ability} you have brought to bear on the
\textbf{story obstacle} is such that overcoming the \textbf{story
obstacle} should be a trivial matter under normal circumstances (e.g.~-
a professional hunter bringing in the evening meal in a forest filled
with game animals). In such cases, \textbf{defeat} would simply not be
credible unless your GM wanted to introduce some further complications
to the story. Generally, your GM will only use \textbf{automatic
victory} when you have a relevant \textbf{ability} to justify its
application.

For cases where overall \textbf{victory} may be a given, but the degree,
timeliness, or cost of that \textbf{victory} may be interesting
concerns, consider the Advanced rules in §2.19.

\hypertarget{simple-contest}{%
\paragraph{2.3.1.2 Simple Contest}\label{simple-contest}}

The \textbf{simple contest} \emph{QuestWorlds}' primary resolution
mechanic for overcoming \textbf{story obstacles}, and is used the most
often. It also provides the foundation for other types of
\textbf{contest}, including several Advanced ones. As such, it receives
both an overview of key concepts here as well as a more detailed
treatment in §4.

At is most basic, a \textbf{simple contest} can be summarized as
follows:

\begin{enumerate}
\def\labelenumi{\arabic{enumi}.}
\tightlist
\item
  You and your GM agree upon the terms of the \textbf{contest}.
\item
  You roll a D20 vs your relevant \textbf{ability}, while your GM rolls
  a D20 vs the \textbf{resistance}.
\item
  Your GM compares the \textbf{success} or \textbf{failure} of the two
  rolls, and assesses your overall \textbf{victory} or \textbf{defeat}.
\item
  Your GM then narrates the \textbf{outcome} of the conflict as
  appropriate.
\end{enumerate}

If you enter into conflict with another player rather than a
\textbf{story obstacle} presented by your GM, you both roll your
relevant abilities for the \textbf{contest} instead of against a GM-set
\textbf{resistance}, and your GM interprets the \textbf{results}, as
described above.

\hypertarget{framing-the-contest}{%
\subsubsection{2.3.2 Framing the Contest}\label{framing-the-contest}}

\hypertarget{contest-framing-overview}{%
\paragraph{2.3.2.1 Contest Framing
Overview}\label{contest-framing-overview}}

When a conflict arises during the game, you and your GM start by clearly
agreeing on:

\begin{itemize}
\tightlist
\item
  What goal you are trying to achieve. We call this the \textbf{prize}.
\item
  What the \textbf{story obstacle} is you are trying to overcome.
\item
  What \textbf{tactic} you are using to and overcome it.
\end{itemize}

This process is called \textbf{framing the contest}.

\hypertarget{conflict-goals-vs-obstacles}{%
\paragraph{2.3.2.2 Conflict: Goals vs
Obstacles}\label{conflict-goals-vs-obstacles}}

\textbf{Contests} in \emph{QuestWorlds} don't simply tell you how well
you performed at a particular task: they tell you whether or not you
overcame a \textbf{story obstacle}, which moves the story in a new
direction. Unlike some other roleplaying games, a \textbf{contest} in
\emph{QuestWorlds} does not resolve a task, it resolves the whole
\textbf{story obstacle}.

If you need secret records which are stored in a vault within a
government compound, your goal is to get the information - while the
fact that it is secured against your access is a \textbf{story obstacle}
you must overcome to attain that goal. Overcoming that \textbf{story
obstacle} may involve many possible tasks, evading guards, lock-picking,
forging credentials, etc. - but the \textbf{contest} doesn't address
those individually. The \textbf{contest} is framed around the entire
conflict against the \textbf{story obstacle} as a whole.

In a fight, your \textbf{story obstacle} may be the opponents
themselves, who you are fighting to capture or kill. Just as often you
are seeking another goal and you might just as easily attain it by
incapacitating or evading your foes. In this case, beating the enemy is
a task, not the \textbf{story obstacle}. For example, if an
\textbf{ally} has been accused of treason by the King, your goal could
be to prove the \textbf{ally's} innocence. The power of the King
threatening your \textbf{ally} is a \textbf{story obstacle} to be
overcome, and a trial by combat could be a \textbf{contest} to resolve
the conflict with an \textbf{ability} like ``Knight Errant.''

In a court trial, your goal is likely a particular verdict, while the
\textbf{story obstacle} might be the opposing lawyer, an unjust law, or
even the justice system itself. In this case, jury selection, a closing
argument, revelatory evidence, or legal procedural challenges are tasks,
not the entire \textbf{story obstacle}. The overall conflict encompasses
all those things.

A conflict to overcome a \textbf{story obstacle} moves the story forward
when it is resolved. If it is merely a step toward resolving a
\textbf{story obstacle} it is a task and not a conflict. While those
component tasks may be interesting parts of narrating \textbf{tactics}
and \textbf{results}, your GM should be sure to look for the
\textbf{story obstacle} in conflict when framing a \textbf{contest}.

If there is no \textbf{story obstacle} to your actions, your GM should
not call for a \textbf{contest} but simply let you narrate what you do,
provided that seems credible.

For example, you are traveling from one star system to another. In the
next star system you hope to confront the aged rebel who holds
long-forgotten secrets that could bring freedom to the galaxy. Your GM
feels there is no useful \textbf{story obstacle} for you to
\textbf{contest} against, and so lets you describe heading down to the
spaceport to secure a ship, meeting the captain and crew of your vessel,
and traveling to the next world. Your GM encourages you to summarize
what happens quickly so you can get to the meeting with the old rebel.
Your GM knows that will be the real \textbf{story obstacle}, convincing
the old rebel to part with their secrets.

\hypertarget{tactics}{%
\paragraph{2.3.2.3 Tactics}\label{tactics}}

You either choose an \textbf{ability} that represents any `key moment'
in overcoming that \textbf{story obstacle}, or a broad \textbf{ability}
that lets you overcome the whole \textbf{story obstacle}. We call this
choosing a \textbf{tactic}.

Your \textbf{tactic} might describe your using an \textbf{ability} that
helps you overcome a task within the \textbf{story obstacle}: sneaking
past the guards, picking the locks, choosing the right jury or skewering
your opponent with your foil. Or, your \textbf{tactics} might describe
using a broad ability like ``Ninja'', ``Lawyer'', or ``Fencer'' to
overcome all those challenges that might form part of the \textbf{story
obstacle}. Either way, if you succeed at that roll, you overcome the
whole \textbf{story obstacle}. Or by failing at that roll, you fail to
overcome the \textbf{story obstacle}, not just fail at one task.

When deciding on your \textbf{tactic}, focus on how your unique
abilities would help you overcome the \textbf{story obstacle}. This as
the ``key moment'' where we focus on your PC. Use this moment to reveal
your PC's strengths to the group.

Your GM will determine if your \textbf{tactic} passes a
\textbf{credibility test}. If you try to jump a 100 meters gap or run
faster than a speeding car, your action is not credible and your GM will
ask you to choose a different \textbf{tactic}.

Credibility depends on the genre, as what is not credible in a gritty
police procedural might be in pulp where you might be able to leap from
a bridge onto a speeding train. If in dispute, your GM should discuss
with the group whether they consider your \textbf{tactic} credible for
the genre.

\textbf{Extraordinary abilities} in some genres give you the capability
to do the incredible. For example in a superhero genre you might fly or
be invulnerable to bullets, in a fantasy genre hurl magical lightning
bolts. A genre pack for the game should help define what incredible
\textbf{tactics} are allowed for that game as part of an
\emph{Extraordinary Powers Framework}.

The GM can narrate the remaining tasks that make sense of the story
depending on your \textbf{success} with that roll, or have them occur
`off-stage' for speed. Think of the way TV or Cinema often cuts to the
key moment of drama in a break-in, over showing us the whole heist from
beginning to end.

\hypertarget{no-repeat-attempts}{%
\paragraph{2.3.2.4 No Repeat Attempts}\label{no-repeat-attempts}}

A \textbf{contest} represents all of your attempts to overcome a
\textbf{story obstacle}. If you lose it means that no matter how many
times you tried to solve the problem, you finally had to give up. You
can try again only if you use a new \textbf{tactic} to overcome the
\textbf{story obstacle}.

\hypertarget{resistance}{%
\subsubsection{2.3.3 Resistance}\label{resistance}}

Your GM chooses a \textbf{resistance} to represent the difficulty of the
\textbf{story obstacle}.

When setting \textbf{resistance}s it is important to understand that
whilst traditional roleplaying games simulate an imaginary reality,
\emph{QuestWorlds} emulates the techniques of fictional storytelling.

Understanding this distinction will help you to play the game in a
natural, seamless manner.

For example, let's say that your GM is playing a game inspired by
fast-paced, non-fantastic, martial arts movies in a contemporary
setting. You are running along a bridge, pacing a hovercraft, piloted by
the main bad guy. You want your character, Joey Chun, to jump onto the
hovercraft and punch the villain's lights out.

In a traditional, simulative game, your GM would determine how hard this
is based on the physical constraints you've already described. In doing
so, they would come up with imaginary numbers and measurements. Your GM
would have to work out the distance between bridge and hovercraft.
Depending on the rules set, they might take into account your relative
speed to the vehicle. Then they would use whatever resolution mechanic
the rules provide them with to see if Joey succeeds or fails. If you
blow it, your GM will probably consult the falling rules to see how
badly you injure yourself (if you land poorly), or the drowning rules,
if you end up in the river.

In \emph{QuestWorlds}, your GM starts not with the physical details, but
with the proposed action's position in the storyline. They consider a
range of narrative factors, from how entertaining it would be for you to
\textbf{succeed}, how much \textbf{failure} would slow the pacing of the
current sequence, and how long it has been since you last scored a
thrilling \textbf{victory}. If, after this, they need further reference
points, your GM can draw inspiration more from martial arts movies than
the physics of real-life jumps from bridges onto moving hovercraft.
Having decided how difficult the task ought to be dramatically,your GM
will then supply the physical details as color, to justify their choice
and create suspension of disbelief, the illusion of authenticity that
makes us accept fictional incidents as credible on their own terms. If
they want Joey to have a high chance of \textbf{success}, your GM
describes the distance between bridge and vehicle as impressive (so it
feels exciting if you make it) but not insurmountable (so it seems
believable if you make it).

In other words, in \emph{QuestWorlds} your GM will pick a
\textbf{resistance} based on dramatic needs and then justify it by
adding details into the story.

\textbf{Resistance} numbers are derived from a \textbf{base resistance},
which is modified according to the \textbf{resistance class}, as per the
following table:

\hypertarget{resistance-class-table}{%
\subsubsection{RESISTANCE CLASS TABLE}\label{resistance-class-table}}

\begin{longtable}[]{@{}cc@{}}
\toprule
Class & Value\tabularnewline
\midrule
\endhead
Extreme & Base +M2\tabularnewline
Huge & Base +M\tabularnewline
Very High & Base +9\tabularnewline
High & Base +6\tabularnewline
Raised & Base +3\tabularnewline
Moderate & Base\tabularnewline
Low & Base -3\tabularnewline
Very Low & Base -6\tabularnewline
Tiny & Base -9 or 6, whichever is lower\tabularnewline
Rock-bottom & Base -M or 6, whichever is lower\tabularnewline
\bottomrule
\end{longtable}

By default, the \textbf{base resistance} starts at 14.

It is often easier to remember that resistances follow the usual range
of incrementing by 3: +3, +6, +9, and then incrementing by masteries
etc. Similarly resistances decrement by 3: -3, -6, -9 and then decrement
by masteries with a floor of -6.

All \textbf{contests} use the base number + \textbf{resistance class},
except for \textbf{contests} to determine \textbf{augments}.

\textbf{Augmenting} always faces a \textbf{moderate resistance}, this is
always the unmodified base value.

\hypertarget{die-rolls}{%
\subsubsection{2.3.4 Die Rolls}\label{die-rolls}}

To determine how well you use an \textbf{ability}, roll a 20-sided die
(D20). At the same time, your GM rolls for the \textbf{resistance}.

Compare your rolled number with the \textbf{TN} to determine the
\textbf{result}, a level of \textbf{success} or \textbf{failure} for the
roll (not the \textbf{contest} as a whole).

\begin{itemize}
\tightlist
\item
  \textbf{Critical}: If the die roll is 1 (even when the \textbf{TN} is
  1), you succeed brilliantly. This is the best \textbf{result}
  possible.
\item
  \textbf{Success}: If the die roll is greater than 1 and less than or
  equal to the \textbf{TN}, you succeed, but there is nothing remarkable
  about the success.
\item
  \textbf{Failure}: If the die roll is greater than the \textbf{TN} but
  not 20, you fail. Things do not happen as hoped.
\item
  \textbf{Fumble}: If the die roll is 20, you fumble (even when the
  \textbf{TN} is 20). You fail miserably. This is the worst
  \textbf{result} possible.
\end{itemize}

Note that whatever your \textbf{result} the \textbf{outcome} will depend
on comparing your roll with your opponents. So you might
*\textbf{succeed}, but still lose the \textbf{prize}. At the same time,
your GM should take into account your \textbf{result} when narrating the
\textbf{outcome}, and not use your incompetence as a reason you failed
to gain the \textbf{prize} if you succeeded, instead focusing on the
\textbf{resistance}'s superiority despite your \textbf{success}.

\hypertarget{outcome}{%
\subsubsection{2.3.5 Outcome}\label{outcome}}

Your roll and that of your GM's roll are compared to determine your
overall \textbf{outcome} which will be either \textbf{victory} or
\textbf{defeat} for the \textbf{contest} as a whole.

If you have a better \textbf{result} than the GM, then you have a
\textbf{victory} and you gain the \textbf{prize} set out when the
\textbf{contest} was framed.

If you have a worse \textbf{result}, then you are \textbf{defeated} and
do not gain the \textbf{prize}.

If you both have the same \textbf{result}, the better roll wins.

If your rolls tie, then it is a standoff.

A \textbf{critical} is a better result than a \textbf{success} which is,
in turn, a better result than a \textbf{failure}, which is a better
result than a \textbf{fumble}.

Your GM describes what happens, based on their interpretation of the
\textbf{outcome}.

\hypertarget{better-roll}{%
\paragraph{2.3.5.1 Better Roll}\label{better-roll}}

\emph{QuestWorlds} supports two options for the ``better roll'': the
highest roll, or the lowest roll. Some groups prefer lowest roll, some
higher. The preference toward ``low is better'', is often because 1 is a
\textbf{critical} and 20 a \textbf{fumble}, and toward ``high is
better'' because the winner has rolled a higher number. Groups wanting
higher abilities to win out slightly more often should use higher roll.

We also use the phrase ``worse roll'' to indicate the losing roll.

\hypertarget{confusing-ties}{%
\paragraph{2.3.5.2 Confusing Ties}\label{confusing-ties}}

Your GM will describe most tied \textbf{outcomes} as inconclusive
standoffs, in which neither of you gets what you wanted.

In some situations, ties become difficult to visualize. Chief among
these are \textbf{contest}s with binary \textbf{outcomes}, where only
two possible results are conceivable.

Your GM can either change the situation on such a tie, introducing a new
element that likely renders the original \textbf{prize} irrelevant to
both participants, or they can resolve the ties in your favor as a
\textbf{marginal victory}.

\hypertarget{bumps}{%
\paragraph{2.3.5.3 Bumps}\label{bumps}}

A \textbf{bump} affects the degree of \textbf{success} or
\textbf{failure} of your die roll. A \textbf{bump} up improves your
\textbf{result} by one step, changing a \textbf{fumble} to a
\textbf{failure}, a \textbf{failure} to a \textbf{success}, or a
\textbf{success} to a \textbf{critical}. \textbf{Bump} ups come from two
sources: \textbf{masteries} and \textbf{hero points} (applied in that
order). A \textbf{bump} down reduces \textbf{result} by one step,
changing a \textbf{critical} to a \textbf{success}, a \textbf{success}
to a \textbf{failure}, or a \textbf{failure} to a \textbf{fumble}.
\textbf{Bump} downs come from one source: \textbf{masteries}.

\textbf{Bumps} always affect \textbf{results} not \textbf{outcomes},
although the outcome could change as a effect of gaining a different
\textbf{result}.

\hypertarget{bump-up-with-mastery}{%
\paragraph{2.3.5.4 Bump Up with Mastery}\label{bump-up-with-mastery}}

If you're engaged in a \textbf{contest} against a \textbf{resistance},
and you have an \textbf{ability} of 10M versus a \textbf{resistance} of
10, you enjoy an advantage. You get a \textbf{bump} to your die roll
from that \textbf{mastery}.

You get one \textbf{bump} up for each level of \textbf{mastery} your PC
has greater than your opponent's. So against a \textbf{resistance} of 14
a PC's \textbf{ability} of 7M is treated as 7 vs.~14 but we
\textbf{bump} the \textbf{result} one step in the favor of the PC; a
PC's \textbf{ability} of 3M2 is treated as 3 vs.~14 but we \textbf{bump}
the \textbf{result} two steps in favor of the PC.

This reflects the fact that an \textbf{ability} above 20 would always
succeed on a D20. Because each \textbf{mastery} represents
\textbf{automatic success} (apart from a \textbf{fumble}) on a D20, you
roll against the remainder, and treat the \textbf{mastery} as a
\textbf{bump}. So on an \textbf{ability} of 27 is 7M, which means a
\textbf{target number} of 7 and \textbf{bump} the \textbf{result}; an
\textbf{ability} of 43 means 3M2 or a \textbf{target number} of 3 and
\textbf{bump} the \textbf{result} twice.

Opposed \textbf{masteries} cancel out, each contestant reducing their
\textbf{rating} by the same number of \textbf{masteries} until only one
or neither of them has \textbf{masteries}. If you have two
\textbf{masteries}, then you enjoy the same great advantage over an
opponent with a single \textbf{mastery} as someone with one
\textbf{mastery} has over an opponent with no \textbf{masteries}. If you
have an advantage of two or more \textbf{masteries} over an opponent,
you can pretty much count on pounding them into the dust.

This allows \emph{QuestWorlds} to represent large differences in ability
or \textbf{resistance}.

\hypertarget{bump-up-with-hero-points}{%
\paragraph{2.3.5.5 Bump Up with Hero
Points}\label{bump-up-with-hero-points}}

You can spend a \textbf{hero point} to \textbf{bump} up any
\textbf{result} by one step. You may only \textbf{bump} your own rolls,
not those of other PCs or \textbf{supporting characters}---with the
exception of \textbf{sidekicks} and \textbf{retainers}, which, as
extensions of your character, you may spend \textbf{hero points} on. You
can decide to use a \textbf{hero point} for a \textbf{bump} after the
die roll \textbf{results} are calculated (including any \textbf{bump}
ups from \textbf{masteries}).

You can only spend one \textbf{hero point} per roll.

\hypertarget{augments}{%
\subsection{2.4 Augments}\label{augments}}

You may sometimes face \textbf{contests} where more than one
\textbf{ability} may be applicable to the conflict at hand. In such
cases, you may attempt to use one \textbf{ability} to give a supporting
bonus to the main ability you are using to frame the \textbf{contest}.
This is called an \textbf{augment}. For example, if your character has
the \textbf{abilities} ``The Queen's Intelligencer'' and ``Master of
Disguise'', you might use the latter to \textbf{augment} the former when
infiltrating a rival nation's capitol. Similarly, a character with
``Knight Errant'' and ``My Word is my Bond'' \textbf{abilities} might
use one to \textbf{augment} the other when in conflict with a
\textbf{story obstacle} the character has sworn to overcome.

Abilities that represent special items, weapons, armor, or other
noteworthy equipment can be a common source of \textbf{augments}.
However, this grows tired if over-used and you should try and restrict
repeated use equipment used in this way to \textbf{contests} where they
are particularly interesting or apropos.

\textbf{Augments} can also come from other characters'
\textbf{abilities} if one character uses an \textbf{ability} to support
another's efforts rather than directly engaging in the \textbf{contest}.
\textbf{Augments} can even come from outside resources like support from
a community, see §8, or other circumstantial help.

If you have a good idea for an \textbf{augment}, propose it to your GM
while the \textbf{contest} is being framed. When making your proposal,
describe how the \textbf{augmenting ability} supports the main one in a
way that is both \emph{entertaining} and \emph{memorable}. Don't just
hunt for mechanical advantage, show your group more about your PC when
you \textbf{augment}, their attitudes, passions, or lesser known
\textbf{abilities}. If you are \textbf{augmenting} with a \textbf{broad
ability} like ``Fool's Luck'', be prepared to describe the unlikely
events that tilt the scales in your favor. Your GM will decide whether
the \textbf{augment} is justified and can refuse boring and uninspired
attempts to \textbf{augment}, where you are just looking for a bonus to
your roll and not adding to the story.

You may only use one of your own \textbf{abilities} to \textbf{augment}
the \textbf{ability} you are using in the \textbf{contest}, and you may
not use an \textbf{ability} to \textbf{augment} itself. However,
\textbf{augments} from other players supporting you can add together
with your own, along with other \textbf{modifiers}, including those from
\textbf{benefits of victory} and from \textbf{plot augments}.

If you GM accepts your \textbf{augment} proposal, it will be resolved by
one of the methods below. The main \textbf{contest} then proceeds as
normal, with any bonus from the \textbf{augment} added onto the
\textbf{rating} of the \textbf{ability} chosen when \textbf{framing the
contest}. The \textbf{augment} remains in effect for the duration of the
\textbf{contest}.

\hypertarget{rolled-augments}{%
\subsubsection{2.4.1 Rolled Augments}\label{rolled-augments}}

To grant an \textbf{augment} to yourself, or another PC, in an upcoming
\textbf{contest}, you engage in a \textbf{simple contest} against a
\textbf{moderate resistance} before the main contest begins to determine
whether the \textbf{augment} attempt results in an advantage. Frame this
\textbf{augment contest} with your GM, making it clear how your
supporting \textbf{ability} will achieve the goal of making your
\textbf{ability} in the main \textbf{contest} more effective.
\textbf{Augment contests} may not themselves be \textbf{augmented}, and
if your \textbf{augment} attempt ends in \textbf{defeat}, you may not
make another attempt at an \textbf{augment} for the main
\textbf{contest}.

If you are victorious in the \textbf{augment contest}, your GM will
award a bonus of +3 to the \textbf{ability} used in the main
\textbf{contest}. If your description of how you were using the
\textbf{augmenting ability} was particularly entertaining, your GM may
increase the bonus to +6.

If you are using the advanced mechanic for \textbf{degrees of victory or
defeat}, use the table below to interpret the \textbf{outcome} of the
\textbf{simple contest}. Note that \textbf{penalties} for
\textbf{defeat} when attempting an \textbf{augment} are much lessened
compared to a regular \textbf{contest}.

\hypertarget{augment-table}{%
\subsubsection{AUGMENT TABLE}\label{augment-table}}

\begin{longtable}[]{@{}cc@{}}
\toprule
Contest Outcome & Modifier\tabularnewline
\midrule
\endhead
Complete Victory & +M\tabularnewline
Major Victory & +9\tabularnewline
Minor Victory & +6\tabularnewline
Marginal Victory & +3\tabularnewline
Marginal Defeat & 0\tabularnewline
Minor Defeat & 0\tabularnewline
Major Defeat & 0\tabularnewline
Complete Defeat & -3\tabularnewline
\bottomrule
\end{longtable}

\hypertarget{quick-augments}{%
\subsubsection{2.4.2 Quick Augments}\label{quick-augments}}

While an \textbf{augment contest} before the main \textbf{contest} can
be dramatic and exciting, it does also slow down the flow of play. An
alternative approach, called a \textbf{quick augment}, alleviates this
by replacing rolling with a bonus equal to one-fifth of the
\textbf{augmenting rating}. If your GM accepts your proposal for an
\textbf{augment}, simply divide the \textbf{augmenting ability rating}
by five, round down if necessary, and add the modifier as a bonus to
your \textbf{ability} in the upcoming \textbf{contest}.

It is at your GM's discretion whether to roll \textbf{augments} or use
\textbf{quick augments}. Your GM may choose one preferred approach and
use it in every instance, or choose one or the other on a case-by-case
basis based on interest and flow of play.

\hypertarget{plot-augments}{%
\subsubsection{2.4.3 Plot Augments}\label{plot-augments}}

A plot \textbf{augment} occurs when your GM decides that your
\textbf{victory} over a previous \textbf{story obstacle} is sufficiently
relevant to the \textbf{contest} at hand as to warrant a bonus much like
a supporting \textbf{ability} can. The previous \textbf{contest} need
not have been immediately prior to the current one, as long as its
\textbf{outcome} remains relevant. The magnitude of a plot
\textbf{augment} can be +3, +6, +9, or +M. The more challenging the
previous \textbf{contest} was and the more relevant that
\textbf{victory} is to the current challenge, the higher the
\textbf{augment}.

\hypertarget{advanced-mechanics}{%
\subsection{2.5 Advanced Mechanics}\label{advanced-mechanics}}

You can play \emph{QuestWorlds} with just the basic mechanics and we
recommend you start with this simpler set of rules whilst your GM
masters the less-familiar ideas like \textbf{story obstacles} instead of
task resolution, that make \emph{QuestWorlds} different from traditional
RPGs.

The advanced mechanics add detail to the Basic mechanics, which add
detail, or generate in-game story developments, at the cost of more
complexity.

\hypertarget{resolution-methods-1}{%
\subsection{2.6 Resolution Methods}\label{resolution-methods-1}}

\hypertarget{long-contest}{%
\subsubsection{2.6.1 Long Contest}\label{long-contest}}

Your GM should resolve most conflicts using the \textbf{simple contest}
rules. However, every so often, your GM may want to stretch out the
resolution, breaking it down into a series of actions.

An advanced rule, \textbf{long contests}, lets you zoom-in to task-level
in \textbf{contests} to focus on your character struggling with a series
of tasks in order to overcome the \textbf{story obstacle}, such as
blow-by-blow exchanges with the king's champion in front of the tower
where your father is held hostage, or exposing the lies of a witness in
the court room to free a wrongly accused innocent man. Your GM should do
this when they want to give you a chance to make your PC shine, by
focusing in on how they handle themselves in a difficult situation, and
giving them a chance to show their range of abilities.

As a rule of thumb, if you want resolution that moves closer to the task
level, a \textbf{long contest} may be appropriate.

A sequence of die rolls, between one or more PCs and one or more
\textbf{supporting characters}, breaks the conflict resolution into a
series of actions.

A \textbf{long contest} trades speed for detail. For the flow of a story
we recommend using \textbf{simple contest}s as much as possible. If you
come from other roleplaying games that have task based
\textbf{contests}, we recommend playing only using \textbf{simple
contests}, until you have mastered the different approach of conflict
based \textbf{contests}, before introducing \textbf{long contests}.

More details on \textbf{long contests} are available in §5.

\hypertarget{escalating-contests}{%
\subsubsection{2.6.2 Escalating Contests}\label{escalating-contests}}

If your GM chooses either the \textbf{scored contests} or
\textbf{chained contests} form of \textbf{long contest} you can use an
option called \textbf{escalating contests}.

\textbf{Escalating contests} allow your GM to switch a \textbf{simple
contest} to a \textbf{long contest}. To do this, simply re-interpret the
\textbf{simple contest outcome} as the first \textbf{round} of a
\textbf{scored contest} or \textbf{chained contest}. For example, in a
\textbf{contest} of magic at the Thaumaturgical Academy, Billiard, your
PC, gains a \textbf{marginal victory} over their opponent Crowsky. The
GM narrates the \textbf{outcome}, but although you get what you agreed,
you want Crowsky to be humiliated. Your GM agrees to \textbf{escalating
contest}, treating the \textbf{outcome} as the first \textbf{round} of a
\textbf{scored contest}, and scores 1 \textbf{RP} against Crowsky before
beginning adjudication with the second \textbf{round}.

\textbf{Escalating contests} can serve three functions:

\begin{itemize}
\tightlist
\item
  If you take part in a \textbf{contest} and after hearing the narrative
  feel aggrieved that you could not bring more \textbf{abilities} to
  bear, and the GM agrees that it would be interesting to let you shine
  for a moment, an escalating \textbf{contest} helps you bring more
  \textbf{abilities} into the \textbf{contest}.
\item
  If you take part in a \textbf{contest} but the degree of
  \textbf{success} was not emotionally satisfying, an escalating
  \textbf{contest} gives you another chance to achieve the
  \textbf{outcome} you want.
\item
  If your GM is unsure if a \textbf{simple or long contest} is
  appropriate then they can default to a \textbf{simple contest} and
  switch to \textbf{escalating} if required.
\end{itemize}

If you wish to use \textbf{escalating contests}, you need to choose
\textbf{scored contests} as your \textbf{long contest} option.

\hypertarget{resistance-progression}{%
\subsection{2.7 Resistance Progression}\label{resistance-progression}}

Your GM may decide that \textbf{resistance} to your actions gets harder,
as the campaign progresses. This reflects the trope of the type of
challenges you face getting tougher as you improve.

Your GM should adopt a strategy that mimics a TV show where the
\textbf{resistance} does not increase during a season of the show,
allowing our protagonists to get more competent as the show progresses
towards its climax. In the next season though the \textbf{resistance}
usually goes up, and the writers reflect this with more challenging
opposition in the new season of the show. At the same time, the
opposition that was tough in the first season, now become mooks that can
be easily dispatched to show the increased competence of the
protagonists.

In that case your GM should increment the \textbf{resistance} by +3, +6
or +9 for the next campaign you play with the same characters. The size
of the change should reflect the increase in your previous
\textbf{abilities} in the last campaign. For example, if in the last
season you increased your \textbf{occupation keyword} by +6, your GM may
decide to increase the \textbf{resistance} by +3 or +6 to reflect the
more challenging opposition in the new campaign. Your GM should also
take into account that the opposition you were improving with respect to
in the previous season should now be considered more-easily defeated
mooks, and use lower \textbf{ratings} for them when they appear in the
story.

\hypertarget{no-progression}{%
\subsubsection{2.7.1 No Progression}\label{no-progression}}

Your GM may also decide that the \textbf{resistances} do not get harder
as the campaign progresses, reflecting the PCs \textbf{ability} to
disregard minor challenges, and simply choose harder
\textbf{resistances} to challenge the players.

\hypertarget{degree-of-victory-or-defeat}{%
\subsection{2.8 Degree of Victory or
Defeat}\label{degree-of-victory-or-defeat}}

Often all you need to know to interpret the \textbf{outcome} of a
resolution is whether you gained \textbf{victory} or suffered a
\textbf{defeat}.

Sometimes, you'll want to know how great a \textbf{victory} you won, or
how bad a \textbf{defeat} you endured. This may be important in
providing \textbf{consequences or benefits} that drive further story.

All of the resolution methods have an option to yield the \textbf{Degree
of Victor or Defeat} for the PC. The possible \textbf{Degree of Victory
or Defeat}, from least to greatest, are: \textbf{marginal},
\textbf{minor}, \textbf{major}, \textbf{complete}. \textbf{Ties} are
also possible.

If you struggle against NPCs or abstract forces, the interpretation of
the \textbf{outcome} reveals whether you overcome the \textbf{story
obstacle}, and any \textbf{consequences or benefits}; your GM narrates
the fate of the NPCs or other forces depending on what makes sense.
However, when you and another PC engage in a \textbf{contest} then a
\textbf{victory} for one contestant means a corresponding
\textbf{defeat} for the loser.

So whilst in a PC vs.~PC duel the PC would only be killed on a
\textbf{complete defeat}, an NPC, described as a \textbf{resistance},
might be killed on any \textbf{victory}, depending on how the
\textbf{contest} was framed.

\textbf{Tie}: Tie means no \textbf{outcome}. Effort was expended, but
the net \textbf{outcome} is that nothing consequential occurs, or else
both sides lose or gain equally. If this is confusing, and you are not
contending with another PC, your GM can rule that you gain a
\textbf{marginal victory}.

\textbf{Marginal Victory}: Yes, but\ldots~You get what you want, the
\textbf{prize}, but there are complications, the effect is more limited
than you desired, or you have to make a hard choice between benefits
or~accept a loss to get one

\textbf{Minor Victory}: Yes\ldots{} You get exactly what they want
i.e.~whatever was the \textbf{prize} in the \textbf{contest}.

\textbf{Major \& Complete Victory}: Yes, and\ldots{} You get the
\textbf{prize}, and something else. You gain something, stealing
a~possession, gaining a new \textbf{follower}, or become renowned in
song. If you want to distinguish a \textbf{complete} the effect is often
permanent and no new \textbf{contests} should be framed for this
\textbf{story obstacle}.

\textbf{Marginal Defeat}: No, but\ldots~You don't get what you want, you
lose the \textbf{prize}, but it's not a total loss. You are able to
salvage something from the \textbf{defeat}, a little~more if you
sacrifice something other than the \textbf{prize} to your opponent,~that
the opponent agrees to take instead.

\textbf{Minor Defeat}: No\ldots{} You don't get what you want, you lose
the \textbf{prize}. Any consequences or complications such as injury or
loss of influence are short term and easily shrugged off. Just take
the~loss and rest up.

\textbf{Major \& Complete Defeat}: No and\ldots~You don't get what you
want, you lose the \textbf{prize}, and~there are long-term consequences.
The situation might grow worse or more complicated or you might suffer
adverse consequences~that will require other conflicts to resolve: an
injury that~needs a healer, an insult that~requires a formal apology, a
loss of influence with the community that requires a triumph to win
their trust again etc. You might be dead, or as good as. The
\textbf{prize} is likely lost to you permanently. Or perhaps you lose
something, an item is taken from you, a \textbf{follower} deserts you,
your reputation lies in ruins as poets mock your defeat. If you want to
distinguish, a \textbf{complete} should be bigger loss than a
\textbf{major}, but you can often ignore this distinction.

Your GM will use the \textbf{degree of success} to determine any
\textbf{benefits and consequences}, but be sure to describe the
\textbf{success} in narrative terms.

If you are using a \textbf{stretch}, see §2.12.1, then \textbf{major or
complete victories} you obtain are instead treated as \textbf{minor
victories}.

\hypertarget{benefits-and-consequences}{%
\subsection{2.9 Benefits and
Consequences}\label{benefits-and-consequences}}

\textbf{Contests}, in addition to deciding whether you overcome a
\textbf{story obstacle}, carry additional \textbf{consequences}. These
are negative if you lose, and positive if you win.

Your GM may simply determine these from what makes fictional sense,
given the agreed \textbf{prize} for the \textbf{contest}, as described
above. Optionally your GM may impose \textbf{consequences of defeat} or
provide \textbf{benefits of victory} if they desire ongoing
\textbf{penalties} or \textbf{bonuses}. This rule is used in conjunction
with \textbf{degree of victory or defeat}. Your GM should always respond
to the flow of the story, if narrative consequences are enough, they
should not reach for additional mechanical \textbf{penalties}.

\hypertarget{the-consequences-of-defeat}{%
\subsubsection{2.9.1 The Consequences of
Defeat}\label{the-consequences-of-defeat}}

When you lose a \textbf{contest}, you may suffer \textbf{consequences}:
literal or metaphorical injuries which make it harder for you to use
related \textbf{abilities}.

\begin{itemize}
\tightlist
\item
  In a fight or test of physical mettle, you wind up literally wounded.
\item
  In a social contest, you suffer damage to your reputation.
\item
  If commanding a war, you lose battalions, equipment, or territories.
\item
  In an economic struggle, you lose money, other resources, or
  opportunities.
\item
  In a morale crisis, you may suffer bouts of crippling self-doubt.
\end{itemize}

From the least to the most punishing, the five \textbf{states of
adversity} are: \textbf{hurt}, \textbf{impaired}, \textbf{injured},
\textbf{dying}, and \textbf{dead}. The first four are possible
\textbf{consequences} of any \textbf{contest}. \textbf{Dying} PCs become
\textbf{dead}, unless they receive intervention of some sort.

Although the levels refer to physical \textbf{states of adversity}, the
consequences can be emotional, social, spiritual, magical, and so on.

\hypertarget{hurt}{%
\paragraph{2.9.1.2 Hurt}\label{hurt}}

If you are \textbf{hurt}, you show signs of adversity and find it harder
to succeed at \textbf{contests} related to your \textbf{defeat}. Either
your flesh or pride may be bruised. Until you recover, you suffer a --3
\textbf{penalty} to all related \textbf{abilities}.

You may suffer multiple \textbf{hurts} to the same \textbf{ability}.
These are cumulative until recovery occurs.

Unless your GM has a dramatic reason to decide otherwise, your
\textbf{hurts} vanish at the end of a session, after one day of rest per
accumulated \textbf{hurt}, or when in-game events justify their removal.

\hypertarget{impaired}{%
\paragraph{2.9.1.3 Impaired}\label{impaired}}

If you are \textbf{impaired}, you have taken a jarring blow, physically,
socially, or emotionally, and you are much likelier to fail when
attempting similar actions in the future. You suffer a --6
\textbf{penalty} to all related \textbf{abilities}. Impairments combine
with \textbf{hurts} and with other impairments.

As bad as your condition may be, there's nothing wrong with you that
some prolonged inactivity won't fix. A single \textbf{impairment} goes
away after one week of rest, or when an in-game event (like miraculous
or extraordinary treatment) occurs to make their removal seem
believable.

\hypertarget{injured}{%
\paragraph{2.9.1.4 Injured}\label{injured}}

If you are \textbf{injured}, you have suffered a debilitating blow which
leaves you reeling. Physically you may have lost the use of a limb or
sense, socially you may be shunned, and emotionally you may in shock.
Although you should heal with time, you suffer a -9 \textbf{penalty} to
all related \textbf{abilities}. Injuries combine with impairments and
\textbf{hurts}.

A single \textbf{injury} goes away after a month's rest, or by
miraculous intervention, as above.

\hypertarget{dying}{%
\paragraph{2.9.1.5 Dying}\label{dying}}

If you are \textbf{dying} you will, without rapid and appropriate
intervention, expire. To save you, the other PCs must overcome a
\textbf{story obstacle}. Their attempt must be credible, using medicine
or magic, as defined by your genre. Your GM should use a \textbf{very
high resistance} for this \textbf{contest}, unless the story suggests
otherwise. According to the conventions of dramatic storytelling, the
character typically has just enough time left for the other characters
to make this one attempt.

Successful intervention leaves the PC \textbf{injured}. Depending on the
narrative circumstances, a \textbf{complete victory} on the intervention
attempt may leave them merely \textbf{impaired}.

If intervention fails, you will die, but not necessarily immediately.
Although irrevocably doomed, your GM may rule that the story suggests
that you survive long enough to take one final, heroic, action.

To even take that \textbf{final action} if the GM offers you the chance,
then you must succeed at a prior \textbf{contest of wherewithal} to
rouse yourself to action. Appropriate abilities for the \textbf{contest
of wherewithal} include:

\begin{itemize}
\tightlist
\item
  Physical action: Endurance, High Pain Threshold, Grim Determination,
  etc.
\item
  Intellectual activity: Concentration, Iron Will, Love of Country (if
  action to be attempted is patriotic), etc.
\item
  Social humiliation: Savoir Faire, Unflappable, Stoic Dignity
\end{itemize}

A \textbf{contest of wherewithal} faces a \textbf{moderate resistance}.
Even if you succeed at the \textbf{contest of wherewithal}, you take an
automatic \textbf{bump} down \textbf{penalty} whenever you use any
related \textbf{ability} in a \textbf{contest}. (The \textbf{bump} down
does not apply to the \textbf{contest of wherewithal} itself.) Where it
seems apt, your GM may choose to ignore the \textbf{bump} down if you
score a \textbf{major or complete victory} on the \textbf{contest of
wherewithal}.)

Any active \textbf{hurts} or \textbf{impairments} continue to be counted
against you as well.

Your \textbf{final action} cannot reverse the \textbf{outcome} of the
\textbf{contest} that you lost, it must involve a new \textbf{story
obstacle}. Your GM will rule if your action is allowable.

Like other \textbf{states of adversity}, \textbf{dying} may be literal
or metaphorical. Your standing in society, business or politics may be
on the brink of permanent extinction. You may be facing mental death ---
a permanent lapse into madness or senility.

\hypertarget{dead}{%
\paragraph{2.9.1.6 Dead}\label{dead}}

If you die as a consequence of physical injuries, you are gone from the
game, period.

Death from a non-physical \textbf{contest} will likely be metaphorical.
If you die in an economic, social, spiritual, or artistic
\textbf{contest}, you permanently lose abilities.

Even only metaphorically dead, your GM may declare that you have
undergone changes so dire as to make your PC unplayable. You may be
incurably insane, or be so socially shamed that you retire to a life of
obscurity or religious meditation. You may be shunned by all around you,
sent into permanent exile, or sentenced to long-term imprisonment with
no hope of escape.

\hypertarget{consequences-of-defeat-table}{%
\subsubsection{CONSEQUENCES OF DEFEAT
TABLE}\label{consequences-of-defeat-table}}

\begin{longtable}[]{@{}ccc@{}}
\toprule
\begin{minipage}[b]{0.09\columnwidth}\centering
Defeat Level\strut
\end{minipage} & \begin{minipage}[b]{0.13\columnwidth}\centering
State of Adversity\strut
\end{minipage} & \begin{minipage}[b]{0.70\columnwidth}\centering
Penalty\strut
\end{minipage}\tabularnewline
\midrule
\endhead
\begin{minipage}[t]{0.09\columnwidth}\centering
Marginal\strut
\end{minipage} & \begin{minipage}[t]{0.13\columnwidth}\centering
Hurt\strut
\end{minipage} & \begin{minipage}[t]{0.70\columnwidth}\centering
--3 penalty to appropriate abilities\strut
\end{minipage}\tabularnewline
\begin{minipage}[t]{0.09\columnwidth}\centering
Minor\strut
\end{minipage} & \begin{minipage}[t]{0.13\columnwidth}\centering
Impaired\strut
\end{minipage} & \begin{minipage}[t]{0.70\columnwidth}\centering
--6 penalty to appropriate abilities\strut
\end{minipage}\tabularnewline
\begin{minipage}[t]{0.09\columnwidth}\centering
Major\strut
\end{minipage} & \begin{minipage}[t]{0.13\columnwidth}\centering
injured\strut
\end{minipage} & \begin{minipage}[t]{0.70\columnwidth}\centering
--9 penalty to appropriate abilities\strut
\end{minipage}\tabularnewline
\begin{minipage}[t]{0.09\columnwidth}\centering
Complete\strut
\end{minipage} & \begin{minipage}[t]{0.13\columnwidth}\centering
Dying\strut
\end{minipage} & \begin{minipage}[t]{0.70\columnwidth}\centering
No actions allowed. If `final action', automatic \textbf{bump} down on
uses of appropriate \textbf{ability}\strut
\end{minipage}\tabularnewline
\bottomrule
\end{longtable}

\hypertarget{benefits-of-victory}{%
\subsubsection{2.9.2 Benefits of Victory}\label{benefits-of-victory}}

Just as when you experience \textbf{defeat} you can suffer ongoing ill
effects in addition to the loss of the \textbf{prize} at hand, when you
win you can gain benefits from that \textbf{victory}.

A \textbf{benefit of victory} gives you a bonus on the selected
\textbf{abilities}, or in the specified situation, as determined by your
\textbf{victory} level.

\begin{itemize}
\tightlist
\item
  In a fight or test of physical mettle, your workout leaves you sharp
  for the next encounter.
\item
  In a social contest, you gain confidence and admiration from your
  triumph.
\item
  If commanding a war, you gain strategic advantage over your enemy.
\item
  In an economic struggle, your profits can be re-invested, or you drive
  competitors into the ground.
\item
  In a morale crisis, you are buoyed up by success, nothing can stop you
  now.
\end{itemize}

Remember that the \textbf{benefit} does not have to be directly related
to the \textbf{ability} used. Look to the goal of the \textbf{contest}.
The abilities or situation should reflect the \textbf{story obstacle}
that was overcome and the \textbf{tactic} used to overcome it.

\begin{itemize}
\tightlist
\item
  In a fight or test of physical mettle, your triumph has everyone
  rallying to your cause.
\item
  In a social contest, you win powerful \textbf{allies} who will
  strengthen you in your fight against your enemies.
\item
  If commanding a war, you pillage the enemy city and enrich your army.
\item
  In an economic struggle, you gain status as one of the wealthy elite.
\item
  In a morale crisis, your rallied troops strengthen your army.
\end{itemize}

A PC may apply \textbf{bonuses} from multiple \textbf{benefits} to a
single \textbf{contest}.

From the least to the most robust the four \textbf{states of fortune}
are: \textbf{fresh}, \textbf{pumped}, \textbf{invigorated}, and
\textbf{heroic}.

\hypertarget{fresh}{%
\paragraph{2.9.2.1 Fresh}\label{fresh}}

If you are \textbf{fresh}, you are lively and find it easier to succeed
at \textbf{contests} related to your \textbf{victory}. You are on a roll
and feel confident and able. Until you are \textbf{defeated}, you gain a
+3 \textbf{bonus} to all related abilities.

You may be refreshed multiple times on the same \textbf{ability}. These
are cumulative until \textbf{defeat} occurs.

Unless your GM has a dramatic reason to decide otherwise, your
\textbf{freshness} vanishes at the end of a session, after one day of
idleness, or when in-game events justify their removal.

\hypertarget{pumped}{%
\paragraph{2.9.1.3 Pumped}\label{pumped}}

If you are \textbf{pumped}, you are energized, physically, socially, or
emotionally, and you are much likelier to succeed when attempting
similar actions in the future. You gain a +6 \textbf{bonus} to all
related abilities. \textbf{Pumped} combines with \textbf{fresh} and
\textbf{pumped}.

As good as your condition may be, an extended period of idleness will
cause you to lose your edge. A single \textbf{pumped} goes away after
one week of idleness, or when an in-game event (like long drunken party)
occurs to make their removal seem believable.

\hypertarget{invigorated}{%
\paragraph{2.9.1.4 Invigorated}\label{invigorated}}

If you are \textbf{invigorated}, you are pulsing with hormones, mentally
focused, or exuding confidence. Physically you can push your body to new
personal bests of achievement, socially confidant and exuding charisma,
and emotionally you are in touch with your feelings and resonate with
those of others. Although this will fade with time, you gain a +9
\textbf{bonus} to all related abilities. \textbf{Invigorated} combines
with \textbf{pumped} and \textbf{fresh}.

Being \textbf{invigorated} goes away after a month's idleness, or an
in-game event, as above.

\hypertarget{heroic}{%
\paragraph{2.9.1.5 Heroic}\label{heroic}}

If you are \textbf{heroic}, you have become unstoppable, physically at
peak performance, socially, everyone wants to be you or be with you, and
emotionally you have gained new insights into yourself and others around
you. Although this will fade with time, you gain a \textbf{bump}
\textbf{bonus} to all related abilities. Being \textbf{heroic} combines
with \textbf{invigorated}, \textbf{pumped} and \textbf{fresh}.

Being \textbf{heroic} goes away after a season's idleness, or an in-game
event, as above.

\hypertarget{benefits-of-victory-table}{%
\subsubsection{BENEFITS OF VICTORY
TABLE}\label{benefits-of-victory-table}}

\begin{longtable}[]{@{}ccc@{}}
\toprule
\begin{minipage}[b]{0.13\columnwidth}\centering
Victory Level\strut
\end{minipage} & \begin{minipage}[b]{0.14\columnwidth}\centering
State of Fortune\strut
\end{minipage} & \begin{minipage}[b]{0.64\columnwidth}\centering
Benefit\strut
\end{minipage}\tabularnewline
\midrule
\endhead
\begin{minipage}[t]{0.13\columnwidth}\centering
Marginal\strut
\end{minipage} & \begin{minipage}[t]{0.14\columnwidth}\centering
Fresh\strut
\end{minipage} & \begin{minipage}[t]{0.64\columnwidth}\centering
+3\strut
\end{minipage}\tabularnewline
\begin{minipage}[t]{0.13\columnwidth}\centering
Minor\strut
\end{minipage} & \begin{minipage}[t]{0.14\columnwidth}\centering
Pumped\strut
\end{minipage} & \begin{minipage}[t]{0.64\columnwidth}\centering
+6\strut
\end{minipage}\tabularnewline
\begin{minipage}[t]{0.13\columnwidth}\centering
Major\strut
\end{minipage} & \begin{minipage}[t]{0.14\columnwidth}\centering
Invigorated\strut
\end{minipage} & \begin{minipage}[t]{0.64\columnwidth}\centering
+9\strut
\end{minipage}\tabularnewline
\begin{minipage}[t]{0.13\columnwidth}\centering
Complete\strut
\end{minipage} & \begin{minipage}[t]{0.14\columnwidth}\centering
Heroic\strut
\end{minipage} & \begin{minipage}[t]{0.64\columnwidth}\centering
You receive an automatic \textbf{bump} up on uses of an appropriate
\textbf{ability}\strut
\end{minipage}\tabularnewline
\bottomrule
\end{longtable}

\hypertarget{clearly-inferior-opponents}{%
\paragraph{2.9.2.1 Clearly Inferior
Opponents}\label{clearly-inferior-opponents}}

Defeating clearly inferior opponents neither teaches you anything nor
significantly enhances your reputation; you are ineligible for a
\textbf{benefit of victory} if the \textbf{resistance} you used in the
\textbf{contest} exceeded the \textbf{resistance} by 6 or more. If, in
the case of a \textbf{long contest}, you or your opponent used multiple
\textbf{abilities}, compare the best \textbf{ability} you used to their
worst.

\hypertarget{recovery-and-healing}{%
\subsubsection{2.9.3 Recovery and Healing}\label{recovery-and-healing}}

Consequences of \textbf{injured} or less lapse on their own with the
passage of time. However, you'll often want to remove them ahead of
schedule, with the use of \textbf{abilities}.

\hypertarget{healing-abilities}{%
\paragraph{2.9.3.1 Healing Abilities}\label{healing-abilities}}

The \textbf{ability} used to bring about recovery from a \textbf{state
of adversity} must relate to the type of harm.

You can heal physical injuries with medical or extraordinary
\textbf{abilities}.

You can remove mental traumas, including those of confidence and morale,
with mundane psychology or through \textbf{extraordinary abilities}. You
might also remove them through a dramatic confrontation between the
victim and the source of the psychic injury.

You use social abilities to heal social injuries. You probably have to
make a public apology of some sort, often including a negotiation with
the offended parties and the payment of compensation, either in
disposable wealth or something more symbolic.

You can fix damage to items and equipment with some sort of repair
\textbf{ability}. If you want to fix an extraordinary item, you may
require genre-specific expertise: a broken magic ring may require a
ritual to reforge.

Your GM should almost always resolve healing attempts as \textbf{simple
contests}. An exception might be a medical drama, in which surgeries
would comprise the suspenseful set-piece sequences of the game, and your
GM might chose a \textbf{long contest}.

\hypertarget{healing-resistances}{%
\paragraph{2.9.3.2 Healing Resistances}\label{healing-resistances}}

Default \textbf{resistances} to remove states of adversity are as
follows:

\hypertarget{healing-resistances-table}{%
\subsubsection{HEALING RESISTANCES
TABLE}\label{healing-resistances-table}}

\begin{longtable}[]{@{}cc@{}}
\toprule
Consequence of Defeat & Difficulty\tabularnewline
\midrule
\endhead
Hurt & Low\tabularnewline
Impaired & Moderate\tabularnewline
Injured & High\tabularnewline
Dying & Very High\tabularnewline
\bottomrule
\end{longtable}

\hypertarget{outcomes-of-healing}{%
\paragraph{2.9.3.3 Outcomes of Healing}\label{outcomes-of-healing}}

When you make a successful healing attempt, you remove one level of
\textbf{adversity} for each level of \textbf{victory}. A \textbf{major
defeat} increases the subject's \textbf{consequences of defeat} by 1; a
\textbf{complete defeat} adds an additional 2 levels to the
\textbf{state of adversity}.

\hypertarget{waning-benefits}{%
\subsubsection{2.9.4 Waning Benefits}\label{waning-benefits}}

Just as you recover from \textbf{consequences} with time, or through
healing, so \textbf{benefits} fade with time.

At the end of a storyline, especially when a significant period of
game-world time passes between the conclusion of one episode and the
beginning of the next, the GM may declare that all \textbf{benefits}
have expired.

\hypertarget{benefits-meet-consequences}{%
\subsubsection{2.9.5 Benefits Meet
Consequences}\label{benefits-meet-consequences}}

Because it is confusing to track both \textbf{benefits and consequences}
against the same \textbf{ability} your GM may simply rule that one
cancels the other out. This is particularly true of social
\textbf{contest}s where a moment of shame can erase your previous
triumphs, or your confidence eroded by a \textbf{failure}. Physical
benefits may cancel out, flushed with victory you may be able to ignore
pain, but it may defy credibility for wounds to be healed by an athletic
performance.

Your GM may simply rule that \textbf{benefits} and \textbf{consequences}
cancel out, or they may take the difference between the two benefits and
create a new one. For example if you are \textbf{invigorated} by your
previous performance in the dance \textbf{contest}, but then suffer a
\textbf{hurt}, your GM may rule that your twisted ankle cancels out your
energy from the last performance, or your GM might rule that your
success sees you through the pain, but you are now only \textbf{pumped}.

\hypertarget{modifiers}{%
\subsection{2.10 Modifiers}\label{modifiers}}

Your \textbf{rating} represents a general \textbf{ability} to succeed in
the narrative, but modifiers reflect specific conditions that may make
it easier or harder to overcome particular \textbf{story obstacles}.
They are applied to your \textbf{ability} to get a final \textbf{target
number} (\textbf{TN}).

Positive modifiers are called \textbf{bonuses}; negative modifiers are
called \textbf{penalties}.

\textbf{Bonuses}, may raise your \textbf{ability} high enough to gain a
\textbf{mastery}, in which case you get the \textbf{bumps} up or down
that a \textbf{mastery} would normally supply.

\textbf{Penalties}, may lower an \textbf{ability} to the point where it
loses one or more \textbf{masteries}. In this case, you lose the
\textbf{bumps} up or down you would normally get.

Your GM should only use modifiers to alter your \textbf{target number}
due to unusual circumstances you helped to create, or have some control
over. If an unusual situation applies to a \textbf{resistance}, the GM
should choose a \textbf{resistance} that reflects that. Modifiers never
apply to the \textbf{resistance}.

If \textbf{penalties} reduce your \textbf{target number} to 0 or less,
any attempt to use it automatically \textbf{results} in
\textbf{failure}. You must find another way to achieve your aim.

\hypertarget{stretches}{%
\subsubsection{2.10.1 Stretches}\label{stretches}}

When you propose an action using an \textbf{ability} that seems
completely inappropriate, your GM rules it impossible. If you went ahead
and tried it anyway, you'd automatically fail---but you won't, because
that would be silly.

In some cases, though, your proposed match-up of action and
\textbf{ability} is only somewhat implausible. A successful attempt with
it wouldn't completely break the illusion of fictional reality---just
stretch it a bit.

Using a somewhat implausible \textbf{ability} is known as a
\textbf{stretch}. If your GM deems an attempt to be a \textbf{stretch},
the PC suffers a -3, --6, -9 \textbf{penalty}, or a \textbf{bump} down,
to their \textbf{target number}, depending on how incredible the
\textbf{stretch} seems to the GM and other players. Your GM should
\textbf{penalize} players who try to create a `do anything'
\textbf{ability} that they then \textbf{stretch} to gain from raising
fewer \textbf{abilities} in advancement to ensure balance with other
PCs.

A default \textbf{stretch penalty} should be -6.

The definition of \textbf{stretch} is elastic, depending on genre.

Your GM should not impose \textbf{stretch penalties} on action
descriptions that add flavor and variety to a scene, but do not
fundamentally change what you can do with your \textbf{ability}. These
make the scene more fun but don't really gain any advantage.

\hypertarget{situational-modifiers}{%
\subsubsection{2.10.2 Situational
Modifiers}\label{situational-modifiers}}

Your GM may also impose \textbf{modifiers} when, given the description
of the current situation, believability demands that you should face a
notable \textbf{bonus} or \textbf{penalty}. Your GM should choose
\textbf{modifiers} of +6, +3, --3, or --6. \textbf{Modifiers} of less
than 3 don't exert enough effect to be worth the bother. Those higher
than 6 give the situational \textbf{modifier} a disproportionate role in
determining \textbf{outcomes}.

During a \textbf{long contest}, they should typically last for a single
\textbf{round}, and reflect clever or foolish choices.

\hypertarget{combined-abilities}{%
\subsection{2.11 Combined Abilities}\label{combined-abilities}}

On certain occasions your GM may rule that you can only hope to achieve
the \textbf{prize} by using two disparate \textbf{abilities}. When this
occurs, average your two \textbf{ability ratings}, then apply any
modifiers, to arrive at your \textbf{TN}.

Combining your abilities, rather than using the best one and
\textbf{augmenting} it with other, is always a disadvantage. Your GM
should only require combined \textbf{ability} use when story logic
absolutely demands that you face a lower chance of \textbf{success},
because you have to do two things at once.

\hypertarget{mobs-gangs-and-hordes}{%
\subsection{2.12 Mobs, Gangs, and Hordes}\label{mobs-gangs-and-hordes}}

Sometimes you will face large numbers of opponents. Your GM can treat
many as one. Your GM divides the number of opponents by the number of
contesting PCs. Your GM then treats each of these sections of the crowd
as a single opponent with one \textbf{rating}. Their numbers are
factored into the \textbf{rating} your GM assigns to them.

If in doubt, your GM should think of the \textbf{resistance} that would
be dramatically appropriate for a single opponent and then adjust it
with a \textbf{bonus} of +3, +6 or +9 depending on how outnumbered you
are. No more than six foes can typically contend with you in a physical
confrontation, or two in a social one, or they tend to get in each
other's way.

When the mob loses an exchange, your GM describes individuals within it
as being hurt or falling away. When it wins, describe them overwhelming
you, or swelling in numbers.

\hypertarget{ganging-up}{%
\subsection{2.13 Ganging Up}\label{ganging-up}}

Sometimes you may outnumber your opponent. For a \textbf{simple contest}
this can be often be handled by having your PC on point, and the other
PCs \textbf{augmenting} you, breaking the usual rule on multiple
\textbf{augments} if more than one person should be able to support you.
Your GM may only allow as many of you to \textbf{augment} as they feel
would help, not get in each other's way. For a \textbf{group contest,
long or simple} if you enjoy a numerical advantage you should separately
contest against a reduced \textbf{resistance}. Your GM should adjust the
\textbf{resistance} to reflect your superior numerical advantage.

As above, if in doubt, the GM should think of the \textbf{resistance}
that would be dramatically appropriate for a one-on-one confrontation
and then adjust it with a \textbf{penalty} of -3, -6, or -9 depending on
how significantly you outnumber them. As above, note that unless your
opponent is extraordinarily large, you cannot confront them physically
with more than about six people (include \textbf{followers}) or socially
with about two people (again include \textbf{followers}) or people just
get in each other's way.

\hypertarget{mass-effort}{%
\subsection{2.14 Mass Effort}\label{mass-effort}}

Clashes of massive forces resolve like any other \textbf{contest},
\textbf{simple} or \textbf{long}. These include:

\begin{itemize}
\tightlist
\item
  Military engagements
\item
  Corporate struggles for market share • Building competitions
\item
  Efforts to spread a faith or ideology • Dance competitions
\end{itemize}

If you are not participating in the \textbf{contest} and have no stake
in its \textbf{outcome}, then your GM doesn't bother to run a
\textbf{contest}. The GM just chooses an \textbf{outcome} for dramatic
purposes.

Otherwise, your GM will start by determining your degree of influence
over the \textbf{outcome}. They are either:

\begin{itemize}
\tightlist
\item
  Determining factors: The success of the effort depends mostly on your
  choices and successes. For example, you might be a military leader
  facing a force of roughly equal potency. As all else is equal, the
  better general will win the day. In this instance, your
  \textbf{tactic} should be a relevant leadership \textbf{ability}.
\item
  Contributors: One of the forces enjoys a clear advantage over the
  others, but your efforts may tip the balance in favor of a chosen
  side. Your GM will give you a \textbf{TN} to roll against that
  represents the strength of your force, but you can \textbf{augment}
  that \textbf{TN} with an appropriate leadership \textbf{ability}.
\item
  Acted Upon: You have little influence over the \textbf{outcome}, but
  are stuck in the middle of the conflict and must struggle to prosper
  within it. The GM predetermines the \textbf{outcome} of the overall
  competition on dramatic grounds. To determine your fate in the battle,
  you \textbf{contest} against a \textbf{resistance} determined by the
  GM, derived from the overall battle \textbf{outcome}.
\end{itemize}

\hypertarget{pyrrhic-victories}{%
\subsection{2.15 Pyrrhic Victories}\label{pyrrhic-victories}}

In a Pyrrhic victory, you \textbf{boost} your chance of \textbf{success}
in a \textbf{contest} by accepting negative consequence at its end, even
if you succeed in overcoming the \textbf{story obstacle}. You gain a
\textbf{bump} in the \textbf{ability} you are using to prosecute the
\textbf{contest}, but at the \textbf{contest}'s end, you suffer a
\textbf{state of adversity}, as per the Pyrrhic Victory Table.

As with any proposed action, you must convincingly describe the
potentially suicidal risks you are taking to achieve their objective.
You must also show how these risks can bring you the \textbf{victory}
they seek.

\hypertarget{pyrrhic-victory-table}{%
\subsubsection{PYRRHIC VICTORY TABLE}\label{pyrrhic-victory-table}}

\begin{longtable}[]{@{}cc@{}}
\toprule
Outcome & State of Adversity\tabularnewline
\midrule
\endhead
Any Defeat & Dead\tabularnewline
Marginal Victory & Dead\tabularnewline
Minor Victory & Dying\tabularnewline
Major Victory & Injured\tabularnewline
Complete Victory & Impaired\tabularnewline
\bottomrule
\end{longtable}

\hypertarget{mismatched-and-graduated-goals}{%
\subsection{2.16 Mismatched and Graduated
Goals}\label{mismatched-and-graduated-goals}}

Sometimes, the two sides in a \textbf{contest} may have goals that do
not directly conflict one another. A huntsman pursues a nurse, who is
trying to escape through the forest with two small children. The
huntsman wants to kill the nurse. The nurse wants to save the children.

When encountering \textbf{mismatched goals}, your GM should determine
whether the mismatch is complete, or partial.

In a \textbf{complete mismatch}, neither side is at all interested in
preventing the other's goal. A \textbf{complete mismatch} does not end
in a \textbf{contest}; your GM asks what you are doing, and then
describes each participant succeeding at their goals.

In most instances, the \textbf{contest} goals are not actually
\textbf{mismatched}, but \textbf{graduated}. You have both a
\textbf{primary} and a \textbf{secondary} goal. In this case, your GM
frames the \textbf{contest}, identifying which goal is which. To achieve
both, you must score a \textbf{major or complete victory}. On a
\textbf{minor or marginal victory}, you achieve only the
\textbf{primary} goal. Your GM may present you with the choice of which
objective you obtain, where that choice illuminates your PC's
priorities.

\hypertarget{difficult-automatic-victory}{%
\subsection{2.17 Difficult Automatic
Victory}\label{difficult-automatic-victory}}

Two distinct methods allow your GM to create a sense of difficulty
without allowing for the prospect of \textbf{failure}: the
\textbf{arduous automatic victory}, and the \textbf{costly automatic
victory}. Your GM will use a \textbf{costly automatic victory} where
\textbf{failure} offers no entertaining plot branches, but when they
still want the group to work for victory. In a \textbf{costly automatic
victory}, you succeed, but at a price.

\hypertarget{arduous-automatic-victory}{%
\subsubsection{2.17.1 Arduous Automatic
Victory}\label{arduous-automatic-victory}}

Your GM will use an \textbf{arduous automatic victory} to help with
pacing. They will simply allow you to succeed without a
\textbf{contest}, and then describe in sweat-inducing detail your
hard-won \textbf{victory}. Unlike the standard \textbf{automatic
victory}, the objective here is not to make you feel competent (though
it may do that as well) but to emphasize the rigors of your grueling
task, preserving suspension of disbelief.

\hypertarget{costly-automatic-victory}{%
\subsubsection{2.17.2 Costly Automatic
Victory}\label{costly-automatic-victory}}

Your GM may use a \textbf{costly automatic victory} where
\textbf{failure} offers no entertaining plot branches, but they still
want you to work for \textbf{victory}. In a \textbf{costly automatic
victory}, you succeed, but at a price.

Even if you suffer a \textbf{defeat}, you still overcome the
\textbf{story obstacle}. You also, however, suffer a \textbf{state of
adversity} to one or more \textbf{abilities}, in keeping with the
\textbf{defeat} level, as per the \textbf{consequences of defeat} table.

The \textbf{state of adversity} might apply, as in an ordinary
\textbf{contest}, to the abilities you used in that \textbf{contest}.
Usually, though, they apply to some other resource-related
\textbf{ability}, which may come to haunt you later on. When in doubt,
ask yourself if the \textbf{penalty} will cripple the group in dealing
with \textbf{contests}.

\hypertarget{character-creation}{%
\section{3.0 Character Creation}\label{character-creation}}

The first step in creating your character is to come up with a concept
that fits in with the genre of game that your GM intends to run. With
that, you can assign \textbf{abilities}, \textbf{ratings} for those
\textbf{abilities}, and if required \textbf{flaws}.

In addition, you will want to give your character a name, and provide a
physical description. We recommend focusing on three physical things
about your PC that others would immediately notice, over anything more
detailed.

\hypertarget{as-you-go-method}{%
\subsection{3.1 As-You-Go Method}\label{as-you-go-method}}

\begin{enumerate}
\def\labelenumi{\arabic{enumi}.}
\tightlist
\item
  Concept
\end{enumerate}

The concept is a brief phrase, often just a couple of words that tells
the GM and other players what you do and how you act. Start with a noun
or phrase indicating your \textbf{occupation keyword} or area of
expertise, and modify it with an adjective suggesting a
\textbf{distinguishing characteristic}, a personality trait that defines
you in broad strokes:

\begin{itemize}
\tightlist
\item
  haughty priestess
\item
  hotshot lawyer
\item
  noble samurai
\item
  remorseful assassin
\item
  sardonic ex-mercenary
\item
  slothful vampire
\item
  naive warrior
\end{itemize}

\begin{enumerate}
\def\labelenumi{\arabic{enumi}.}
\setcounter{enumi}{1}
\tightlist
\item
  Now provide your character with a name.
\item
  If the series uses other \textbf{keywords}, such as those for culture
  or religion, you may gain one for free.
\item
  When events in the story put you in a situation where you want to
  overcome a \textbf{story obstacle}, make up an applicable
  \textbf{ability} on the spot. The first time you use an
  \textbf{ability} (including the two you start play with:
  \textbf{distinguishing characteristic} and \textbf{occupational
  keyword}), assign a \textbf{rating} to it. This may be a
  \textbf{breakout ability} from a \textbf{keyword}. You are restricted
  to only one \textbf{sidekick}.
\item
  If you want, describe \textbf{flaws}.
\item
  Once you have 12 \textbf{abilities} (including the two for character
  concept), and up to three \textbf{flaws} you are done creating your
  character.
\end{enumerate}

\hypertarget{assigning-ability-ratings}{%
\subsection{3.2 Assigning Ability
Ratings}\label{assigning-ability-ratings}}

You have now defined your \textbf{abilities}. These tell everyone what
you can do. Now assign numbers to each \textbf{ability}, called
\textbf{ratings}, which determine how well you can do these things.

Assign a starting \textbf{rating} of 17 to the \textbf{ability} you find
most important or defining. Although most players consider it wisest to
assign this \textbf{rating} to their \textbf{occupational keyword}, you
don't have to do this. Assign a \textbf{rating} of 17 to your
\textbf{distinguishing characteristic}. In some cases, you may treat
your \textbf{distinguishing characteristic} as a \textbf{breakout
ability} from a \textbf{keyword} in this case, treat it as a +4.

All other \textbf{abilities} start at a \textbf{rating} of 13.

Now spend up to 20 points to increase any of your various
\textbf{ratings}, including \textbf{keywords}. Each point spent
increases a \textbf{rating} by 1 point. You can't spend more than 10
points on any one \textbf{ability}.

Some genre packs may require you to have additional \textbf{keywords}
that reflect the setting. These additional \textbf{keywords} come from
the 12 \textbf{abilites} allowance, so in many genres you will have
fewer wildcard \textbf{abilities} but better fit the setting.

\hypertarget{keywords}{%
\subsection{3.3 Keywords}\label{keywords}}

You may build your PC around one or more \textbf{keywords}. A
\textbf{keyword} gives you a package deal: you get a number of
\textbf{abilities} by selecting a pre-existing character concept, which
the player then modifies.

\textbf{Keywords} are best suited for use as the PC's
\textbf{occupation}.

In certain genres, you may require multiple \textbf{keywords}: for
example, one for \textbf{occupation}, another for species or culture,
and perhaps a third for religious affiliation.

Here are two ways to handle \textbf{keywords}. If in doubt, choose
Umbrella.

\textbf{Keywords as Packages}: Treat \textbf{keywords} simply as
shorthand for a package of \textbf{abilities}. These can be increased
together during character creation, but are too unrelated to increase
together during a game. You are still free to use the \textbf{keyword}
as an \textbf{ability}, and in fact may prefer to write only the
specific \textbf{abilities} they've improved on their character sheet.

\textbf{Keywords as an Umbrella}: Treat \textbf{keywords} both as
raisable \textbf{abilities} and as a collection of more specific
\textbf{abilities}. This approach keeps the character sheet from getting
too cluttered but encourages specialization. If your character is
particularly good at an aspect of that keyword, you create a
\textbf{breakout ability} under the \textbf{keyword} at a \textbf{bonus}
from the \textbf{rating} of the \textbf{keyword} you write these
specialized \textbf{breakout abilities} under the \textbf{keyword},
along with how much they've improved from the \textbf{keyword}:

Detective 17 Forensics +2 Handgun +1

In this example, whilst the \textbf{rating} for most \textbf{contests}
in which Detective was an appropriate \textbf{tactic} would be 17, for
contests involving Forensics it would be 19, and for those involving
firing a handgun it would be 18.

In some settings, an \textbf{ability} may be listed in more than one of
a PC's \textbf{keywords}. Choose only one to detail it under.

\hypertarget{flaws}{%
\subsection{3.4 Flaws}\label{flaws}}

You may assign up to three \textbf{flaws} to their PC. Common flaws
include:

\begin{itemize}
\tightlist
\item
  Personality traits: surly, petty, compulsive.
\item
  Physical challenges: blindness, lameness, diabetes.
\item
  Social hurdles: outcast, ill-mannered, hated by United supporters.
\end{itemize}

\textbf{Flaws} are assigned a \textbf{rating} equivalent to your
\textbf{abilities}. The first \textbf{flaw} is rated at the highest
\textbf{ability}, the second shares the same \textbf{rating} as the
second-highest \textbf{ability}, and the third equals the lowest
\textbf{ability}.

Certain \textbf{keywords} include \textbf{flaws}. \textbf{Flaws} gained
through \textbf{keywords} do not count against the limit of three chosen
\textbf{flaws}. All \textbf{flaws} after the third are given the same
\textbf{rating} as the third \textbf{ability}. You may designate
\textbf{flaws} from \textbf{keywords} as your first or second-ranked
\textbf{flaw}. When \textbf{flaws} manifest during play, your GM places
you in a \textbf{contest} against them, and rolls their associated
\textbf{ratings} as \textbf{resistances} to your efforts. This method
applies to \textbf{flaws} that primarily present you with additional
\textbf{story obstacles} to overcome.

Your GM may decide during play that certain \textbf{flaws} are better
expressed as \textbf{penalties} to your attempts to overcome other
\textbf{resistances}. Divide the value by 5 and round (a \textbf{flaw}
of 19 imposes a --4 \textbf{penalty}). This is appropriate where the
player specifies that your \textbf{ability} to solve problems drops
under certain specific conditions. Examples might include:

\begin{itemize}
\tightlist
\item
  Tongue-tied in large gatherings.
\item
  Lousy with a stick shift.
\item
  Can't stand snakes.
\end{itemize}

\hypertarget{advanced-character-creation}{%
\subsection{3.5 Advanced Character
Creation}\label{advanced-character-creation}}

\emph{QuestWorlds} offers two advanced methods of character creation:
prose and list.

\hypertarget{the-list-method}{%
\subsubsection{3.5.1 The List Method}\label{the-list-method}}

This is like the As-You-Go method (see §3.1) but you spend all their
points before the game begins. This is possible with the As-You-Go
method as well, but the list method allows you to signal what they want
the game to be about from the abilities you pick, as opposed to reacting
to material once the game begins.

\hypertarget{the-prose-method}{%
\subsubsection{3.5.2 The Prose Method}\label{the-prose-method}}

This is the most different method as you write a piece of prose and then
pull \textbf{abilities} from that. Its intent is to emulate a character
description in fiction, and indeed PCs can be built by copying text from
a story and then identifying \textbf{keywords}. It is the least `fair'
of the character creation options.

\hypertarget{advanced-character-creation-1}{%
\subsection{3.5 Advanced Character
Creation}\label{advanced-character-creation-1}}

\emph{QuestWorlds} offers two advanced methods of character creation:
prose and list.

\hypertarget{the-list-method-1}{%
\subsubsection{3.5.1 The List Method}\label{the-list-method-1}}

This is like the As-You-Go method (see §3.1) but you spend all their
points before the game begins. This is possible with the As-You-Go
method as well, but the list method allows you to signal what they want
the game to be about from the abilities you pick, as opposed to reacting
to material once the game begins.

\hypertarget{the-prose-method-1}{%
\subsubsection{3.5.2 The Prose Method}\label{the-prose-method-1}}

This is the most different method as you write a piece of prose and then
pull \textbf{abilities} from that. Its intent is to emulate a character
description in fiction, and indeed PCs can be built by copying text from
a story and then identifying \textbf{keywords}. It is the least `fair'
of the character creation options.

\hypertarget{list-method}{%
\subsection{3.6 List Method}\label{list-method}}

\begin{enumerate}
\def\labelenumi{\arabic{enumi}.}
\tightlist
\item
  Concept
\end{enumerate}

The concept is a brief phrase, often just a couple of words that tells
the GM and other players what you do and how you act. When in doubt,
start with a noun or phrase indicating your \textbf{occupation}, and
modify it with an adjective suggesting a \textbf{distinguishing
characteristic}:

\begin{itemize}
\tightlist
\item
  haughty priestess
\item
  hotshot lawyer
\item
  noble samurai
\item
  remorseful assassin
\item
  sardonic ex-mercenary
\item
  slothful vampire
\item
  naive warrior
\end{itemize}

\begin{enumerate}
\def\labelenumi{\arabic{enumi}.}
\setcounter{enumi}{1}
\tightlist
\item
  Now provide the character with a name.
\item
  Note their \textbf{occupation}, which is usually a \textbf{keyword}.
  You probably already picked this when you came up with your character
  concept.
\item
  If the series uses other \textbf{keywords}, such as those for culture
  or religion, you may have one of them for free.
\item
  Pick 10 additional \textbf{abilities}, describing them however the
  player wants. Only one of these \textbf{abilities} may be a
  \textbf{sidekick}.
\item
  If you want, describe up to 3 \textbf{flaws}.
\end{enumerate}

\hypertarget{prose-method}{%
\subsection{3.7 Prose Method}\label{prose-method}}

You write a paragraph of text like you would see in a story outline,
describing the most essential elements of your character. Include
\textbf{keywords}, personality traits, important possessions,
relationships, and anything else that suggests what you can do and why.
The paragraph should be about 100 words long.

Compose the description in complete, grammatical sentences. No lists of
\textbf{abilities}; no sentence fragments. Your GM may choose to allow
sentences like the previous one for emphasis or rhythmic effect, but not
simply to squeeze in more cool things you can do.

Once your narrative is finished, convert the description into a set of
\textbf{abilities}. Mark any \textbf{keywords} with double underlines.
Mark any other word or phrase that could be an \textbf{ability} with a
single underline. Then write these \textbf{keywords} and
\textbf{abilities} on your character sheet.

There is no limit to the number of \textbf{abilities} you can gain from
a single sentence, as long as the sentence is not just a list of
\textbf{abilities}. If your GM decides a sentence is just a list, they
may allow the first two \textbf{abilities}, or they may tell the player
to rewrite the sentence. Note, however, that you cannot specify more
than one \textbf{sidekick} in your prose description.

\hypertarget{simple-contests}{%
\section{4.0 Simple Contests}\label{simple-contests}}

\textbf{Simple contest}s are the default resolution method for all
\textbf{story obstacles}.

\hypertarget{simple-contest-1}{%
\subsection{4.1 Simple Contest}\label{simple-contest-1}}

\hypertarget{procedure}{%
\subsubsection{4.1.1 Procedure}\label{procedure}}

\begin{enumerate}
\def\labelenumi{\arabic{enumi}.}
\tightlist
\item
  Your GM \textbf{frames the contest}.
\item
  You choose a \textbf{tactic}, and figure your PC's \textbf{target
  number} (\textbf{TN}) using the \textbf{rating} and any
  \textbf{modifiers}. The PCs \textbf{TN} is the \textbf{rating} of
  their \textbf{ability}, plus or minus \textbf{modifiers} the GM may
  give you.
\item
  Your GM determines the \textbf{resistance}. If two PCs contend, your
  opponent figures their \textbf{TN} as described in step 2.
\item
  Roll a D20 to determine your \textbf{success or failure}, then apply
  any \textbf{bumps}. Your GM does the same for the \textbf{resistance}.
  Compare your rolled number with your \textbf{TN} to see how well you
  succeeded or failed with your \textbf{ability}. Remember to apply any
  \textbf{bumps} from \textbf{masteries} or \textbf{hero points}.
\item
  Determine \textbf{victory} or \textbf{defeat}
\item
  Describe the \textbf{outcome} based on the \textbf{story obstacle}.
\end{enumerate}

\hypertarget{group-simple-contest}{%
\subsection{4.2 Group Simple Contest}\label{group-simple-contest}}

In the \textbf{group simple contest}, multiple participants take part in
a \textbf{simple contest}. Each of you in your group conducts an
individual \textbf{simple contest} against the GM, and the
\textbf{outcomes} for each side are collated to determine the victor.

A \textbf{group simple contest} may pit all of you against a single
\textbf{resistance}, representing one \textbf{story obstacle}.
Alternatively, a \textbf{group simple contest} may be a series of paired
match-ups between two groups of contestants. If you are forced to
participate in more than one \textbf{contest}, then you face the
standard multiple opponent \textbf{penalties}.

\hypertarget{procedure-1}{%
\subsubsection{4.2.1 Procedure}\label{procedure-1}}

\begin{enumerate}
\def\labelenumi{\arabic{enumi}.}
\tightlist
\item
  Your GM \textbf{frames the contest}.
\item
  You choose a \textbf{tactic}, and figure your PC's \textbf{target
  number} (\textbf{TN}) using the \textbf{rating} and any
  \textbf{modifiers}. Your \textbf{TN} is the \textbf{rating} of their
  \textbf{ability}, plus or minus \textbf{modifiers} the GM may give
  you.
\item
  Your GM determines the \textbf{resistance}. If two PCs contend, your
  opponent figures their \textbf{TN} as described in step 2.
\item
  For each of your group, roll a D20 to determine your \textbf{success
  or failure}, then apply any \textbf{bumps}. Your GM does the same for
  the \textbf{resistance}. Compare your rolled number with your
  \textbf{TN} to see how well you succeeded or failed with your
  \textbf{ability}. Remember to apply any \textbf{bumps} from
  \textbf{masteries} or \textbf{hero points}.
\item
  Determine the number of \textbf{Outcome Points (OPs)} scored by the
  victor in the \textbf{contest} from the \textbf{group simple contest}
  table. On a tie the \textbf{victory} goes to the \textbf{better roll}.
\item
  Sum the \textbf{OPs} gained for each side.
\item
  The side with the highest number of \textbf{OPs} is the victor in the
  \textbf{contest}.
\item
  Describe the \textbf{outcome} based on the agreed \textbf{prize}.
\end{enumerate}

\hypertarget{group-simple-contest-resource-points}{%
\subsubsection{4.2.2 Group Simple Contest Resource
Points}\label{group-simple-contest-resource-points}}

\hypertarget{group-simple-contest-table}{%
\subsubsection{GROUP SIMPLE CONTEST
TABLE}\label{group-simple-contest-table}}

\begin{longtable}[]{@{}ccccc@{}}
\toprule
& Critical & Success & Failure & Fumble\tabularnewline
\midrule
\endhead
Critical & 1 & 2 & 3 & 5\tabularnewline
Success & 2 & 1 & 2 & 3\tabularnewline
Failure & 3 & 2 & 1 & 2\tabularnewline
Fumble & 5 & 3 & 2 & 0\tabularnewline
\bottomrule
\end{longtable}

\hypertarget{advanced-simple-contests}{%
\subsection{4.3 Advanced Simple
Contests}\label{advanced-simple-contests}}

Advanced options for a simple contest allow greater granularity of the
\textbf{outcome}, through \textbf{benefits of victory} and
\textbf{consequences of defeat}.

\hypertarget{simple-contest-2}{%
\subsection{4.4 Simple Contest}\label{simple-contest-2}}

\hypertarget{advanced-procedure}{%
\subsubsection{4.4.1 Advanced Procedure}\label{advanced-procedure}}

\begin{enumerate}
\def\labelenumi{\arabic{enumi}.}
\item
  Your GM \textbf{frames the contest}.
\item
  You choose a \textbf{tactic}, and figure your PC's \textbf{target
  number} (\textbf{TN}) using the \textbf{rating} of your
  \textbf{ability}, plus or minus \textbf{modifiers} the GM may give
  you.
\item
  Your GM determines the \textbf{resistance}. If two PCs contend, your
  opponent figures their \textbf{TN} as described in step 2.
\item
  Roll a d20 to determine your \textbf{degree of victory or defeat},
  then apply any \textbf{bumps}. Your GM does the same for the
  \textbf{resistance}. Compare your rolled number with your \textbf{TN}
  to see how well you succeeded or failed with your \textbf{ability}.
  Remember to apply any \textbf{bumps} from \textbf{masteries} or
  \textbf{hero points}.
\item
  Determine \textbf{victory} or \textbf{defeat}

  \emph{1. Determine \textbf{degree of victory}. Compare your
  \textbf{success or failure} with the GM's on the \textbf{simple
  contest table}.}

  \emph{2. Determine \textbf{benefits of victory} or
  \textbf{consequences of defeat}.}
\item
  Describe the \textbf{outcome} based on the \textbf{story obstacle}.
\end{enumerate}

\hypertarget{degree-of-victory}{%
\subsubsection{4.4.2 Degree of Victory}\label{degree-of-victory}}

\hypertarget{simple-contest-table}{%
\subsubsection{SIMPLE CONTEST TABLE}\label{simple-contest-table}}

\begin{longtable}[]{@{}ccccc@{}}
\toprule
\begin{minipage}[b]{0.10\columnwidth}\centering
Roll\strut
\end{minipage} & \begin{minipage}[b]{0.18\columnwidth}\centering
Critical\strut
\end{minipage} & \begin{minipage}[b]{0.19\columnwidth}\centering
Success\strut
\end{minipage} & \begin{minipage}[b]{0.19\columnwidth}\centering
Failure\strut
\end{minipage} & \begin{minipage}[b]{0.19\columnwidth}\centering
Fumble\strut
\end{minipage}\tabularnewline
\midrule
\endhead
\begin{minipage}[t]{0.10\columnwidth}\centering
Critical\strut
\end{minipage} & \begin{minipage}[t]{0.18\columnwidth}\centering
Better roll = Marginal Victory, else tie\strut
\end{minipage} & \begin{minipage}[t]{0.19\columnwidth}\centering
Minor Victory\strut
\end{minipage} & \begin{minipage}[t]{0.19\columnwidth}\centering
Major Victory\strut
\end{minipage} & \begin{minipage}[t]{0.19\columnwidth}\centering
Complete Victory\strut
\end{minipage}\tabularnewline
\begin{minipage}[t]{0.10\columnwidth}\centering
Success\strut
\end{minipage} & \begin{minipage}[t]{0.18\columnwidth}\centering
Minor Victory\strut
\end{minipage} & \begin{minipage}[t]{0.19\columnwidth}\centering
Better roll = Marginal Victory, else tie\strut
\end{minipage} & \begin{minipage}[t]{0.19\columnwidth}\centering
Minor Victory\strut
\end{minipage} & \begin{minipage}[t]{0.19\columnwidth}\centering
Major Victory\strut
\end{minipage}\tabularnewline
\begin{minipage}[t]{0.10\columnwidth}\centering
Failure\strut
\end{minipage} & \begin{minipage}[t]{0.18\columnwidth}\centering
Major Victory\strut
\end{minipage} & \begin{minipage}[t]{0.19\columnwidth}\centering
Minor Victory\strut
\end{minipage} & \begin{minipage}[t]{0.19\columnwidth}\centering
Better roll = Marginal Victory, else tie\strut
\end{minipage} & \begin{minipage}[t]{0.19\columnwidth}\centering
Minor Victory\strut
\end{minipage}\tabularnewline
\begin{minipage}[t]{0.10\columnwidth}\centering
Fumble\strut
\end{minipage} & \begin{minipage}[t]{0.18\columnwidth}\centering
Complete Victory\strut
\end{minipage} & \begin{minipage}[t]{0.19\columnwidth}\centering
Major Victory\strut
\end{minipage} & \begin{minipage}[t]{0.19\columnwidth}\centering
Minor Victory\strut
\end{minipage} & \begin{minipage}[t]{0.19\columnwidth}\centering
Tie*\strut
\end{minipage}\tabularnewline
\bottomrule
\end{longtable}

\begin{itemize}
\tightlist
\item
  In a \textbf{group simple contest} (see below), your GM may declare
  that both contestants suffer a *\textbf{marginal defeat} to indicate
  that, although their \textbf{results} cancel out with respect to each
  other, their situation worsens compared to other contestants.
\end{itemize}

\hypertarget{advanced-group-simple-contest}{%
\subsection{4.5 Advanced Group Simple
Contest}\label{advanced-group-simple-contest}}

\hypertarget{procedure-2}{%
\subsubsection{4.5.1 Procedure}\label{procedure-2}}

\begin{enumerate}
\def\labelenumi{\arabic{enumi}.}
\item
  Your GM \textbf{frames the contest}.
\item
  You choose a \textbf{tactic}, and figure your PC's \textbf{target
  number} (\textbf{TN}) using the \textbf{rating} of your
  \textbf{ability}, plus or minus \textbf{modifiers} the GM may give
  you.
\item
  Your GM determines the \textbf{resistance}. If two PCs contend, your
  opponent figures their \textbf{TN} as described in step 2.
\item
  For each of your group, roll a D20 to determine your \textbf{degree of
  victory or defeat}, then apply any bumps. Your GM does the same for
  the \textbf{resistance}. Compare your rolled number with your
  \textbf{TN} to see how well you succeeded or failed with your
  \textbf{ability}. Remember to apply any \textbf{bumps} from
  \textbf{masteries} or \textbf{hero points}.
\item
  Determine the number of \textbf{OPs} scored by the victor in the
  \textbf{contest} from the \textbf{group simple contest table}. On a
  tie the \textbf{victory} goes to the \textbf{better roll}.
\item
  Sum the \textbf{OPs} gained for each side.
\item
  The side with the highest number of \textbf{OPs} is the victor in the
  \textbf{contest}.

  \emph{1. Determine \textbf{degree of victory} based on the difference
  between the winner and loser's \textbf{outcome point} totals on the
  \textbf{group simple contest table}.}

  \emph{2. Determine \textbf{benefits of victory} or
  \textbf{consequences of defeat}.}
\item
  Describe the \textbf{outcome} based on the agreed \textbf{prize}.
\end{enumerate}

\hypertarget{degree-of-victory-table}{%
\subsubsection{DEGREE OF VICTORY TABLE}\label{degree-of-victory-table}}

\begin{longtable}[]{@{}cc@{}}
\toprule
Difference Between OPs & Winning Group's Degree of
Victory\tabularnewline
\midrule
\endhead
1 & Marginal\tabularnewline
2 & Minor\tabularnewline
3-4 & Major\tabularnewline
5+ & Complete\tabularnewline
\bottomrule
\end{longtable}

\hypertarget{consequences-and-benefits}{%
\subsubsection{4.5.2 Consequences and
Benefits}\label{consequences-and-benefits}}

Depending on which approach seems to grow from the story, your GM may
assign \textbf{consequences of defeat} or \textbf{benefits of victory}
from \textbf{group simple contests} to the entire group, or to
individual members who performed either especially well, or especially
poorly. Your GM should default to rewarding or penalizing everyone. Your
GM should resort to individualized repercussions or benefits only when a
group reward defies dramatic credibility, or when competition within the
group is a pivotal dramatic issue.

\hypertarget{boosting-outcomes}{%
\subsubsection{4.5.3 Boosting Outcomes}\label{boosting-outcomes}}

Because they average together the \textbf{outcomes} of multiple
participants, \textbf{group simple contests} tend to flatten
\textbf{outcomes}, making \textbf{victories} more likely to be
\textbf{marginal} or \textbf{minor} than \textbf{major} or
\textbf{complete}.

To overcome this flattening effect, you can, at the beginning of a
\textbf{group simple contest}, spend one or more \textbf{hero points} to
purchase a \textbf{boost}. A \textbf{boost} assures a clearer
\textbf{victory}, should you prevail.

The cost varies by the number of PCs participating:

\begin{itemize}
\tightlist
\item
  1 \textbf{hero point} for 1-3 PCs.
\item
  2 \textbf{hero points} for 4-6 PCs.
\item
  3 \textbf{hero points} for 7-9 PCs.\\
\item
  and so on\ldots{}
\end{itemize}

You may spend twice as many \textbf{hero points} as required to gain a
\textbf{double boost}. The points may be spent by any combination of
players. They remain spent no matter how the \textbf{contest} resolves.

On a \textbf{tie} or a \textbf{victory}, the \textbf{boost} increases
the collective \textbf{victory} level by one step. A \textbf{double
boost} increases it by two steps.

\hypertarget{advanced-long-contests}{%
\section{5.0 (Advanced) Long Contests}\label{advanced-long-contests}}

Most conflicts should be resolved simply and quickly, using the
\textbf{simple contest} rules.

However, every so often, your GM wants to draw out the resolution,
breaking it down into a series of smaller actions, increasing the
suspense you feel as you wait to see if they \textbf{succeed} or
\textbf{fail}.

Think of the different ways a film director can choose to portray a
given moment, depending on how important it is to the story, and how
invested they want us to feel in its \textbf{outcome}.

For example, there are two ways to shoot a scene in which a thief breaks
into the bank to steal the contents of the safe.

The action can be portrayed quickly, cutting to a moment with the thief,
their ear pressed against the safe trying to get the tumblers to fall
into place. Then they sigh with relief, open the safe, and get whatever
is inside. In this instance, the story is about what happens after the
thief gets what's in the safe, not about what might happen to them if
they fail.

Another film might instead choose to make the bank robbery a pivotal
turning point in the story, if not its climactic moment. It would spend
many scenes building up to the safe-cracking sequence: obtaining the
plans of the bank, learning the movements of the guards, crawling
through the air conditioning ducts, sliding past the motion sensors and
pressure plates, and finally cracking the safe itself. All of these
scenes would be \textbf{rounds} of a \textbf{long contest}.

Remember that \emph{QuestWorlds} uses conflict resolution. If you want
to describe how you overcome a sequence of \textbf{story obstacles} to
overcome the \textbf{resistance} then your GM should use a \textbf{long
contest}, if you just want to move on to the next scene, use a
\textbf{simple contest}.

Even a movie driven by action and suspense will typically include only a
handful of these set-piece sequences. They need the rest of their
running time to build up to their big moments, to make us care about the
characters, and to give us quiet moments to contrast with the
white-knuckle parts.

So pacing may always trump your desire to work through the sequence of
events, as your GM may wish to resolve this conflict quickly. This is
especially true if only one player is involved.

Your GM may be tempted, to adjudicate every fight with a \textbf{long
contest}, because fights seem like they should be played out
blow-by-blow. They should resist this temptation, as fights are often
repetitive trading of blows that can drag when everyone repeats actions
from \textbf{round} to \textbf{round}. Only use \textbf{long contests}
for fights where the PCs want to do more than slug it out toe-to-toe
with their opponents until only one is left standing.

There are three types of \textbf{long contest}. Your GM should choose
one to use with their campaign: \textbf{scored contest},
\textbf{extended contest}, or \textbf{chained contest}.

\hypertarget{scored-contest}{%
\subsection{5.1 Scored Contest}\label{scored-contest}}

\textbf{Scored contests} are longer and more dramatic than
\textbf{simple contests}. Your GM uses \textbf{scored contests} when the
\textbf{outcome} of the struggle is important, to generate suspense for
you, or when your GM want a back-and-forth struggle. It is something you
and your GM should visualize and describe.

A \textbf{scored contest} consists of one or more \textbf{rounds}, in
which you perform actions that are similar to \textbf{simple contests}.
However, actions and \textbf{rounds} do not decide the \textbf{outcome}
of the whole \textbf{contest}, only who gains or loses
\textbf{resolution points} at that time. In a \textbf{scored contest}
there is no distinction between aggressor and defender, each
\textbf{round} represents attempts by both parties to overcome their
opponent. Your GM should determine who has the initiative to describe
what they are doing for any \textbf{exchange}, based on their
interpretation of the flow of events. If in doubt your GM should defer
to you over your opponent to describe what you do in the \textbf{round},
and describe the NPC reacting to that.

\hypertarget{procedure-3}{%
\subsubsection{5.1.1 Procedure}\label{procedure-3}}

\begin{enumerate}
\def\labelenumi{\arabic{enumi}.}
\tightlist
\item
  Your GM \textbf{frames the contest}.
\item
  You choose a \textbf{tactic}, and figure your PC's \textbf{target
  number} (\textbf{TN}) using the \textbf{rating} of your
  \textbf{ability}, plus or minus \textbf{modifiers} the GM may give
  you.
\item
  Your GM determines the \textbf{resistance}. If two PCs contend, your
  opponent figures their \textbf{TN} as described in step 2.
\item
  Carry out one or more \textbf{rounds}, repeating as necessary.

  \begin{enumerate}
  \def\labelenumii{\arabic{enumii}.}
  \tightlist
  \item
    A \textbf{scored contest} unfolds as a series of \textbf{simple
    contests}. At the end of each \textbf{simple contest}, the winner
    scores a number of \textbf{resolution points (RPs)} to their tally,
    which varies between 1 and 5, depending on the \textbf{result}. Tied
    \textbf{results} leave the score unchanged.
  \item
    Your GM decides which opponent has the initiative and describes what
    they are trying to do to achieve the \textbf{prize}, the
    `aggressor'. The `defender' describes how they counter the
    aggressor's attempt to seize the \textbf{prize}. If it is not
    obvious from the unfolding narrative, your GM should choose your PC
    as the `aggressor'.
  \item
    Conduct a \textbf{simple contest} as normal, but once the
    \textbf{outcome} has been determined, it becomes a number of
    \textbf{resolution points} scored by the winning side.
  \item
    The number of \textbf{resolution points} the winner garners at the
    end of each \textbf{round} depends on the \textbf{degree of victory}
    they scored. They get 1 point for a \textbf{marginal victory}, 2 for
    a \textbf{minor victory}, 3 for a \textbf{major victory}, and 5 for
    a \textbf{complete victory}.
  \item
    The first to accumulate a total of 5 points wins; their opponent is
    knocked out of the \textbf{contest} and loses the \textbf{prize}.
  \end{enumerate}
\item
  Determine the \textbf{scored contest outcome} based on \textbf{rising
  action} or \textbf{climax} (below).
\item
  Determine \textbf{benefits of victory} or \textbf{consequences of
  defeat}.
\item
  Describe the \textbf{outcome} based on the \textbf{story obstacle}.
\end{enumerate}

Unlike in an \textbf{extended contest} (see below), where you usually
take part in two \textbf{exchanges} with your opponent per
\textbf{round} (one in which you choose the \textbf{AP bid}, and one in
which your opponent does), here you and your opponent engage in a single
\textbf{exchange} per \textbf{round} (in which whoever the GM determines
has initiative describes an action to seize the \textbf{prize} and their
opponent how they intend to stop them).

Your \textbf{resolution point} score tells you how well you're doing,
relative to your opponent, in the ebb and flow of a fluid, suspenseful
conflict. If you're leading your opponent by 0--4, you're giving them a
thorough pasting. If you're behind 4--0, you're on your last legs, while
your opponent has had an easy time of it. If you're tied, you've each
been getting in some good licks.

In a fight, scoring 1 \textbf{RP} might mean that you hit your opponent
with a grazing blow, or knocked him into an awkward position.

Scoring 2 \textbf{RPs} might mean a palpable hit, most likely with
bone-crunching sound effects.

A 3 \textbf{RP} hit sends them reeling, and, depending on the realism
level of the genre, may be accompanied by a spray of blood.

However, the exact physical harm you've dished out to them remains
unclear until the \textbf{contest's} end. When that happens, the real
effects of your various \textbf{victories} become suddenly apparent.
Perhaps they stagger, merely dazed, up against a wall. Maybe they fall
over dead.

In a debate, a 1 \textbf{RP} might occasion mild head nodding from
spectators, or a frown on your opponent's face.

2 \textbf{RPs} would occasion mild applause from onlookers, or send a
flush to your opponent's face.

On 3 \textbf{RPs}, your opponent might be thrown completely off-track,
as audience members wince at the force of your devastating verbal jab.

In interpreting the individual \textbf{simple contests} within a
\textbf{scored contest}, your GM is guided by two principles:

\begin{enumerate}
\def\labelenumi{\arabic{enumi}.}
\tightlist
\item
  No consequence is certain until the entire \textbf{scored contest} is
  over.
\item
  When a character scores points, it can reflect any positive change in
  fortunes, not just the most obvious one.
\end{enumerate}

\hypertarget{resource-point-table}{%
\subsubsection{Resource Point Table}\label{resource-point-table}}

\begin{longtable}[]{@{}ccccc@{}}
\toprule
& Critical & Success & Failure & Fumble\tabularnewline
\midrule
\endhead
Critical & 1 & 2 & 3 & 5\tabularnewline
Success & 2 & 1 & 2 & 3\tabularnewline
Failure & 3 & 2 & 1 & 2\tabularnewline
Fumble & 5 & 3 & 2 & n/a\tabularnewline
\bottomrule
\end{longtable}

\hypertarget{scored-contest-outcomes}{%
\subsubsection{5.1.2 Scored Contest
Outcomes}\label{scored-contest-outcomes}}

As with all \textbf{contests}, if the contest involved a resistance we
care about your \textbf{outcome}, win or lose, and the GM should feel
free to narrate the \textbf{outcome} for the \textbf{resistance}
depending on their interpretation of your \textbf{outcome} which may not
be symmetrical. For example, if the \textbf{benefit of victory} for your
PC is \textbf{pumped} the GM should feel free to interpret what this
means for the resistance: in a melee they might be dead, in a social
contest they might be exiled, or they might surrender in the melee or
cede ground in a social contest. Your GM should focus on the
\textbf{prize} that was agreed when determining how to narrate over the
outcome for the \textbf{resistance}.

In a PC vs.~PC contest however, your GM should treat the results as
symmetrical when determining the \textbf{outcome}.

\hypertarget{rising-action}{%
\subsubsection{5.1.2.1 Rising Action}\label{rising-action}}

\textbf{Rising action} refers to all of the many plot events and
complications that occur between the beginning and the climax of a
story. During this phase of your GM's story, they will use the
\textbf{rising action} consequence table to assess \textbf{outcomes}.

Find the difference between you and your opponent's \textbf{resolution
point} scores at the \textbf{contest}'s conclusion. Your GM then
determines your \textbf{outcome} by cross-referencing with the following
table to find your \textbf{benefits of victory} or \textbf{consequences
of defeat}.

Note, you may suffer a \textbf{state of adversity}, even if you win the
\textbf{prize}.

\hypertarget{rising-action-contest-table}{%
\subsubsection{RISING ACTION CONTEST
TABLE}\label{rising-action-contest-table}}

\begin{longtable}[]{@{}cccc@{}}
\toprule
\begin{minipage}[b]{0.19\columnwidth}\centering
Difference Between RPs\strut
\end{minipage} & \begin{minipage}[b]{0.26\columnwidth}\centering
Negative Consequences for Loser\strut
\end{minipage} & \begin{minipage}[b]{0.26\columnwidth}\centering
Consequences/Benefit for Winner\strut
\end{minipage} & \begin{minipage}[b]{0.17\columnwidth}\centering
Victory/Defeat Level\strut
\end{minipage}\tabularnewline
\midrule
\endhead
\begin{minipage}[t]{0.19\columnwidth}\centering
1\strut
\end{minipage} & \begin{minipage}[t]{0.26\columnwidth}\centering
Hurt\strut
\end{minipage} & \begin{minipage}[t]{0.26\columnwidth}\centering
Hurt\strut
\end{minipage} & \begin{minipage}[t]{0.17\columnwidth}\centering
Marginal\strut
\end{minipage}\tabularnewline
\begin{minipage}[t]{0.19\columnwidth}\centering
2\strut
\end{minipage} & \begin{minipage}[t]{0.26\columnwidth}\centering
Hurt\strut
\end{minipage} & \begin{minipage}[t]{0.26\columnwidth}\centering
Fresh\strut
\end{minipage} & \begin{minipage}[t]{0.17\columnwidth}\centering
Marginal\strut
\end{minipage}\tabularnewline
\begin{minipage}[t]{0.19\columnwidth}\centering
3\strut
\end{minipage} & \begin{minipage}[t]{0.26\columnwidth}\centering
Impaired\strut
\end{minipage} & \begin{minipage}[t]{0.26\columnwidth}\centering
Pumped\strut
\end{minipage} & \begin{minipage}[t]{0.17\columnwidth}\centering
Minor\strut
\end{minipage}\tabularnewline
\begin{minipage}[t]{0.19\columnwidth}\centering
4\strut
\end{minipage} & \begin{minipage}[t]{0.26\columnwidth}\centering
Impaired\strut
\end{minipage} & \begin{minipage}[t]{0.26\columnwidth}\centering
Pumped\strut
\end{minipage} & \begin{minipage}[t]{0.17\columnwidth}\centering
Minor\strut
\end{minipage}\tabularnewline
\begin{minipage}[t]{0.19\columnwidth}\centering
5\strut
\end{minipage} & \begin{minipage}[t]{0.26\columnwidth}\centering
Injured\strut
\end{minipage} & \begin{minipage}[t]{0.26\columnwidth}\centering
Invigorated\strut
\end{minipage} & \begin{minipage}[t]{0.17\columnwidth}\centering
Major\strut
\end{minipage}\tabularnewline
\begin{minipage}[t]{0.19\columnwidth}\centering
6\strut
\end{minipage} & \begin{minipage}[t]{0.26\columnwidth}\centering
Injured\strut
\end{minipage} & \begin{minipage}[t]{0.26\columnwidth}\centering
Invigorated\strut
\end{minipage} & \begin{minipage}[t]{0.17\columnwidth}\centering
Major\strut
\end{minipage}\tabularnewline
\begin{minipage}[t]{0.19\columnwidth}\centering
7\strut
\end{minipage} & \begin{minipage}[t]{0.26\columnwidth}\centering
Dying\strut
\end{minipage} & \begin{minipage}[t]{0.26\columnwidth}\centering
Heroic\strut
\end{minipage} & \begin{minipage}[t]{0.17\columnwidth}\centering
Complete\strut
\end{minipage}\tabularnewline
\begin{minipage}[t]{0.19\columnwidth}\centering
8\strut
\end{minipage} & \begin{minipage}[t]{0.26\columnwidth}\centering
Dead\strut
\end{minipage} & \begin{minipage}[t]{0.26\columnwidth}\centering
Heroic\strut
\end{minipage} & \begin{minipage}[t]{0.17\columnwidth}\centering
Complete\strut
\end{minipage}\tabularnewline
\begin{minipage}[t]{0.19\columnwidth}\centering
9\strut
\end{minipage} & \begin{minipage}[t]{0.26\columnwidth}\centering
Dead\strut
\end{minipage} & \begin{minipage}[t]{0.26\columnwidth}\centering
Heroic\strut
\end{minipage} & \begin{minipage}[t]{0.17\columnwidth}\centering
Complete\strut
\end{minipage}\tabularnewline
\bottomrule
\end{longtable}

\hypertarget{climax}{%
\subsubsection{5.1.2.2 Climax}\label{climax}}

For the final, climactic confrontation that wraps up your GM's story,
you may suffer a \textbf{state of adversity}, even if the
\textbf{outcome} shows that you won the \textbf{prize}. This represents
that at the \textbf{climax} you may triumph, but be laid low by the
effort.

First, determine your \textbf{outcome} for the \textbf{contest} as for
rising action, but in addition, if the outcome show that you gained the
\textbf{prize} your GM now cross-references the \textbf{resolution
points} scored against you by your opponent on the \textbf{climactic
state of adversity} table to determine the \textbf{state of adversity}
you suffered in winning that \textbf{victory}. If you lost the
\textbf{prize} use the \textbf{RPs} scored against you to determine your
\textbf{outcome} as per the \textbf{rising action} table above.

\hypertarget{climactic-state-of-adversity-table}{%
\subsubsection{CLIMACTIC STATE OF ADVERSITY
TABLE}\label{climactic-state-of-adversity-table}}

\begin{longtable}[]{@{}cc@{}}
\toprule
Total Resolution Points Scored Against PC & State of
Adversity\tabularnewline
\midrule
\endhead
0 & Unharmed\tabularnewline
1 & Dazed\tabularnewline
2 & Hurt\tabularnewline
3 & Hurt\tabularnewline
4 & Impaired\tabularnewline
5 & Impaired\tabularnewline
6 & Injured\tabularnewline
7 & Injured\tabularnewline
8 & Dying\tabularnewline
9 & Dead\tabularnewline
\bottomrule
\end{longtable}

\hypertarget{parting-shot}{%
\subsubsection{5.1.3 Parting Shot}\label{parting-shot}}

In the \textbf{round} immediately after you take an opponent out of the
\textbf{contest}, you may attempt to increase the severity of the
consequences your opponent suffers by engaging in a \textbf{parting
shot}. This is an attempt (metaphoric or otherwise) to kick your
opponent while he's down:

• Striking an incapacitated enemy • Attacking a retreating army •
Attaching one more punitive rider to a legal settlement • Demanding
additional money from a business partner • Delivering one last
humiliating insult

Your GM should not use a \textbf{parting shot}.

If you succeed in your \textbf{parting shot} roll, you add the result
from your roll to the final number of \textbf{resolution points} scored
against your opponent in the round that removed them from the
\textbf{contest}.

However, if your opponent succeeds, they take the number of
\textbf{resolution points} they would, in a standard \textbf{round},
score against you, and instead subtracts them from the number of
\textbf{resolution points} scored against them in the \textbf{round}
that removed them from the \textbf{contest}. If the revised total is now
less than 5 \textbf{RPs}, they return to the \textbf{contest}, and may
re-engage you. Your GM describes this as a dramatic turnaround, in which
your overreaching has somehow granted them an advantage allowing them to
recover from their previous misfortune. The provisional consequences
they suffered now go away, and are treated as a momentary or seeming
disadvantage.

The \textbf{ability} you use must relate to the consequences the
opposition will suffer, but needn't be the same one you used to win the
\textbf{contest}. If the loser is a PC they use a suitable
\textbf{ability} to resist; otherwise the GM rolls a suitable
\textbf{resistance} value.

Where it makes sense, unengaged PCs may attempt \textbf{parting shots}
against opponents taken out of the \textbf{contest} by someone else. You
may not revive your teammates by using your lamest abilities to make
\textbf{parting shots} on them; this, by definition, does not pass a
\textbf{credibility test}.

\hypertarget{asymmetrical-round}{%
\subsubsection{5.1.4 Asymmetrical Round}\label{asymmetrical-round}}

You may choose to briefly suspend your attempt to best your opponent in
a \textbf{scored contest}, in order to do something else. An instance
where you are trying to do something else and your opponent is trying to
win the \textbf{contest} is called an \textbf{asymmetrical round}.

In an \textbf{asymmetrical round}, you do not score \textbf{RPs} against
your opponent if you win the \textbf{round}. Instead, you succeed at
whatever else you were doing. You still lose \textbf{RPs} if you fail.
Often you will be using an \textbf{ability} other than the one you've
been waging the \textbf{contest} with, one better suited to the task at
hand. This becomes additionally dangerous when the \textbf{rating}
associated with your substitute \textbf{ability} is significantly lower
than the one used for the rest of the \textbf{contest}.

In addition to secondary objectives, as in the above example, you may
engage in \textbf{asymmetrical round} to grant \textbf{augments} (see
above) to yourself or others.

\hypertarget{disengaging}{%
\subsubsection{5.1.5 Disengaging}\label{disengaging}}

You can always abandon a \textbf{contest}, but, in addition to failing
at the \textbf{story obstacle}, you may also suffer negative
consequences. In a \textbf{contest} where your opponent intends to harm
you, you will always suffer negative consequences if you withdraw,
unless you successfully disengage.

To disengage, you make an \textbf{asymmetrical round}, using the
\textbf{ability} relevant to the \textbf{contest} you're trying to
wriggle out of.

If you fail, your effort is wasted and the score against you increases,
as it would have during a normal \textbf{round}. If you succeed, you
escape the clutches, literal or metaphorical, of your opponent, without
further harm from a \textbf{contest} during the \textbf{rising action}.
In a \textbf{climactic} scene, however, \textbf{RPs} scored during
\textbf{contests} you disengaged from are still taken into account when
determining \textbf{consequences}. In the case of a \textbf{group
contest}, \textbf{consequences} against you are determined as soon as
you disengage.

\hypertarget{group-scored-contest}{%
\subsection{5.2 Group Scored Contest}\label{group-scored-contest}}

\textbf{Group scored contests} proceed as a series of \textbf{scored
contests} between pairs of PC and opponents, interwoven so that they
happen nearly simultaneously.

As in a \textbf{scored contest} between a single PC and an opponent,
only one \textbf{simple contest} per pair of adversaries occurs each
\textbf{round}. Usually the PCs make up one team, and their antagonists
the other.

A \textbf{group scored contest} continues until one side has no active
participants. If you \textbf{defeat} your opponent you can pair with a
new opponent. The new opponent might be unengaged, but might also be
engaged in an existing pairing. When you pair with a new opponent, you
begin a new \textbf{contest}, even if your opponent is already engaged
in a \textbf{contest}. Alternatively, if you are unopposed, you may
choose to \textbf{assist}. Of course, you may be later engaged by an
opponent who becomes free yourself.

You may lose some pairings amongst the PCs, but still win if the last
participant standing is a PC; otherwise if the last participant belongs
to the opposition you lose.

To determine the winning side's \textbf{victory} level, as opposed to
individual \textbf{outcomes}, if there is only one opponent, use their
\textbf{consequence of defeat}, otherwise, use the second-worst
\textbf{consequence} among the defeated opponents.

\hypertarget{procedure-4}{%
\subsubsection{5.2.1 Procedure}\label{procedure-4}}

\begin{enumerate}
\def\labelenumi{\arabic{enumi}.}
\tightlist
\item
  Your GM \textbf{frames the contest}.
\item
  You choose a \textbf{tactic}, and figure your PC's \textbf{target
  number} (\textbf{TN}) using the \textbf{rating} of your
  \textbf{ability}, plus or minus \textbf{modifiers} the GM may give
  you.
\item
  The GM determines the \textbf{resistance}. If two PCs contend, your
  opponent figures their \textbf{TN} as described in step 2.
\item
  The PCs to choose their opponents in order of their \textbf{TN} where
  it makes sense. Otherwise your GM will allocate opponents to you
  dependent on what makes narrative sense.
\item
  Establish an order of the paired \textbf{contests}. There is no
  significant advantage to going first, but use your group's
  \textbf{TN}s from highest to lowest if no other option presents
  itself.
\item
  For each pairing your GM carries out one \textbf{round}. Then they
  repeat by carrying out more \textbf{rounds} in order, as necessary.
  The Group \textbf{scored contest} ends as soon as there are no active
  participants on one side of the conflict. The side with one or more
  participants left standing wins.

  \begin{enumerate}
  \def\labelenumii{\arabic{enumii}.}
  \tightlist
  \item
    A \textbf{scored contest} unfolds as a series of \textbf{simple
    contests}. At the end of each \textbf{simple contest}, the winner
    scores a number of \textbf{resolution points (RPs)} to their tally,
    which varies between 1 and 5, depending on the result. Tied results
    leave the score unchanged.
  \item
    Your GM decides which opponent in a pair has the initiative and
    describes what they are trying to do to achieve the \textbf{prize},
    the `aggressor'. The `defender' describes how they counter the
    aggressor's attempt to seize the \textbf{prize}. If it is not
    obvious from the unfolding narrative, your GM should choose your PC
    as a the `aggressor'.
  \item
    Conduct a \textbf{simple contest} as normal, but once the
    \textbf{outcome} has been determined, it becomes a number of
    \textbf{resolution points} scored by the winning side.
  \item
    The number of \textbf{resolution points} the winner garners at the
    end of each \textbf{round} depends on the \textbf{degree of victory}
    they scored. They get 1 point for a \textbf{marginal victory}, 2 for
    a \textbf{minor victory}, 3 for a \textbf{major victory}, and 5 for
    a \textbf{complete victory}.
  \item
    The first to accumulate a total of 5 points wins; their opponent is
    knocked out of the \textbf{contest}.

    \begin{enumerate}
    \def\labelenumiii{\arabic{enumiii}.}
    \tightlist
    \item
      As one of a pair is eliminated from the \textbf{group contest},
      their victorious opponents may then move on to engage new targets,
      starting new \textbf{contests}, which are then added to the end of
      the existing sequence.
    \item
      If participating in multiple pairings, each pairing is the first
      to 5 points, points already scored do not count. But the
      accumulated points do count when determining \textbf{consequences}
      in the \textbf{climatic} phase.
    \end{enumerate}
  \end{enumerate}
\item
  Determine \textbf{degree of victory} based on \textbf{rising action}
  or \textbf{climax} (above).
\item
  Determine \textbf{benefits of victory} or \textbf{consequences of
  defeat}.
\item
  Describe the \textbf{outcome} based on the \textbf{story obstacle}.
\end{enumerate}

\hypertarget{group-scored-contest-outcomes}{%
\subsubsection{5.2.2 Group Scored Contest
Outcomes}\label{group-scored-contest-outcomes}}

In a \textbf{group scored contest} the side that has the last undefeated
contestant gains the \textbf{prize}.

If the PCs won, determine the group's overall \textbf{outcome} by using
the second-best \textbf{outcome} obtained by the PCs, or if there is
only one opponent, the \textbf{outcome}. If the PCs lost, determine the
group's overall \textbf{outcome} by using the second-worst outcome
obtained by the PCs, or if there is only one PC, the \textbf{outcome}.

\emph{For example, your PC Lieutenant Jackson of the Royal Navy has led
a shore-action against a French outpost. Lieutenant Jackson and two
other PCs have \textbf{victory} \textbf{outcomes} at the end of the
contest, so the Royal Navy wins the day. To determine how well the Royal
Navy has done your GM looks at the three \textbf{victorious}
\textbf{outcomes} for the Royal Navy, a \textbf{major victory}, a
\textbf{minor victory} and a \textbf{marginal victory}. The second best
outcome is a \textbf{minor victory} so your GM declares that the Royal
Navy have a \textbf{minor victory} and have overrun the French outpost,
but gained little else.}

\emph{Later you lead your men in a spirited defense against a French
boarding action of your ship. However, the French win the day, with
Lieutenant Jackson and the other PCs suffering \textbf{defeat}
\textbf{outcomes} at the end of the \textbf{contest}. Looking at your
PCs \textbf{outcomes} there is a \textbf{major defeat}, two
\textbf{minor defeats} and a \textbf{marginal defeat}. The French win
the day with a \textbf{minor defeat} for your Royal Navy crew.}

To determine individual \textbf{consequences} or \textbf{benefits}, in
\textbf{rising action}, even if you engage multiple opponents in a
\textbf{rising action scored contest}, only use the last opponent you
engaged to determine your individual \textbf{outcome}. In a climatic
contest total the \textbf{resolution points} scored against you by all
your opponents. If you engage more than one opponent, be sure to add the
\textbf{resolution points} scored against you by all of them. If you
lost, add 1 to your total. Your GM cross-references the total against
the \textbf{climactic state of adversity} table.

\hypertarget{unrelated-actions}{%
\subsubsection{5.2.3 Unrelated Actions}\label{unrelated-actions}}

If you are not currently enmeshed in a \textbf{round}, either after a
successful disengagement, or after winning a \textbf{round}, you may
take actions within the scene that do not directly contribute to the
\textbf{defeat} of the other side. These \textbf{unrelated actions} may
grant an \textbf{augment} to yourself or to a teammate. You may achieve
a secondary story objective. This resembles an \textbf{asymmetrical
round}, except that, as you are not targeted by any opponents, there is
no additional risk.

\hypertarget{assists}{%
\subsubsection{5.2.4 Assists}\label{assists}}

You may take an \textbf{unrelated action} to grant an \textbf{assist} to
a teammate enmeshed in a \textbf{round}. \textbf{Assists} are subject to
the same restrictions as \textbf{augments}: they must be both credible
and interesting.

Your first \textbf{assist} faces a \textbf{moderate resistance}. Each
subsequent \textbf{assist} attempt to the same beneficiary, steps up by
one factor on the table: high, then very high, then \textbf{nearly
impossible}. The \textbf{resistance} escalation occurs even when another
PC steps in to make a subsequent \textbf{assist}. This escalation allows
the occasional dramatic rescue but makes it difficult for players to
prolong losing battles to excruciating length. Your GM should make it
seem credible by justifying the increasing \textbf{resistances} with
descriptions of ever-escalating countermeasures on the part of the
opposition.

Your GM may adjust the starting \textbf{resistance} up or down by one
step to account for campaign credibility or other dramatic factors. If
an \textbf{assist} as proposed seems too improbable or insufficiently
useful, your GM should collaborate with you to propose alternate
suggestions which would face \textbf{moderate resistance}.

The \textbf{assist} alters the score against your teammate according to
the \textbf{outcome} of a \textbf{simple contest}

\hypertarget{assist-table}{%
\subsubsection{ASSIST TABLE}\label{assist-table}}

\begin{longtable}[]{@{}cc@{}}
\toprule
Contest Outcome & Change to Score Against Recipient\tabularnewline
\midrule
\endhead
Complete Victory & -4\tabularnewline
Major Victory & -3\tabularnewline
Minor Victory & -2\tabularnewline
Marginal Victory & -1\tabularnewline
Marginal Defeat & 0\tabularnewline
Minor Defeat & +1\tabularnewline
Major Defeat & +2\tabularnewline
Complete Defeat & +3\tabularnewline
\bottomrule
\end{longtable}

Scores can never be reduced below 0.

\hypertarget{followers}{%
\subsubsection{5.2.4 Followers}\label{followers}}

You may choose to have your \textbf{followers} take part in
\textbf{group scored contests} in one of three ways: as full
contestants, as secondary contestants, or as supporters.

\textbf{Contestant}: The \textbf{follower} takes part in the
\textbf{contest} as any other PC would. You roll for your
\textbf{followers} as you would their main characters. However, your
\textbf{followers} are removed from the \textbf{contest} whenever 3
\textbf{resolution points} are scored against them in a given
\textbf{round}. An additional 2 \textbf{resolution points} are then
scored against them, increasing the severity of any
\textbf{consequences} they suffer.

\textbf{Secondary contestant}: To act as a secondary contestant, your
\textbf{follower} must have an \textbf{ability} relevant to the
\textbf{contest}. The \textbf{follower} sticks by your side,
contributing directly to the effort: fighting in a battle, tossing in
arguments in a legal dispute, acting as the ship's navigator, or
whatever. Although you describe this, you do not roll for the
\textbf{follower}. Instead, you may, at any point, shift any number of
\textbf{resolution points} to a \textbf{follower} acting as a secondary
contestant. Followers with 3 or more \textbf{resource points} lodged
against them are removed from the scene. When a \textbf{follower} is
removed from the scene, an additional 2 \textbf{resolution points} are
lodged against them, increasing the severity of \textbf{consequences}
they suffer.

\textbf{Supporter}: Your \textbf{follower} is present in the scene, but
does not directly engage your opponents. Instead they may perform
\textbf{assists} and other \textbf{unrelated actions}.

\textbf{Followers} acting in any of these three capacities may be
removed from the \textbf{contest} by otherwise unengaged opponents. To
remove a \textbf{follower} from a scene, an opponent engages your
\textbf{follower} in a \textbf{simple contest}. Your GM sets the
\textbf{resistance}, or if it is another PC's \textbf{follower} they
determine the relevant \textbf{ability} of the \textbf{follower}
engaging yours. On any failure, your \textbf{follower} is taken out of
the \textbf{contest}. For \textbf{consequence} determination purposes,
the follower has X+2 \textbf{resolution points} lodged against them,
where X is the usual number levied by the \textbf{resolution point}
table.

\hypertarget{risky-gambits}{%
\subsubsection{5.2.5 Risky Gambits}\label{risky-gambits}}

During a \textbf{scored contest}, you can attempt to force a conflict to
an early resolution by making a \textbf{risky gambit}. If you win the
\textbf{round}, you lodge an additional 1 \textbf{resolution point}
against your opponent. However, if you lose the \textbf{round}, your
opponent lodges an additional 2 \textbf{resolution points} against you.

If both contestants engage in a \textbf{risky gambit}, the winner lodges
an additional 2 \textbf{resolution points} against the loser.

\hypertarget{defensive-responses}{%
\subsubsection{5.2.6 Defensive Responses}\label{defensive-responses}}

In a \textbf{scored contest}, you can make a \textbf{defensive
response}, lowering the number of \textbf{resolution points} lodged
against you in a \textbf{round}. If you win the \textbf{round}, the
number of \textbf{resolution points} you lodge against your opponent
decreases by 1. If you lose, your opponent lodges 2 fewer
\textbf{resolution points} against you. The total number of
\textbf{resolution points} assigned by a \textbf{round} is never less
than 0; there is no such thing as a negative \textbf{resolution point}.

\hypertarget{joining-scored-contests-in-progress}{%
\subsubsection{5.2.7 Joining Scored Contests in
Progress}\label{joining-scored-contests-in-progress}}

When you wish to join a \textbf{scored contest} in progress, you and
your GM should discuss whether you accept the current framing. If so,
you can participate. In a \textbf{scored contest}, you simply select an
opponent and enter into a new \textbf{round}. If you want to achieve
something other than the goal established during framing, you may
instead perform \textbf{unrelated actions}, including \textbf{assists}
and \textbf{augments}.

\hypertarget{switching-abilities}{%
\subsubsection{5.2.8 Switching Abilities}\label{switching-abilities}}

You may describe an action in a \textbf{scored contest} that is not
covered by the \textbf{ability} that you started the contest with. There
are two possibilities here: either you are trying to provide color to
your actions in the \textbf{round}, without seeking to gain advantage,
or you are seeking to gain advantage over your opponent with a novel
\textbf{tactic}. In the former case, you can continue to use the
\textbf{ability} you started the contest with, as you should not be
penalized for wanting to enhance the contest with colorful or
entertaining descriptions. In the latter case you should switch
\textbf{abilities}, and your GM must decide if the \textbf{resistance}
changes because of your new \textbf{ability}. Your GM is encouraged to
reward \textbf{tactics} that exploit weaknesses that have been
identified in the story so far with a lower \textbf{resistance}.
Sometimes your GM may respond with a higher \textbf{resistance} because
your \textbf{tactic} looks less likely to succeed due to conditions
already established in the story.

\hypertarget{extended-contest}{%
\subsection{5.3 Extended Contest}\label{extended-contest}}

\textbf{Extended contests} are longer and more dramatic than
\textbf{simple contests}. Your GM uses \textbf{extended contests} when
the \textbf{outcome} of the struggle is important, to generate suspense
for the players, or when they want a back-and-forth struggle. It is
something you and your GM should visualize and describe.

An \textbf{extended contest} consists of one or more \textbf{rounds}, in
which you perform actions that are similar to \textbf{simple contests}.
However, actions and \textbf{rounds} do not decide the \textbf{outcome}
of the whole \textbf{contest}, only who gains or loses \textbf{advantage
points (AP)} at that time. You take actions in turn, an
\textbf{exchange}, losing and gaining the advantage, until either you or
your opponent runs out of \textbf{advantage points} and is
\textbf{defeated}.

\hypertarget{procedure-5}{%
\subsubsection{5.3.1 Procedure}\label{procedure-5}}

\begin{enumerate}
\def\labelenumi{\arabic{enumi}.}
\tightlist
\item
  Your GM \textbf{frames the contest}.
\item
  You choose a \textbf{tactic}, and figure your PC's \textbf{target
  number} (\textbf{TN}) using the \textbf{rating} of your
  \textbf{ability}, plus or minus \textbf{modifiers} the GM may give
  you. Figure your starting \textbf{advantage point (AP)} total using
  the \textbf{TN}, including all \textbf{modifiers} and
  \textbf{augments}. The \textbf{AP} include +20 for each level of
  \textbf{mastery}, and can also be increased by \textbf{followers}.
\item
  The GM determines the \textbf{resistance}. The GM opposes the PC with
  a \textbf{resistance}---the harder the task or tougher the opponent,
  the higher the \textbf{resistance}. The GM figures starting
  \textbf{APs} for the \textbf{resistance} from the \textbf{resistance}
  \textbf{TN}.
\item
  Carry out one or more \textbf{rounds}, repeating as necessary.

  \begin{enumerate}
  \def\labelenumii{\arabic{enumii}.}
  \tightlist
  \item
    Each \textbf{round} consists of two \textbf{exchanges}: an action
    and immediate response.

    \begin{enumerate}
    \def\labelenumiii{\arabic{enumiii}.}
    \tightlist
    \item
      Describe your action towards the desired \textbf{prize}, what
      \textbf{ability} you use, and how much risk you take. ``I want to
      climb straight up to that outcrop, taking chances if needed.'' You
      can specify your \textbf{AP bid}; if you do not, your GM
      determines this based on the amount of risk you are taking. The
      size of the \textbf{bid} mirrors how bold and risky your
      character's action is. Extreme or aggressive actions mean a high
      \textbf{AP bid}, and cautious actions require less.
    \item
      The \textbf{ability} used can be varied, but \textbf{APs} are
      always calculated on the first \textbf{ability} that you use in a
      contest. That \textbf{ability} must be used in the first
      \textbf{round}.
    \item
      You gamble a number of your \textbf{APs} in an attempt to reduce
      your opponent's \textbf{AP}, but if you fail the attempt you lose
      the \textbf{AP}.
    \item
      Roll a die to determine your \textbf{degree of victory or defeat},
      then apply any \textbf{bumps}. Your GM does the same.
    \item
      Compare the results of the two die rolls on the Extended Contest
      Table to determine who loses \textbf{AP}; only when you have a
      \textbf{critical} can you gain \textbf{AP} from your opponent. The
      \textbf{AP bid} is multiplied by this number before applying the
      results. Thus, if you \textbf{bid} 3 \textbf{AP} and the result is
      ``Loser loses 2x \textbf{bid},'' the loser loses 3 x 2 = 6
      \textbf{AP}. Round half points up.
    \item
      Determine \textbf{outcome}. Each contestant's \textbf{AP} total
      rises or falls during the contest as they gain the upper hand or
      are driven back. \textbf{Exchanges} continue until one contestant
      reaches 0 \textbf{advantage points} or fewer. At that point, the
      contest is over (even if it is the middle of a \textbf{round}).
      The loser's final \textbf{AP} total determines the
      \textbf{outcome} for the victor.
    \item
      The GM then hazards a number of \textbf{APs} for the
      \textbf{resistance}, in the same way.
    \end{enumerate}
  \end{enumerate}
\item
  \textbf{Advantage points} are only relevant for the length of a
  particular \textbf{contest}. Your PC does not have any until the next
  \textbf{extended contest} begins, when you calculate them all over
  again
\end{enumerate}

\hypertarget{extended-contest-table}{%
\subsubsection{EXTENDED CONTEST TABLE}\label{extended-contest-table}}

\begin{longtable}[]{@{}ccclc@{}}
\toprule
\begin{minipage}[b]{0.18\columnwidth}\centering
\strut
\end{minipage} & \begin{minipage}[b]{0.17\columnwidth}\centering
Critical\strut
\end{minipage} & \begin{minipage}[b]{0.18\columnwidth}\centering
Success\strut
\end{minipage} & \begin{minipage}[b]{0.17\columnwidth}\raggedright
Failure\strut
\end{minipage} & \begin{minipage}[b]{0.18\columnwidth}\centering
Fumble\strut
\end{minipage}\tabularnewline
\midrule
\endhead
\begin{minipage}[t]{0.18\columnwidth}\centering
Critical\strut
\end{minipage} & \begin{minipage}[t]{0.17\columnwidth}\centering
Worse roll transfers ½x bid, else tie**\strut
\end{minipage} & \begin{minipage}[t]{0.18\columnwidth}\centering
Loser transfers 1x bid**\strut
\end{minipage} & \begin{minipage}[t]{0.17\columnwidth}\raggedright
Loser transfers 2x bid**\strut
\end{minipage} & \begin{minipage}[t]{0.18\columnwidth}\centering
Loser transfers 3x bid**\strut
\end{minipage}\tabularnewline
\begin{minipage}[t]{0.18\columnwidth}\centering
Success\strut
\end{minipage} & \begin{minipage}[t]{0.17\columnwidth}\centering
Loser transfers 1x bid**\strut
\end{minipage} & \begin{minipage}[t]{0.18\columnwidth}\centering
Worse roll loses ½x bid, else tie\strut
\end{minipage} & \begin{minipage}[t]{0.17\columnwidth}\raggedright
Loser loses 1x bid\strut
\end{minipage} & \begin{minipage}[t]{0.18\columnwidth}\centering
Loser loses 2x bid**\strut
\end{minipage}\tabularnewline
\begin{minipage}[t]{0.18\columnwidth}\centering
Failure\strut
\end{minipage} & \begin{minipage}[t]{0.17\columnwidth}\centering
Loser transfers 2x bid**\strut
\end{minipage} & \begin{minipage}[t]{0.18\columnwidth}\centering
Loser loses 1x bid\strut
\end{minipage} & \begin{minipage}[t]{0.17\columnwidth}\raggedright
Worse roll loses ½x bid, else tie\strut
\end{minipage} & \begin{minipage}[t]{0.18\columnwidth}\centering
Loser loses 1x bid**\strut
\end{minipage}\tabularnewline
\begin{minipage}[t]{0.18\columnwidth}\centering
Failure\strut
\end{minipage} & \begin{minipage}[t]{0.17\columnwidth}\centering
Loser transfers 3x bid**\strut
\end{minipage} & \begin{minipage}[t]{0.18\columnwidth}\centering
Loser loses 2x bid\strut
\end{minipage} & \begin{minipage}[t]{0.17\columnwidth}\raggedright
Loser loses 1x bid**\strut
\end{minipage} & \begin{minipage}[t]{0.18\columnwidth}\centering
Tie*\strut
\end{minipage}\tabularnewline
\bottomrule
\end{longtable}

\begin{itemize}
\tightlist
\item
  In a \textbf{group extended contest}, the GM may declare that both
  contestants lose 1⁄2x \textbf{bid} to indicate that, although their
  results cancel out with respect to each other, their situation worsens
  compared to other contestants. ** If the loser has \textbf{ability} 6
  less than their opponent or worse, do not transfer, loser just loses
  AP instead.
\end{itemize}

\hypertarget{advantage-points}{%
\subsubsection{5.3.2 Advantage Points}\label{advantage-points}}

\hypertarget{bidding-advantage-points}{%
\paragraph{5.3.2.1 Bidding Advantage
Points}\label{bidding-advantage-points}}

The size of your \textbf{AP bid} reflects the risk inherent in your
actions. You describe your action and intent, and say how many
\textbf{APs} you want to \textbf{bid}. If you describe an all-out
offensive with your sword cutting vicious arcs, you need to bid a lot of
\textbf{APs}; if you say that you are circling your foe cautiously, a
low \textbf{bid} is in order. Your GM will look at the level of risk you
are taking, and may suggest that you change your \textbf{bid} to better
match your actions. If you do not declare a \textbf{bid} before rolling
the die, your GM will decide how many points are \textbf{bid} (using 3
as a default), with riskier actions calling for higher \textbf{AP bids}.

\hypertarget{followers-and-advantage-points}{%
\paragraph{5.3.2.3 Followers and Advantage
Points}\label{followers-and-advantage-points}}

\textbf{Followers} can act in different ways during a \textbf{contest},
\textbf{augmenting} you with their \textbf{abilities} or allowing you to
use one of your \textbf{abilities} as if it were your own.
Alternatively, a \textbf{follower} with a relevant \textbf{ability} or
\textbf{keyword} can simply add their \textbf{APs} to the PC's at the
beginning of the \textbf{contest}.

Remember to figure any \textbf{modifiers} into your \textbf{follower's}
\textbf{ability} before adding it to your starting \textbf{AP} total.

Neither you nor the GM makes rolls for \textbf{followers}. Instead,
their actions are subsumed into yours. The \textbf{follower's} relevant
\textbf{ability} or \textbf{keyword} is used solely as a source of
\textbf{advantage points}.

You can assign your \textbf{followers} to someone else, although you may
have to succeed at a contest to persuade a reluctant follower to go
along.

\hypertarget{advantage-point-knowledge}{%
\paragraph{5.3.2.3 Advantage Point
Knowledge}\label{advantage-point-knowledge}}

Once your opponent has won or lost \textbf{APs} during the current
contest, you can ask the GM what the opposition's \textbf{AP} total is.
This is where the element of skill comes in. When choosing how many
\textbf{APs} to stake, you must weigh the effect they want to gain if
you succeed versus the risk you face if the action fails.

\hypertarget{extended-contest-outcomes}{%
\subsubsection{5.3.3 Extended Contest
Outcomes}\label{extended-contest-outcomes}}

At the end of the contest the \textbf{APs} of the loser determine the
\textbf{benefits} for the winner or \textbf{consequences} for the loser.
As with all \textbf{contests}, if the contest involved a
\textbf{resistance}, and not another PC, we care about your
\textbf{outcome}, win or lose, and the GM should feel free to narrate
the \textbf{outcome} for the \textbf{resistance} depending on their
interpretation of your \textbf{outcome}, which may not be symmetrical.
For example, if the \textbf{benefit of victory} for your PC is
\textbf{pumped} the GM should feel free to interpret what this means for
the \textbf{resistance}: in a melee they might be dead, in a social
contest they might be exiled, or they might surrender in the melee or
cede ground in a social contest. Your GM should focus on the
\textbf{prize} that was agreed when deciding how to narrate the
resolution of the contest.

\hypertarget{extended-contest-table-1}{%
\paragraph{EXTENDED CONTEST TABLE}\label{extended-contest-table-1}}

\begin{longtable}[]{@{}cccc@{}}
\toprule
Final AP Total & Level of Defeat & Consequence for Loser & Benefit for
Winner\tabularnewline
\midrule
\endhead
0 to --10 AP & Marginal & Hurt & Fresh\tabularnewline
--11 to --20 AP & Minor & Impaired & Pumped\tabularnewline
--21 to --30 AP & Major & Injured & Invigorated\tabularnewline
--31 or fewer AP & Complete & Dying & Heroic\tabularnewline
\bottomrule
\end{longtable}

\hypertarget{group-extended-contests}{%
\subsubsection{5.3.3 Group Extended
Contests}\label{group-extended-contests}}

When an \textbf{extended contest} involves three or more contestants, it
is a \textbf{group extended contest}. The conflict is often between two
groups; each side wants to knock the other out of the contest by
reducing all of its opponents to 0 or fewer \textbf{APs}.

Sometimes a contest will be a free-for-all involving three or more
groups.

Rounds in a \textbf{group extended contest} differ in that the order of
resolution is more complicated. At the start of the \textbf{round}, you
state your action and \textbf{AP bid} and single out one or more
opponents. Your GM then determines the order in which the contestants
act. Taking surprise, withdrawals, and similar situations into effect,
they have three options:

\begin{itemize}
\item
  Contestants can go in order from most daring to least daring
  \textbf{bid}: a reckless \textbf{bid} goes before a daring
  \textbf{bid}, as defined in ``Bidding Advantage Points'' above. Thus,
  the most heroic actions take precedence, acting in order of decreasing
  boldness. (In case of a tie, the contestant whose actual \textbf{bid}
  is higher goes first.)
\item
  Contestants can go in order from highest \textbf{bid} to lowest: a
  \textbf{bid} of 20 \textbf{APs} goes before a \textbf{bid} of 5
  \textbf{APs}. (In case of a tie, the contestant whose \textbf{bid} is
  the most daring goes first.)
\item
  Contestants can go in order from highest to lowest \textbf{AP} total.
  (In case of a tie, the highest or most daring \textbf{bid} goes
  first.)
\end{itemize}

During a standard \textbf{extended contest} an opponent immediately
responds to your action with their own, but in a \textbf{group extended
contest} this is not true---they cannot act (against you or anyone else)
until their turn comes. You may want to change your declared action if
another character attacks you first, and your GM will normally allow you
to do so, usually to return an attack in kind.

The order in which contestants act is also important because a character
(whether PC or \textbf{resistance} such as an NPC) can be knocked out of
the contest before their turn comes. If your chosen opponent is knocked
out before your PC acts, the GM decides if you can change your declared
action.

You always have the option of delaying and allowing other contestants to
act before you. You can jump back into the action at any time during the
\textbf{round}, although again your GM determines if you can change your
stated action.

When all characters still in the contest have completed their action the
\textbf{round} ends and a new one begins.

\hypertarget{group-extended-contest-outcomes}{%
\paragraph{5.3.2.5 Group Extended Contest
Outcomes}\label{group-extended-contest-outcomes}}

In a \textbf{group extended contest} the side that has the last
undefeated contestant gains the \textbf{prize}.

If the PCs won, determine the group's overall \textbf{outcome} by using
the second-best \textbf{outcome} obtained by the PCs, or if there is
only one opponent, the \textbf{outcome}. If the PCs lost, determine the
group's overall \textbf{outcome} by using the second-worst outcome
obtained by the PCs, or if there is only one PC, the \textbf{outcome}.

\emph{For example, your PC Lieutenant Jackson of the Royal Navy has led
a shore-action against a French outpost. Lieutenant Jackson and two
other PCs have \textbf{victory} \textbf{outcomes} at the end of the
contest, so the Royal Navy wins the day. To determine how well the Royal
Navy has done your GM looks at the three \textbf{victorious}
\textbf{outcomes} for the Royal Navy, a \textbf{major victory}, a
\textbf{minor victory} and a \textbf{marginal victory}. The second best
outcome is a \textbf{minor victory} so your GM declares that the Royal
Navy have a \textbf{minor victory} and have overrun the French outpost,
but gained little else.}

\emph{Later you lead your men in a spirited defense against a French
boarding action of your ship. However, the French win the day, with
Lieutenant Jackson and the other PCs suffering \textbf{defeat}
\textbf{outcomes} at the end of the \textbf{contest}. Looking at your
PCs \textbf{outcomes} there is a \textbf{major defeat}, two
\textbf{minor defeats} and a \textbf{marginal defeat}. The French win
the day with a \textbf{minor defeat} for your Royal Navy crew.}

To determine individual \textbf{consequences} or \textbf{benefits}, use
the \textbf{AP} of last opponent you engaged to determine your
individual \textbf{outcome}.

\hypertarget{parting-shot-1}{%
\subsubsection{5.3.4 Parting Shot}\label{parting-shot-1}}

When you \textbf{defeat} an opponent in an \textbf{extended contest},
you can act again immediately to try to make their \textbf{consequences
of defeat} more severe. This is called a \textbf{parting shot}. You once
again \textbf{bid} \textbf{AP} and use an appropriate \textbf{ability}
against your opponent. If you succeed, their \textbf{AP} will decrease;
their \textbf{outcome} may or may not change, but they cannot finish the
\textbf{round} by taking an action against you.

\textbf{Parting shots} are risky; if you fail, an \textbf{AP} transfer
might bring your opponent back into the \textbf{contest}. Your stumble
can give them an opening that they can exploit in an effort to snatch
\textbf{victory} from the jaws of \textbf{defeat}.

The \textbf{consequences of defeat} can remain after a \textbf{parting
shot}, if the GM chooses. Thus, an opponent might keep a
\textbf{penalty} from a defeat even if they are handed another chance by
their opponent's failed \textbf{parting shot}.

\hypertarget{final-action}{%
\subsubsection{5.3.5 Final Action}\label{final-action}}

If your PC falls to 0 or fewer \textbf{advantage points} in a standard
\textbf{extended contest}, you are \textbf{defeated}. In a \textbf{group
extended contest}, however, you can still try a \textbf{final action} to
stay in the \textbf{contest} as long as you are not \textbf{dying}
(which allows for no actions). A \textbf{final action} represents the
knack to come back when your opponent turns away to gloat or deal with
the other player characters. A character may only attempt one
\textbf{final action} in any \textbf{contest}.

To attempt a \textbf{final action}, you must be free from attention by
the opposition. You must spend a \textbf{hero point}. This does not
provide a \textbf{bump} up on the roll to come; it is the cost of
performing a \textbf{final action}. You can use a relevant
\textbf{ability} in a \textbf{simple contest} against the number of
\textbf{APs} your PC is below 0. Even if you succeed, the original
\textbf{consequences of defeat} still apply: a \textbf{hurt} still takes
a --3 to appropriate \textbf{abilities} until healed.

If you win the \textbf{simple contest}, you rejoin the contest with a
positive \textbf{AP} total. Your new total is a fraction of your
original \textbf{AP} total at the outset of the \textbf{contest}. If you
fail the \textbf{simple contest}, your \textbf{AP} total drops even
further, perhaps worsening your \textbf{outcome}.

Your GM should not use a \textbf{final action} for the
\textbf{resistance} (and has no \textbf{hero points} which are required
for this).

\hypertarget{final-action-results-table}{%
\subsubsection{FINAL ACTION RESULTS
TABLE}\label{final-action-results-table}}

\begin{longtable}[]{@{}cc@{}}
\toprule
Result & AP change\tabularnewline
\midrule
\endhead
Marginal Victory & Rejoin contest with 1/8 of your starting
APs\tabularnewline
Minor Victory & Rejoin contest with 1/4 of your starting
APs\tabularnewline
Major Victory & Rejoin contest with 1/2 of your starting
APs\tabularnewline
Complete Victory & Rejoin contest with full starting APs\tabularnewline
Marginal Defeat & Lose APs equal to 1/8 of your starting
value\tabularnewline
Minor Defeat & Lose APs equal to 1/4 of your starting
value\tabularnewline
Major Defeat & Lose APs equal to 1/2 of your starting
value\tabularnewline
Complete Defeat & Lose APs equal to your full starting
value\tabularnewline
\bottomrule
\end{longtable}

\hypertarget{desperation-stake}{%
\subsubsection{5.3.6 Desperation Stake}\label{desperation-stake}}

You can stake more \textbf{advantage points} than you currently have, to
a maximum of your starting \textbf{AP} total. This allows you to attempt
a \textbf{desperation stake} even when you are within a single
\textbf{AP} of \textbf{defeat}. Your GM can never stake more
\textbf{advantage points} than they have.

\hypertarget{unrelated-action}{%
\subsubsection{5.3.7 Unrelated Action}\label{unrelated-action}}

If you are unengaged, you can forfeit your action to do something
unrelated to the object of the contest. You might want to try to open a
door, haul an important piece of equipment out of your saddlebags, heal
yourself with magic, or \textbf{augment} an \textbf{ability}. Depending
on the circumstances, you may have to engage in a \textbf{simple
contest} to find out if you succeed at the \textbf{unrelated action}.

\hypertarget{asymmetrical-exchange}{%
\subsection{5.3.8 Asymmetrical Exchange}\label{asymmetrical-exchange}}

If you are engaged, you may choose to briefly suspend your attempt to
best your opponent in an \textbf{extended contest}, in order to do
something else. An instance where you are trying to do something else
and your opponent is trying to win the \textbf{contest} is called an
\textbf{asymmetrical exchange}.

In an \textbf{asymmetrical exchange}, you do not score \textbf{APs}
against your opponent if you win the \textbf{exchange}. Instead, you
succeed at whatever else you were doing. You still lose \textbf{AP} if
you fail. Often you will be using an \textbf{ability} other than the one
you've been waging the \textbf{contest} with, one better suited to the
task at hand. This becomes additionally dangerous when the
\textbf{rating} associated with your substitute \textbf{ability} is
significantly lower than the one used for the rest of the
\textbf{contest}.

In addition to secondary objectives, as in the above example, you may
engage in \textbf{asymmetrical exchange} to grant \textbf{augments} (see
above) to yourself or others.

\hypertarget{switching-abilities-1}{%
\subsubsection{5.3.9 Switching Abilities}\label{switching-abilities-1}}

You can usually switch freely from one \textbf{ability} to another in
the middle of an \textbf{extended contest}. It makes sense to do so if
you think a different \textbf{ability} will yield an advantage. You may
need to do an \textbf{unrelated action} to switch abilities---for
example, when changing weapons or equipment.

Your \textbf{AP} total stays the same when you change your
\textbf{ability}, so it makes sense to start the contest with your best
\textbf{ability} (appropriate to your goal, of course). If this seems
odd, remember that \textbf{advantage points} measure advantage---how
well the character is doing in the contest at the current moment. They
do not measure proficiency; that is what the \textbf{target number} is
for.

When you switch \textbf{abilities}, your \textbf{prize} does not change,
just the means by which you pursue it.

\hypertarget{disengaging-1}{%
\subsubsection{5.3.10 Disengaging}\label{disengaging-1}}

To disengage from an \textbf{extended contest} when your opponent is
actively trying to keep you in the conflict, take an \textbf{unrelated
action} to make a \textbf{simple contest} roll against the
\textbf{resistance}. You use an \textbf{ability} relevant to your
attempt to disengage; the opponent counters with the \textbf{resistance}
or, if a PC, an appropriate \textbf{ability}. If the GM attempts to
disengage, they use the \textbf{resistance} to do so. These
\textbf{abilities} may or may not be those used in the main
\textbf{contest}.

On any \textbf{victory}, you are able to leave the \textbf{contest}. On
any \textbf{defeat}, you must remain in the \textbf{contest}, and
transfer a fraction of your current \textbf{APs} to your opponent.

If you withdraw from a \textbf{group extended contest} and later decide
to rejoin it (or are forced to), you rejoin with the \textbf{advantage
point} total you had when you left. If you can show how your leaving and
returning substantially changes the situation, the GM may restore some
of your \textbf{AP}---for example, if you leave a street fight to get
your \textbf{followers} from a nearby tavern. Leaving a \textbf{contest}
just to pick up a weapon or catch your breath is an \textbf{unrelated
action}, and does not change your \textbf{advantage points}.

\hypertarget{ap-transfer-fron-failed-disengagement}{%
\subsubsection{AP TRANSFER FRON FAILED
DISENGAGEMENT}\label{ap-transfer-fron-failed-disengagement}}

\begin{longtable}[]{@{}cc@{}}
\toprule
Level of Defeat & AP transferred\tabularnewline
\midrule
\endhead
Marginal & 1/8 of your current total\tabularnewline
Minor & 1/4 of your current total\tabularnewline
Major & 1/2 of your current total\tabularnewline
Complete & Your current total - 1\tabularnewline
\bottomrule
\end{longtable}

\hypertarget{ap-lending}{%
\subsubsection{5.3.10 AP Lending}\label{ap-lending}}

\textbf{AP lending} is a common and important option in \textbf{extended
contests}. You can transfer some or all of your \textbf{advantage
points} to another PC engaged in a \textbf{group extended contest} on
your side. With more \textbf{advantage points}, they can stay in the
\textbf{contest} for longer, or make larger \textbf{bids} without
driving themselves to \textbf{defeat}.

You cannot lend \textbf{advantage points} to yourself.

If a \textbf{follower's AP} are already included in your \textbf{AP}
total, the \textbf{follower} cannot lend them to you.

Use an \textbf{unrelated action} and describe what your character is
trying to do to improve the position of the target. For example, your PC
might throw them a weapon, jeer at an opponent, or simply shout words of
encouragement. Then, state the number of \textbf{AP} you are trying to
\textbf{lend}. (The GM may suggest a higher or lower \textbf{bid} based
on the action you describe.) This determines the \textbf{resistance} you
face in a \textbf{simple contest}, with \textbf{outcomes} as determined
below. Beware: PCs trying to aid their comrades in this way risk
worsening their friend's position.

\hypertarget{ap-lending-1}{%
\subsubsection{AP LENDING}\label{ap-lending-1}}

\begin{longtable}[]{@{}cc@{}}
\toprule
\begin{minipage}[b]{0.24\columnwidth}\centering
Outcome\strut
\end{minipage} & \begin{minipage}[b]{0.70\columnwidth}\centering
AP Transferred\strut
\end{minipage}\tabularnewline
\midrule
\endhead
\begin{minipage}[t]{0.24\columnwidth}\centering
Complete Victory\strut
\end{minipage} & \begin{minipage}[t]{0.70\columnwidth}\centering
Target gains the attempted AP; lender does not lose AP\strut
\end{minipage}\tabularnewline
\begin{minipage}[t]{0.24\columnwidth}\centering
Major or Minor Victory\strut
\end{minipage} & \begin{minipage}[t]{0.70\columnwidth}\centering
Target gains the attempted AP; lender loses the AP\strut
\end{minipage}\tabularnewline
\begin{minipage}[t]{0.24\columnwidth}\centering
Marginal Victory\strut
\end{minipage} & \begin{minipage}[t]{0.70\columnwidth}\centering
Target gains ½ the attempted AP; lender loses ½ the attempted AP\strut
\end{minipage}\tabularnewline
\begin{minipage}[t]{0.24\columnwidth}\centering
Tie\strut
\end{minipage} & \begin{minipage}[t]{0.70\columnwidth}\centering
No Effect\strut
\end{minipage}\tabularnewline
\begin{minipage}[t]{0.24\columnwidth}\centering
Marginal Defeat\strut
\end{minipage} & \begin{minipage}[t]{0.70\columnwidth}\centering
Target gains nothing; lender loses ½ the attempted AP\strut
\end{minipage}\tabularnewline
\begin{minipage}[t]{0.24\columnwidth}\centering
Minor or Major Defeat\strut
\end{minipage} & \begin{minipage}[t]{0.70\columnwidth}\centering
Target gains nothing; lender loses the attempted AP\strut
\end{minipage}\tabularnewline
\begin{minipage}[t]{0.24\columnwidth}\centering
Complete\strut
\end{minipage} & \begin{minipage}[t]{0.70\columnwidth}\centering
Target and lender each deduct the attempted AP from their totals\strut
\end{minipage}\tabularnewline
\bottomrule
\end{longtable}

\hypertarget{ap-gifting}{%
\subsubsection{5.3.11 AP Gifting}\label{ap-gifting}}

If you are uninvolved in the contest you can also increase a
participant's \textbf{AP} total. You \textbf{bid} a number of
\textbf{APs} which may not exceed your \textbf{target number}. The
\textbf{resistance} is twice the \textbf{bid}. The amount transferred
depend on the \textbf{outcome}.

\hypertarget{ap-gifting-1}{%
\subsubsection{AP GIFTING}\label{ap-gifting-1}}

\begin{longtable}[]{@{}cc@{}}
\toprule
Outcome & AP Transferred\tabularnewline
\midrule
\endhead
Complete Victory & Recipient gains 2x bid\tabularnewline
Major or Minor Victory & Recipient gains bid\tabularnewline
Marginal Victory & Recipient gains 1/2 of bid\tabularnewline
Tie & No Effect\tabularnewline
Marginal Defeat & Recipient's opponent gains 1/2 bid\tabularnewline
Minor or Major Defeat & Recipient's opponent gains bid\tabularnewline
Complete & Recipient's opponent gains 2x bids\tabularnewline
\bottomrule
\end{longtable}

\hypertarget{edges-and-handicaps}{%
\subsubsection{5.3.12 Edges and Handicaps}\label{edges-and-handicaps}}

Your GM may want rules to represent opponents who strike rarely but with
great effect or who strike often but with little impact per blow. The
first quality can be represented with an \textbf{edge}; the second, with
a \textbf{handicap}. \textbf{Edges} and \textbf{handicaps} are
designated using \^{} (\^{}5, for example), \textbf{handicaps} with a
minus sign (--\^{}5).

\textbf{Edges} and \textbf{handicaps} affect only the \textbf{advantage
points bid} in an \textbf{extended contest}. Your \textbf{edge} is added
to your \textbf{AP bid} when your opponent must \textbf{lose} or
\textbf{transfer APs}. Your \textbf{handicap} is subtracted from your
bid when your opponent \textbf{loses} or \textbf{transfers APs}. A
contestant's \textbf{edge} or \textbf{handicap} never affects his
\textbf{AP} when he defends, only when he is attacking.

Most GMs find \textbf{edges} and \textbf{handicaps} more trouble than
they're worth, and depict these phenomena with description alone.
Earlier books made more extensive use of \textbf{edges} and
\textbf{handicaps} to represent the quality of equipment carried by the
PCs. For example, your suit for chainmail might be \^{}4 and your sword
\^{}3. In games where restricted access to equipment is a significant
part of the setting and your GM wants to use extended contests it may
make sense to use them, otherwise we recommend ignoring them.

\hypertarget{chained-contest}{%
\subsection{5.4 Chained Contest}\label{chained-contest}}

\textbf{Chained contests} do not defer \textbf{consequences} to the end
of the \textbf{contest}, instead your GM applies the
\textbf{consequences of defeat} to the loser in the \textbf{contest}
immediately following a \textbf{round}. This leads to a grittier feel to
the \textbf{contest}, but at the price of a death spiral: once you lose
the \textbf{consequences of defeat} make it likelier that you will lose
again.

To run an \textbf{extended contest} your GM runs a \textbf{simple
contest} as normal, and then applies the \textbf{outcome} according to
the following table, with the \textbf{consequences} taking immediate
effect.

You decide if you wish to continue the \textbf{chained contest}, and
your GM makes a similar determination for the \textbf{resistance}. Both
you and your GM then express your intent. If your or your GM wishes to
continue, play out another \textbf{simple contest}.

If you, or your GM, wishes to \textbf{disengage}, then on any
\textbf{victory} you leave the contest, without inflicting consequences
on the opposition. If both you and the GM wish to leave the contest,
then you both \textbf{disengage}, and the contest ends. If any
contestant is reduced to a \textbf{complete defeat}, the
\textbf{contest} ends automatically.

\textbf{Chained contests} are the simplest form of \textbf{long
contest}, and you may prefer them for that reason. They also tend to
produce the most extreme \textbf{outcomes}, as participants tend to
continue until \textbf{complete defeat}.

\textbf{Chained contests} are symmetric, in that they indicate the
\textbf{outcome} for the loser, and as this is applied in steps, give
the GM no freedom to interpret the \textbf{outcome} for the loser when
narrating.

\hypertarget{chained-contest-table}{%
\subsubsection{CHAINED CONTEST TABLE}\label{chained-contest-table}}

\begin{longtable}[]{@{}ccccc@{}}
\toprule
\begin{minipage}[b]{0.11\columnwidth}\centering
Roll\strut
\end{minipage} & \begin{minipage}[b]{0.19\columnwidth}\centering
Critical\strut
\end{minipage} & \begin{minipage}[b]{0.19\columnwidth}\centering
Success\strut
\end{minipage} & \begin{minipage}[b]{0.19\columnwidth}\centering
Failure\strut
\end{minipage} & \begin{minipage}[b]{0.19\columnwidth}\centering
Fumble\strut
\end{minipage}\tabularnewline
\midrule
\endhead
\begin{minipage}[t]{0.11\columnwidth}\centering
Critical\strut
\end{minipage} & \begin{minipage}[t]{0.19\columnwidth}\centering
Worse roll is \textbf{hurt}. If tied, no effect.\strut
\end{minipage} & \begin{minipage}[t]{0.19\columnwidth}\centering
Opponent \textbf{hurt}. If already \textbf{hurt} in this contest,
Injured. If already injured, Dying\strut
\end{minipage} & \begin{minipage}[t]{0.19\columnwidth}\centering
Opponent Injured. If already Injured in this contest, Dying\strut
\end{minipage} & \begin{minipage}[t]{0.19\columnwidth}\centering
Opponent Dying: player has them at complete mercy. Contest is
over.\strut
\end{minipage}\tabularnewline
\begin{minipage}[t]{0.11\columnwidth}\centering
Success\strut
\end{minipage} & \begin{minipage}[t]{0.19\columnwidth}\centering
PC is \textbf{hurt}. If already \textbf{hurt} in this contest, Injured.
If already injured, Dying\strut
\end{minipage} & \begin{minipage}[t]{0.19\columnwidth}\centering
Worse roll is \textbf{hurt}. If tied, no effect\strut
\end{minipage} & \begin{minipage}[t]{0.19\columnwidth}\centering
Opponent \textbf{hurt}. If already \textbf{hurt} in this contest,
Injured. If already injured, Dying\strut
\end{minipage} & \begin{minipage}[t]{0.19\columnwidth}\centering
Opponent Injured. If already Injured in this contest, Dying\strut
\end{minipage}\tabularnewline
\begin{minipage}[t]{0.11\columnwidth}\centering
Failure\strut
\end{minipage} & \begin{minipage}[t]{0.19\columnwidth}\centering
PC is Injured. If already Injured in this contest, Dying\strut
\end{minipage} & \begin{minipage}[t]{0.19\columnwidth}\centering
PC is \textbf{hurt}. If already \textbf{hurt} in contest, Injured. If
injured, Dying\strut
\end{minipage} & \begin{minipage}[t]{0.19\columnwidth}\centering
Worse roll is \textbf{hurt}. If tied, no effect\strut
\end{minipage} & \begin{minipage}[t]{0.19\columnwidth}\centering
Opponent \textbf{hurt}. If already \textbf{hurt} in this contest,
Injured. If already injured, Dying\strut
\end{minipage}\tabularnewline
\begin{minipage}[t]{0.11\columnwidth}\centering
Fumble\strut
\end{minipage} & \begin{minipage}[t]{0.19\columnwidth}\centering
PC Dying: opponent has them at complete mercy. Contest is over.\strut
\end{minipage} & \begin{minipage}[t]{0.19\columnwidth}\centering
PC is Injured. If already Injured in this contest, Dying\strut
\end{minipage} & \begin{minipage}[t]{0.19\columnwidth}\centering
PC is \textbf{hurt}. If already \textbf{hurt} contest, Injured. If
already injured, Dying\strut
\end{minipage} & \begin{minipage}[t]{0.19\columnwidth}\centering
Both make a mistake. No effect on contest. Side effects at GM's
discretion\strut
\end{minipage}\tabularnewline
\bottomrule
\end{longtable}

\hypertarget{group-chained-contest}{%
\subsubsection{5.4.1 Group Chained
Contest}\label{group-chained-contest}}

In a \textbf{group chained contest} opponents pair off and fight a
series of \textbf{chained contest} \textbf{rounds} with each other.

Your GM should determine the order of action, but as all rounds
represent actions by both aggressor and defender there is no advantage
to be obtained by going first. If there are surplus characters on your
side, you may engage an already engaged opponent in a second
\textbf{contest}; your GM may choose to apply a \textbf{penalty} to them
as they are already engaged with one opponent. Alternatively you may
choose to \textbf{augment} an existing player character, reflecting
aiding them in their fight instead.

\hypertarget{group-chained-contest-outcomes}{%
\subsubsection{5.4.2 Group Chained Contest
Outcomes}\label{group-chained-contest-outcomes}}

In a \textbf{group chained contest} the side that has the last
undefeated contestant gains the \textbf{prize}.

If the PCs won, determine the group's overall \textbf{outcome} by using
the second-best \textbf{outcome} obtained by the PCs, or if there is
only one opponent, the \textbf{outcome}. If the PCs lost, determine the
group's overall \textbf{outcome} by using the second-worst outcome
obtained by the PCs, or if there is only one PC, the \textbf{outcome}.

\emph{For example, your PC Lieutenant Jackson of the Royal Navy has led
a shore-action against a French outpost. Lieutenant Jackson and two
other PCs have \textbf{victory} \textbf{outcomes} at the end of the
contest, so the Royal Navy wins the day. To determine how well the Royal
Navy has done your GM looks at the three \textbf{victorious}
\textbf{outcomes} for the Royal Navy, a \textbf{major victory}, a
\textbf{minor victory} and a \textbf{marginal victory}. The second best
outcome is a \textbf{minor victory} so your GM declares that the Royal
Navy have a \textbf{minor victory} and have overrun the French outpost,
but gained little else.}

\emph{Later you lead your men in a spirited defense against a French
boarding action of your ship. However, the French win the day, with
Lieutenant Jackson and the other PCs suffering \textbf{defeat}
\textbf{outcomes} at the end of the \textbf{contest}. Looking at your
PCs \textbf{outcomes} there is a \textbf{major defeat}, two
\textbf{minor defeats} and a \textbf{marginal defeat}. The French win
the day with a \textbf{minor defeat} for your Royal Navy crew.}

Individual \textbf{consequences} or \textbf{benefits} will have already
been determined by the \textbf{chained contest} \textbf{outcomes} on
each \textbf{round}.

\hypertarget{followers-in-a-chained-contest}{%
\subsubsection{5.4.3 Followers in a Chained
Contest}\label{followers-in-a-chained-contest}}

\textbf{Followers} may augment your character in a \textbf{chained
contest}.

In addition, if you suffer a defeat in a round of a \textbf{chained
contest} you may transfer that \textbf{outcome} to a follower, but they
suffer a \textbf{state of adversity} one level worse than you would do,
so marginal becomes minor etc., and the \textbf{follower} is removed
from the \textbf{contest}.

\hypertarget{no-nesting}{%
\subsection{5.5 No Nesting}\label{no-nesting}}

Your GM should never ``nest'' one \textbf{long contest} inside another.
If a \textbf{long contest} is in progress and you want to perform an
action your GM should treat it as an \textbf{unrelated action}, or
disallow it completely during the current \textbf{contest}.

\hypertarget{extended-vs-scored-contests-vs-chained-contests}{%
\subsection{5.6 Extended vs Scored Contests vs Chained
Contests}\label{extended-vs-scored-contests-vs-chained-contests}}

We recommend that your GM chooses one form of \textbf{long contest}
only, and sticks to it, within a given campaign of \emph{QuestWorlds}.
If in doubt, use a \textbf{scored contest} by default.

\textbf{Scored contests} have the advantage of speed and simplicity.
\textbf{Extended contests} have the advantage of each \textbf{exchange}
allowing both parties to take turns acting, over your GM adjudicating
who has the initiative; the bidding system also adds drama.
\textbf{Chained contests} offer the benefit of grittier exchanges where
the \textbf{outcomes} of each \textbf{round} have impact, as opposed to
being `cosmetic' until the end of the \textbf{contest}.

\textbf{Scored contests} require more interpretation by your GM, to
determine who has the initiative and describe the nature of the next
\textbf{round}. \textbf{Extended contests} drama comes at the cost of
increased complexity, and some harder to interpret corner cases.
\textbf{Chained contests} create a death spiral which can be hard to
break out of.

Both \textbf{scored contests} and \textbf{chained contests} can be used
as an \textbf{escalating contest}, see §2.8.2

\hypertarget{extremely-long-contests}{%
\subsection{5.7 Extremely Long Contests}\label{extremely-long-contests}}

There's no particular time scale associated with \textbf{contests}. But
some \textbf{contests} may by their very nature be a drama that can't be
resolved at one point in the narrative. Examples include political
campaigns, construction projects, or seductions. These can be resolved
by \textbf{long contests} where each \textbf{round} is conducted at an
appropriate moment, rather than sequentially. Your GM will need to keep
track of the \textbf{resolution or advantage points} and the
\textbf{resistance}, though this might change as the context changes (a
civil war started by the players could impede their castle-building
plans). The challenges of each round will vary, and you may use a
different \textbf{ability} or \textbf{augment} in the next exchange.

\hypertarget{relationships}{%
\section{6.0 Relationships}\label{relationships}}

Abilities may represent your relationship to NPCs.

\hypertarget{supporting-characters}{%
\subsection{6.1 Supporting Characters}\label{supporting-characters}}

Many relationships connect you to NPCs controlled by the GM.

When you try to use one of these relationships to solve a problem, your
\textbf{tactic} is your relationship \textbf{ability}. You can't simply
go to the \textbf{supporting character} you have a relationship with,
stick them with the problem, and expect to see it solved.

If you succeed, the \textbf{supporting character} helps you solve the
problem. If you fail, they don't. As with any \textbf{ability}, you must
still specify how the NPC goes about overcoming the \textbf{story
obstacle}. Calls on relationships are almost always \textbf{simple
contests}.

In crucial situations, it may seem dramatically inappropriate for you to
solve a problem indirectly, by working through others. Your GM can
expose the \textbf{supporting character} to serious risk. If the
character dies or otherwise suffers a change of status that renders them
useless to you, you permanently lose the relationship \textbf{ability}.

Before putting \textbf{supporting characters} at serious risk, your GM
should make sure the players understand the magnitude of the possible
consequences.

When \textbf{supporting characters} undertake significant risk, the
\textbf{supporting character} may suffer a \textbf{consequence of
defeat} commensurate with the level of the \textbf{defeat} in the
\textbf{contest}. Or it may simply be your relationship that is damaged
or destroyed.

\hypertarget{allies}{%
\subsection{6.2 Allies}\label{allies}}

An \textbf{ally} is a character of roughly the same level of
accomplishment as you, often in the same or a similar line of work. For
every favor you ask of them they'll ask one of you. These reciprocal
favors will be roughly equivalent in terms of risk, time commitment,
resistance class, and inconvenience.

\hypertarget{patrons}{%
\subsection{6.3 Patrons}\label{patrons}}

\textbf{Patrons} enjoy greater access to assets than you, either through
personal ownership (as in a Merchant Prince) or authority (as in the
governor of a province). They may lend you advice or provide you with
assets but are too busy and important to personally perform tasks for
you. They may hire you to do jobs, or issue orders within a command
structure to which you both belong.

When you roll your \textbf{patron} relationship, your GM adjusts the
resistance class depending on what you have done for them lately.

\hypertarget{contacts}{%
\subsection{6.4 Contacts}\label{contacts}}

A \textbf{contact} is a specialist in an \textbf{occupation}, skill, or
area of expertise. \textbf{Contacts} provide your information and
perform minor favors, but will expect information or small favors from
you in return.

You can describe a \textbf{contact} as being a particular individual, or
as a group of similar individuals.

\hypertarget{occupational-contacts}{%
\subsubsection{6.4.1 Occupational
Contacts}\label{occupational-contacts}}

Any \textbf{occupational keyword} can be treated as a source of
\textbf{contacts}. However, using an \textbf{occupational keyword} as a
source of \textbf{contacts} will always be a \textbf{stretch}. To more
reliably draw on particular \textbf{contacts} associated with your
occupation, you should take an explicit ability. Use a \textbf{breakout
ability} if you are using \textbf{umbrella keywords}.

\hypertarget{followers-1}{%
\subsection{6.5 Followers}\label{followers-1}}

A \textbf{follower} is a \textbf{supporting character} that travels with
you and contributes on a regular basis to your success.

There are two types of followers: \textbf{sidekicks} and
\textbf{retainers}.

\textbf{Followers} need not be people, or even sentient beings: you can
write up a spirit guardian, trusty robot, or companion animal as a
\textbf{follower}.

\hypertarget{sidekick}{%
\subsubsection{6.5.1 Sidekick}\label{sidekick}}

A \textbf{sidekick} is a \textbf{supporting character} under your
control. Most of the time they stay at your side to render assistance,
but they can also go off and perform errands or missions on their own.

You should give your \textbf{sidekick} a name. You should, when asked,
explain how the \textbf{sidekick} came to be your \textbf{follower}, and
why they continue in that role.

\textbf{Sidekicks} start with three \textbf{abilities}, one rated at 16
and the others at 13. Any of these \textbf{abilities} may be a
\textbf{keyword}. At least one of them should indicate a
\textbf{distinguishing characteristic}.

If the sidekick is nonhuman or a member of an unusual culture, one of
its three starting \textbf{abilities} must be its species or culture
\textbf{keyword}.

Once you have determined the \textbf{sidekick's} base
\textbf{abilities}, they allocate 15 additional points between three of
them, spending no more than 10 on any one \textbf{ability}.

You can improve these \textbf{abilities} through the expenditure of
\textbf{hero points}.

You may use any of your \textbf{sidekick's abilities} as your own. The
\textbf{sidekick} can go off and do things without you.

\hypertarget{replacing-lost-sidekicks}{%
\subsubsection{6.5.2 Replacing Lost
Sidekicks}\label{replacing-lost-sidekicks}}

As a \textbf{consequence of defeats} in which they participated,
\textbf{sidekicks} can be killed or leave your service permanently.

Defeat in physical \textbf{contests} can lead to literal death.
Metaphorical deaths from non-violent \textbf{contests} indicate a break
you. The \textbf{sidekick} may angrily withdraw from your service, but
is more likely to sorrowfully retire. You may be able to bring a
\textbf{sidekick} back from metaphorical death by overcoming
\textbf{story obstacles}.

If you lose a \textbf{sidekick}, you may create a new one without
needing to spend a \textbf{hero point}. You must explain how the new
\textbf{sidekick} has come to be your new \textbf{follower}.

You may find it convenient to promote \textbf{retainers} to
\textbf{sidekick} status, giving them names and personalities, with a
sudden improvement in \textbf{abilities} and \textbf{ratings} to match.

\hypertarget{retainers}{%
\subsubsection{6.5.3 Retainers}\label{retainers}}

A \textbf{retainer} is a more or less anonymous servant or helper. You
may specify a single \textbf{retainer}, or, where appropriate to your
character concept, an entire staff of them.

Like any other \textbf{ability}, a \textbf{retainer} \textbf{ability}
allows you to overcome relevant \textbf{story obstacles} by engaging in
a \textbf{contest}. To model the contribution of \textbf{retainers},
when you are acting, you can use them to \textbf{augment} your
\textbf{ability}. Your GM can rule that \textbf{consequences of defeat}
apply to \textbf{retainers}.

\textbf{Retainers} generally regard you with all the affection and
loyalty due to an employer or master. If you treat them more poorly than
is expected for their culture, your GM should increase the
\textbf{resistance class} of attempts to make use of their talents.

If you lose \textbf{retainers} for any reason, you can replace them
simply by providing a convincing explanation of how you go about it.

\hypertarget{relationships-as-flaws}{%
\subsection{6.6 Relationships as Flaws}\label{relationships-as-flaws}}

Certain relationships with \textbf{supporting characters} act as
\textbf{flaws}. They impose obligations on you, prompting your GM to
present you with \textbf{story obstacles} you have no choice but to
overcome.

\hypertarget{dependents}{%
\subsubsection{6.6.1 Dependents}\label{dependents}}

A \textbf{dependent} is a person, usually a family member or loved one,
who requires your aid and protection. Your GM should periodically create
storylines in which your \textbf{dependent} is endangered.

Rather than taking a \textbf{dependent} as a \textbf{flaw}, you may find
it more fruitful to specify the nature of your relationship as an
\textbf{ability}, such as \emph{Love for Wife} or \emph{Love for Son}.

\hypertarget{adversaries}{%
\subsubsection{6.6.2 Adversaries}\label{adversaries}}

An \textbf{adversary} is a rival, enemy or other individual who can be
relied upon to periodically disrupt your plans.

The \textbf{adversary's} goals are probably the opposite of yours,
although they could be a bitter rival within the same community,
organization, or movement.

To treat an \textbf{adversary} as an \textbf{ability}, rather than a
\textbf{flaw}, describe your emotional response to them. Examples:
\emph{Hates Leonard Crisp}, \emph{Fears the Electronaut}, \emph{Sworn
Vengeance Against Heimdall}. That way, you still inspire your GM to add
the plot elements you desire, but can use your antipathy toward the
enemy to \textbf{augment} your \textbf{target number}s against them.

\hypertarget{hero-points}{%
\section{7.0 Hero Points}\label{hero-points}}

\textbf{Hero points (HP)} are a resource that you must carefully
allocate. They allow you to heighten your \textbf{victories} and dull
your \textbf{defeats}. They are the currency you pay to improve your
\textbf{abilities} over time.

\hypertarget{improving-your-character}{%
\subsection{7.1 Improving Your
Character}\label{improving-your-character}}

You start each session with one \textbf{HP}. Any time that you take a
significant action the GM can award you another \textbf{hero point}. The
action should involve a \textbf{contest} but need not be successful. An
action should be heroic or villainous in order to earn a \textbf{hero
point}, not dull. Are the other players interested in what just
happened? Don't trigger \textbf{contests} just to win \textbf{hero
points}, your actions should drive story or character development
forward.

During a session you can spend those \textbf{HPs} as normal.

Your GM should award a maximum of 5 \textbf{HPs} in a session to you.

Unspent \textbf{HPs} at the end of the session become \textbf{experience
points (XPs)} and accumulate between sessions.

When you accumulate 10 \textbf{XPs}, you can buy an advance. An advance
allows you to select two of the following. You cannot choose an element
more than once.

\begin{itemize}
\tightlist
\item
  {[}{]} +9 to a standalone \textbf{ability} or breakout
  \textbf{ability}; or +6 to a \textbf{keyword}.
\item
  {[}{]} +6 to a standalone \textbf{ability} or breakout
  \textbf{ability}; or +3 to a \textbf{keyword}.
\item
  {[}{]} a new standalone \textbf{ability} at 13; or a new breakout
  \textbf{ability} at + 1.
\item
  {[}{]} a new standalone \textbf{ability} at 13.
\item
  {[}{]} Turn a stand-alone \textbf{ability} into a \textbf{keyword} by
  adding a new +1 breakout \textbf{ability} to it.
\end{itemize}

In some genres you may wish to maintain a tally of the total
\textbf{XPs} earned as a measure of your reputation.

\hypertarget{catch-ups}{%
\subsubsection{7.1.1 Catch-Ups}\label{catch-ups}}

To encourage well-rounded characters, a package deal, called a
\textbf{catch-up}, becomes available whenever you acquire via
improvement a new \textbf{mastery} in one of your \textbf{abilities}
(\textbf{keyword} or stand-alone). Any time you one of your
\textbf{ratings} crosses a \textbf{mastery} threshold (i.e.~20
-\textgreater{} 21, 40 -\textgreater{} 41, etc). you may also improve up
to three \textbf{abilities} or \textbf{keywords} of your choice increase
by three points each, as long as the chosen \textbf{abilities} are
currently rated five or more points lower than your newly adjusted
\textbf{rating} in the raised \textbf{ability} that triggered the
\textbf{catch-up}.

You may not increase the bonus of \textbf{breakout abilities} under a
\textbf{keyword} with a \textbf{catch-up}, nor does net effective value
of a breakout \textbf{ability} crossing a \textbf{mastery} threshold
trigger a \textbf{catch-up}. Only a \textbf{keyword}'s base
\textbf{rating} is considered in this context.

\hypertarget{directed-improvements}{%
\subsubsection{7.1.2 Directed
Improvements}\label{directed-improvements}}

On occasion your GM may increase one of your \textbf{abilities}, by +3,
+6 or +9, or give you a new \textbf{ability}, usually rated at 13. These
are called \textbf{directed improvements}.

\textbf{Directed improvements} are usually rewards for overcoming
particularly important or dramatic \textbf{story obstacles}. They happen
immediately, rather than at session's end.

Your GM will tend to use them to raise \textbf{abilities} that would
otherwise fall behind, but should increase due to story logic, or
introduce new \textbf{abilities} for the same reason.

\hypertarget{advanced-hero-points}{%
\subsection{7.2 Advanced Hero Points}\label{advanced-hero-points}}

An advanced option for \textbf{hero points} allow greater player
authorship in the game.

\hypertarget{plot-edits}{%
\subsection{7.3 Plot Edits}\label{plot-edits}}

\emph{QuestWorlds} is a co-operative game, and you may create details
about the setting as the normal part of narration. Your GM should allow
this, as long as they do not break credibility. So, you may describe
your PC walking over to the pot of soup bubbling on the fire, swiping a
drink from the tray the waiter is carrying at the governor's ball, or
taking the monorail to the next city to continue your investigation.
Your GM should allow these additions without interruption, providing it
does not confer significant advantage to your PC. Mostly this will be
using elements that have already been established as part of the
setting.

A \textbf{plot edit} is a more significant moment of good fortune that
you wish to narrate, that provides advantage to your PC. You are not
just describing something that is plausible in the environment, but
something whose existence aids you in overcoming \textbf{story
obstacles} or uncovering secrets.

A \textbf{plot edit} might be thought of as `fate' or `luck.'

Spending \textbf{hero points} for a \textbf{plot edit} allows you to
modify the setting or environment in your PC's favor. The chance
encounter in the street with an NPC, favorable weather, car keys in the
sun visor, the forthcoming eclipse, the wind that fills the sails.

Your GM is the arbitrator of whether a \textbf{plot edit} is allowed. It
should not suspend the disbelief of the other players in the game or
setting or hamper their enjoyment. It should not derail or short-circuit
the game's entertainment. The \textbf{plot edit} should, by contrast, be
something that enhances the story for all the players.

The cost, in \textbf{hero points}, of a \textbf{plot edit}, is given by
the following table.

\hypertarget{plot-edit-table}{%
\subsubsection{PLOT EDIT TABLE}\label{plot-edit-table}}

\begin{longtable}[]{@{}cccc@{}}
\toprule
\begin{minipage}[b]{0.28\columnwidth}\centering
Level\strut
\end{minipage} & \begin{minipage}[b]{0.06\columnwidth}\centering
Cost\strut
\end{minipage} & \begin{minipage}[b]{0.28\columnwidth}\centering
Impact\strut
\end{minipage} & \begin{minipage}[b]{0.28\columnwidth}\centering
Example\strut
\end{minipage}\tabularnewline
\midrule
\endhead
\begin{minipage}[t]{0.28\columnwidth}\centering
Minor\strut
\end{minipage} & \begin{minipage}[t]{0.06\columnwidth}\centering
1\strut
\end{minipage} & \begin{minipage}[t]{0.28\columnwidth}\centering
A credible change that does not alter the situation but offers an edge
that could be exploited\strut
\end{minipage} & \begin{minipage}[t]{0.28\columnwidth}\centering
The space suit your PC grabbed from the rack during the escape was a
belter's suit with powerful headlamps\strut
\end{minipage}\tabularnewline
\begin{minipage}[t]{0.28\columnwidth}\centering
Moderate\strut
\end{minipage} & \begin{minipage}[t]{0.06\columnwidth}\centering
2\strut
\end{minipage} & \begin{minipage}[t]{0.28\columnwidth}\centering
A substantive change that does not alter the situation but offers an
alternate avenue for resolution\strut
\end{minipage} & \begin{minipage}[t]{0.28\columnwidth}\centering
The gate guard at the secret government facility tonight is an old war
buddy established by the PC in a prior scene and cemented as a
relationship\strut
\end{minipage}\tabularnewline
\begin{minipage}[t]{0.28\columnwidth}\centering
Major\strut
\end{minipage} & \begin{minipage}[t]{0.06\columnwidth}\centering
3\strut
\end{minipage} & \begin{minipage}[t]{0.28\columnwidth}\centering
A substantive change that does not flow from previously established
facts in the story. A \emph{deus ex machina} change\strut
\end{minipage} & \begin{minipage}[t]{0.28\columnwidth}\centering
The XO of the Patrol ship is an old drinking buddy of your PC, a fact
not previously established in play\strut
\end{minipage}\tabularnewline
\begin{minipage}[t]{0.28\columnwidth}\centering
Extreme\strut
\end{minipage} & \begin{minipage}[t]{0.06\columnwidth}\centering
5\strut
\end{minipage} & \begin{minipage}[t]{0.28\columnwidth}\centering
A stroke of good fortune that is unrelated to prior events and resolves
a conflict or reveals a secret\strut
\end{minipage} & \begin{minipage}[t]{0.28\columnwidth}\centering
The vampire has failed to notice the approaching sun rise, which
disintegrates them just as they are about to drain the incapacitated
PC\strut
\end{minipage}\tabularnewline
\bottomrule
\end{longtable}

\hypertarget{advanced-community-resources-and-support}{%
\section{8.0 (Advanced) Community Resources and
Support}\label{advanced-community-resources-and-support}}

Some series revolve around the relationship between a band of
influential figures and the community they protect. In defense of the
community, they can \textbf{bolster}, expend, and juggle its various
\textbf{resources}.

These advanced rules allow your GM to track the rise and fall of the
fortunes of your community, and your impact on them.

If your GM intend to play a game centered around a community, you should
have a relationship \textbf{ability} to that community.

It is possible that you have relationships with other communities that
are not the focus of play. Treat these relationships as
\textbf{abilities} that you can call on, but your GM should not track
these communities with these rules. Your GM should pick the level of
community that provides the greatest dramatic potential from its
competition for \textbf{resources}, friendly or otherwise, with its
rivals.

Some campaigns do not center on a community, with the adventurers being
footloose wanderers. In that case, even if you have community
\textbf{abilities}, your GM will not track any community. Before you
decide this though, consider where your PCs might turn for help, succor,
or aid. Is there somewhere in the campaign defined as a place of refuge
and safety for you. It may well be that there is a community, the bar
where other footloose adventurers all meet, who will help each other out
in a tight spot for example, that your GM can model.

\hypertarget{community-design}{%
\subsection{8.1 Community Design}\label{community-design}}

\hypertarget{defining-resources}{%
\subsubsection{8.1.1 Defining Resources}\label{defining-resources}}

Communities can have a type of \textbf{ability} called a
\textbf{resource} that your GM defines. Your PC can try to draw on their
community's \textbf{resources} to use them as \textbf{abilities}. Your
GM should focus on no more than five or so broadly-labeled
\textbf{resource} types, so that the PCs can care about (and have a
chance of successfully managing) all of them.

Most communities have variants of the following \textbf{resources},
perhaps with more colorful names:

\begin{itemize}
\tightlist
\item
  Wealth --- the \textbf{ability} of the community to provide financial
  help, whether counted primarily in dollars, credits, or cattle
\item
  Diplomacy --- the \textbf{ability} to extract favors from other
  communities, while minimizing the cost of its reciprocal obligations
\item
  Morale --- the community's \textbf{ability} to believe in its capacity
  to achieve its goals, and willingness to follow the directives of its
  leaders
\end{itemize}

The following abilities might appear, depending on setting:

\begin{itemize}
\tightlist
\item
  Military --- its \textbf{ability} to defend itself from outside
  threats, and to aggressively achieve its own aims through force of
  arms (for settings where communities of the size you're tracking field
  their own armed units)
\item
  Magic --- the collective \textbf{ability} of its people to perform
  supernatural acts (for fantasy worlds)
\item
  Technology --- its access to specialized, rare or secret devices or
  scientific knowledge not shared by its rivals (for post- apocalyptic
  or SF worlds)
\end{itemize}

Similar communities in the genre, should have the same set of
\textbf{resources}.

\hypertarget{specify-an-interval}{%
\subsubsection{8.1.2 Specify an interval}\label{specify-an-interval}}

Your GM chooses a suitable interval to mark changes in
\textbf{resources}. For genres bound by the agricultural season, this is
usually a season, for a military genre it might be a campaign, for a
ship a voyage.

\hypertarget{assigning-ability-ratings-1}{%
\subsubsection{8.1.3 Assigning Ability
Ratings}\label{assigning-ability-ratings-1}}

Your GM distributes the following \textbf{ratings} between the five
abilities: 12W, 9W, 18, 18, and 12. Note that the size of the group
doesn't affect the \textbf{ratings}.

Your GM may create a questionnaire that asks the players to make choices
about the history of their community. They can choose their
multiple-choice answers by consensus, majority vote, or take turns. Each
question secretly assigns a score to one or more resource types. When
you're done, rank the \textbf{resources} in the order of the scores,
assigning the high \textbf{ratings} to the highest questionnaire results
and the lowest to the low.

A questionnaire introduces your setting in a punchy, interactive format,
and tailors the community to the players' desires, increasing their
investment in it.

\hypertarget{resource-notation}{%
\subsubsection{8.1.4 Resource Notation}\label{resource-notation}}

Your GM will keep track of \textbf{modifiers} to community
\textbf{resources} with a copy of the following record sheet. They will
use a pencil, because the numbers will fluctuate.

Your GM lists the names and \textbf{ratings} of your chosen
\textbf{resources} in the first row. Under the total column for each,
your GM will list the total current modifier. Under the PC column, your
GM lists \textbf{bonuses} resulting from PC activities (as opposed to
un-cemented \textbf{background events}.) When PCs \textbf{cement a
background benefit}, your GM adds its bonus to the PC column.

When PC activity reduces a \textbf{penalty} but does not eliminate it,
your GM will alter the entry under the Total column to reflect the
reduction, but leave the PC column blank.

\hypertarget{resource-notation-table}{%
\subsubsection{RESOURCE NOTATION TABLE}\label{resource-notation-table}}

\begin{longtable}[]{@{}llllllllll@{}}
\toprule
Total & PC & Total & PC & Total & PC & Total & PC & Total &
PC\tabularnewline
\midrule
\endhead
& & & & & & & & &\tabularnewline
\bottomrule
\end{longtable}

\hypertarget{drawing-on-resources}{%
\subsection{8.2 Drawing on Resources}\label{drawing-on-resources}}

You can use community \textbf{resources} as \textbf{abilities} after
convincing the community to let you expend precious assets. This
requires a preliminary \textbf{contest} using a social \textbf{ability},
most likely your community relationship. Your GM will use a
\textbf{moderate resistance} as the baseline, with higher
\textbf{resistance}s when your proposals seem selfish or likely to fail,
and lower ones when everyone but the dullest dolt would readily see
their collective benefits. Your GM may increase \textbf{resistance}s if
your group draws constantly on community \textbf{resources} without
replenishing them.

The lobbying effort and the actual resource use require framing, a clear
description of what you are doing, and other details to bring them to
fictional life. You can use \textbf{resource abilities} directly, or to
\textbf{augment} your own \textbf{abilities}.

Unlike character abilities, each use of community \textbf{resources}
temporarily \textbf{depletes} it.

On a \textbf{victory}, you win the \textbf{prize} specified by
\textbf{contest framing}, and a \textbf{penalty} is applied to
subsequent uses of the \textbf{resource}.

On a \textbf{defeat}, you lose the \textbf{prize} and an even more
severe \textbf{penalty} is applied to subsequent \textbf{resource} uses.
If you fail to secure the \textbf{prize} you were seeking, the depletion
\textbf{penalty} is also applied to your social and community
\textbf{abilities} when interacting with members of your community. This
reflects community displeasure at your fruitless expenditure.

\textbf{Penalties} from the Resource Depletion Table replace standard
\textbf{penalties} for \textbf{defeat}, not add to them.

Like other \textbf{modifiers} to \textbf{resources}, depletion
\textbf{penalties} end at the end of the current interval. These include
\textbf{depletion penalties} applied to character \textbf{abilities}.
However, a \textbf{depletion penalty} left unattended at the end of the
interval can result in a permanent drop in the relevant
\textbf{resource}.

If your GM wants resource depletion to lead to longer-lasting social
\textbf{penalties}, at the cost of some extra bookkeeping, they can have
the characters shed a 3-point \textbf{penalty} at the end of each
interval.

\hypertarget{resource-depletion-table}{%
\subsubsection{RESOURCE DEPLETION
TABLE}\label{resource-depletion-table}}

\begin{longtable}[]{@{}cc@{}}
\toprule
Contest Outcome & Depletion Penalty\tabularnewline
\midrule
\endhead
Complete Victory & 0\tabularnewline
Major Victory & -3\tabularnewline
Minor Victory & -3\tabularnewline
Marginal Victory & -3\tabularnewline
Marginal Defeat & -6\tabularnewline
Minor Defeat & -6\tabularnewline
Major Defeat & -6\tabularnewline
Complete Defeat & -9\tabularnewline
\bottomrule
\end{longtable}

\hypertarget{required-resource-use}{%
\subsubsection{8.2.1 Required Resource
Use}\label{required-resource-use}}

As part of your GM's setting design, they may specify that certain
actions in a setting always require the use of a community
\textbf{resource}. Because the \textbf{resource} use is obligatory, it
need not meet the usual criteria for entertainment value. Also, when the
resource is used as an \textbf{augment}, you can also add a second
\textbf{augment} from some other \textbf{ability}, adjudicated according
to the standard rules, including entertainment value criteria. (This
way, the required \textbf{resource} use doesn't penalize you by forcing
you to \textbf{augment} with a low-rated \textbf{resource} when you
could otherwise use a higher-rated \textbf{ability}.)

\hypertarget{penalties-to-resources}{%
\subsubsection{8.2.2 Penalties to
Resources}\label{penalties-to-resources}}

Threats to community \textbf{resources} act as a spur to PC action. Your
GM may rule that the \textbf{penalty} from any \textbf{outcome} may be
applied to a \textbf{resource}. (It might at the same time be applied to
one or more PC \textbf{abilities}.)

When choosing a \textbf{penalty} arising from a player \textbf{defeat}
in a \textbf{simple contest}, your GM will use the \textbf{consequences
of defeat} table. For a \textbf{long contest}, the \textbf{penalty}
corresponds to the second worst \textbf{state of adversity} suffered by
a defeated group member.

If your group voluntarily concede a \textbf{contest} by withdrawing,
your community suffers \textbf{resource depletion} equivalent to a
\textbf{major defeat}.

\hypertarget{bolstering-resources}{%
\subsubsection{8.2.3 Bolstering Resources}\label{bolstering-resources}}

You can add \textbf{bonuses} to \textbf{bolster} community
\textbf{resources} by seeking out and overcoming relevant \textbf{story
obstacles}, specifying in the \textbf{contest framing} that the proceeds
of \textbf{victory} go the community. If you succeed, \textbf{bonuses}
from the \textbf{benefits of victory} table are applied to a resource
instead of one or more character abilities. (Your GM may rule that the
bonus also applies to you in social situations that involve community
members, reflecting gratitude for their efforts on behalf of the
community.)

\hypertarget{background-events}{%
\subsubsection{8.2.4 Background Events}\label{background-events}}

Your changes to \textbf{resources} take center stage in a series, but in
the background all sorts of other events periodically alter the
community's prosperity. These include the actions of other community
members, who are \textbf{depleting and bolstering resources} all the
time, as well as the unexpected intrusion of outside forces.

At the beginning of each interval, one of your group should perform a
\textbf{simple contest} of each \textbf{resource} against a
\textbf{resistance} equal to the average value of all
\textbf{resources}. These \textbf{contests} simulate \textbf{background
events} outside of your control or influence; they can't be
\textbf{augmented} or \textbf{bumped} up with \textbf{hero points}.

The \textbf{outcome} of the \textbf{contest} may apply a
\textbf{modifier} to a \textbf{resource}, as per the following table:

\hypertarget{resource-fluctuation-table}{%
\subsubsection{RESOURCE FLUCTUATION
TABLE}\label{resource-fluctuation-table}}

\begin{longtable}[]{@{}cc@{}}
\toprule
Outcome & Depletion Penalty\tabularnewline
\midrule
\endhead
Complete Victory & +9\tabularnewline
Major Victory & +6\tabularnewline
Minor Victory & +3\tabularnewline
Marginal Victory & 0\tabularnewline
Marginal Defeat & 0\tabularnewline
Minor Defeat & -3\tabularnewline
Major Defeat & -6\tabularnewline
Complete Defeat & -9\tabularnewline
\bottomrule
\end{longtable}

Except where your group is exceptionally keen on tracking
\textbf{resources}, your GM should skip the \textbf{background events}
process when the PCs are long absent from home. Your GM should rejigger
them to serve their plot purposes when they return. The GM may also want
to shuffle this process offstage when the PCs are occupied by epic
events. This prevents them from having to flee from a climactic plot
development to go home and tend to the beet crop.

\hypertarget{crisis-tests}{%
\subsubsection{8.2.5 Crisis Tests}\label{crisis-tests}}

When \textbf{resources} endure \textbf{penalties}, you conduct a
\textbf{crisis test} at the beginning of each game session to see if
trouble strikes the community. A high but \textbf{penalized rating} can
still lead to crisis, because people have adjusted to the equilibrium it
offers and feel squeezed when it shifts on them.

A \textbf{crisis test} is a \textbf{simple contest} (one for each
\textbf{penalized ability}) of the \textbf{resource rating} against a
\textbf{resistance} equal to the average of all \textbf{resource
ratings}. Like \textbf{background event} checks, these can't be
\textbf{augmented} or \textbf{bumped} up by player action. On any
\textbf{defeat}, the community starts to visibly suffer.

Your GM invents the specific reasons for each fluctuation and narrates
them to you.

\textbf{Crisis tests} should spur you to action, challenging you to find
ways to \textbf{bolster} the affected \textbf{resources} (see above).
When \textbf{bolstered}, the \textbf{crisis} is reversed. If you neglect
your duties or fail, the \textbf{crisis} worsens.

Your GM will call for \textbf{crisis tests} only as needed, as a tool to
generate story. If your group already has enough story on its hands,
your GM will suspend them until you next need a new plot hook.

\hypertarget{cementing-benefits-of-background-events}{%
\subsubsection{8.2.6 Cementing Benefits of Background
Events}\label{cementing-benefits-of-background-events}}

\textbf{Bonuses} from \textbf{background events} are temporary, unless
you take steps to \textbf{cement your benefits}. Doing so requires you
to overcome a major \textbf{story obstacle}, perhaps taking focus for an
evening's worth of play. If you succeed, the \textbf{background event
bonus} may, as per the next section, later solidify into a permanent
increase in the \textbf{resource's rating}.

When you \textbf{cement a background bonus}, your GM changes their
notation of that \textbf{bonus}.

\hypertarget{changes-to-resource-ratings}{%
\subsection{8.3 Changes to Resource
Ratings}\label{changes-to-resource-ratings}}

At the end of your GM's chosen interval, they review the Resource
Notation Table.

Any \textbf{resource} with a \textbf{bonus} of 3 or more in its PC
column increases by 1 for each 3 points of \textbf{bonus}, for a maximum
increase of 3.

Any \textbf{resource} with a \textbf{penalty} in its Total column
decreases by 1 for each 3 points of \textbf{penalty}, for a maximum loss
of 2.

Any remaining \textbf{modifiers} are now reduced to 0.

The GM now start a new Resource Notation Table, with \textbf{resource
ratings} altered to reflect any changes from the above process.

Having made permanent changes to the community's \textbf{resource
ratings}, your GM then restarts the cycle by again testing for a new set
of \textbf{background events}.

\hypertarget{changes-from-plot-events}{%
\subsection{8.4 Changes from Plot
Events}\label{changes-from-plot-events}}

Your GM may decide that certain remarkable triumphs or horrifying
catastrophes may directly alter a \textbf{resource rating}, independent
of the resource tracking system given here. The possibility of a
dramatic swing in community fortunes should be made clear by your GM
during \textbf{contest framing}, so that you know the \textbf{prize} and
can pull out all the stops to secure \textbf{victory} or stave off
\textbf{defeat}.

\hypertarget{appendix}{%
\section{9.0 Appendix}\label{appendix}}

\hypertarget{glossary-of-terms}{%
\subsection{9.1 Glossary of Terms}\label{glossary-of-terms}}

\textbf{Ability} Anything you can apply to solve a problem or overcome
an obstacle

\textbf{Advantage Point (AP)} A measure of advantage in an
\textbf{extended contest}.

\textbf{Ally} A \textbf{supporting character} of roughly equal ability
to your own.

\textbf{AP} Abbreviation for Advantage Point.

\textbf{AP Gifting} When you help another character, whilst uninvolved
in a \textbf{contest}, by giving them \textbf{advantage points} in an
\textbf{extended contest}.

\textbf{AP Lending} When you help another character, whilst engaged in a
\textbf{contest}, by lending them \textbf{advantage points}, in an
\textbf{extended contest}.

\textbf{Asymmetrical Exchange} In a \textbf{extended contest}, where you
are pressed by an opponent, but want to do something other than contend
directly for the \textbf{prize}.

\textbf{Asymmetrical Round} In a \textbf{scored contest}, where you are
pressed by an opponent, but want to do something other than contend
directly for the \textbf{prize}.

\textbf{Assist} In a \textbf{scored contest}, if you are unengaged you
may use an \textbf{assist} to reduce the \textbf{resource points} scored
against another character.

\textbf{Augment} Using one \textbf{ability} to help another
\textbf{ability}

\textbf{Automatic Victory} You have an appropriate \textbf{ability} and
the GM feels \textbf{failure} is not interesting, or makes the PC looks
un-heroic.

\textbf{Background Event} An off-stage \textbf{bonus} or
\textbf{penalty} applied to a \textbf{resource}.

\textbf{Base resistance} The \textbf{TN} for a \textbf{moderate
resistance class}, from which all other \textbf{resistance classes} are
figured as a \textbf{bonus} or \textbf{penalty}.

\textbf{Benefit of Victory} Long term positive modifier, because you won
a \textbf{contest}, against a challenging opponent (not -6 or less than
your \textbf{ability}). Usually a \textbf{state of fortune}.

\textbf{Bid} Also an \textbf{AP Bid} or \textbf{advantage point bid} is
your wager in an \textbf{extended contest}.

\textbf{Bolster} A \textbf{story obstacle} to apply a bonus to a
community \textbf{resource}

\textbf{Bonus} A positive modifier.

\textbf{Boost} Spending points ahead of a \textbf{group simple contest
outcome}, to improve the victory.

\textbf{Bump} An increment of the \textbf{result} of a roll, up or down.
So a bump up moves a \textbf{fumble}, to a \textbf{failure}, to a
\textbf{success} to a \textbf{critical}, a bump down moves a
\textbf{critical}, to a \textbf{success}, to a \textbf{failure} to a
\textbf{fumble}. One step is moved per \textbf{bump}. It is usually the
impact of a \textbf{hero point} or \textbf{mastery}.

\textbf{Catch-Up} When you cross a \textbf{mastery} threshold you can
increase lesser used \textbf{abilities} to ensure they keep pace.

\textbf{Climax} A \textbf{long contest} \textbf{story obstacle} that
provides the conclusion to a story.

\textbf{Contact} A \textbf{supporting character} who shares an
\textbf{occupation} or interest with your character.

\textbf{Contest} Where there is uncertainty as to whether a PC can
overcome a \textbf{story obstacle} or discover a secret, then your GM
can call for a contest to determine if the PC succeeds or fails. A
contest may be \textbf{simple} (one roll) of \textbf{extended} (a series
of rolls).

\textbf{Consequences of Defeat} Long term negative modifier, because you
lost a contest. Usually a \textbf{state of adversity}.

\textbf{Contest of Wherewithal} A \textbf{contest} that allows a
\textbf{dying} character to complete one \textbf{final action}.

\textbf{Contest Framing} Setting the stakes of the \textbf{contest},
what is this conflict about. Often not the immediate aftermath of
victory.

\textbf{Complete Defeat} No, and\ldots. You have lost, and the impact is
long-lasting, maybe even fatal or terminal.

\textbf{Complete Victory} Yes, and\ldots{} You have won, and the impact
is long-lasting, possibly a permanent change in your favor.

\textbf{Credibility Test} Is it possible to perform the action without
an \textbf{ability}, with an ordinary \textbf{ability}, or only with a
\textbf{extraordinary ability}?

\textbf{Crisis Test} Used to determine if a \textbf{resource} that has a
\textbf{penalty} creates a crisis.

\textbf{Defeat} Your \textbf{result} is worse than the
\textbf{resistance's} result.

\textbf{Defensive Response} In a \textbf{long contest} you can choose a
defensive \textbf{tactic} which reduces the \textbf{resource points} you
lose on a negative \textbf{result}.

\textbf{Degree of Victory or Defeat} How well did you triumph, or how
badly did you fail: \textbf{Critical}, \textbf{Success},
\textbf{Failure}, \textbf{Fumble}.

\textbf{Dependent} A \textbf{supporting character} who depends on your
PC.

\textbf{Depletion} Use of a community \textbf{resource} leads to its
depletion.

\textbf{Directed Improvement} When your GM grants you a new
\textbf{ability}, or an increase to an existing one, to recognize a
story event.

\textbf{Distinguishing Characteristic} The dominant personality
\textbf{ability} that others recognize in a character.

\textbf{Dying} A \textbf{state of adversity}, where the character's
\textbf{defeat} will end their participation.

\textbf{Edge} In an \textbf{extended contest} adds to the \textbf{APs}
lost or transferred when you win an \textbf{exchange}.

\textbf{Exchange} In an \textbf{extended contest} a round is divided
into two \textbf{exchanges} where both aggressor and defender act. In a
\textbf{group extended contest} a round consists of a sequence of
\textbf{exchanges} where everyone acts in turn. The GM determines the
order of action.

\textbf{Extended Contest} A type of \textbf{long contest} in which you
track the relative advantage one opponent has over another using
\textbf{advantage points}.

\textbf{Experience Points (XP)} When you do not spend a \textbf{hero
point} in a session it becomes an \textbf{experience point}, which can
accumulate between sessions.

\textbf{Extraordinary ability} Certain genres allow player characters to
have \textbf{abilities} that exceed human norms, these are
\textbf{extraordinary abilities}. A genre pack normally outlines what is
possible as part of its extraordinary powers framework.

\textbf{Failure} Rolling over your \textbf{target number}. It can be a
\textbf{fumble} or just a plain \textbf{failure}.

\textbf{Final Action} An attempt by \textbf{defeated}, but unengaged,
PCs to re-enter an \textbf{extended contest}.

\textbf{Flaw} An \textbf{ability} that penalizes you instead of helping
you.

\textbf{Fumble} The worst \textbf{failure} \textbf{result}, a notable
failure either due to incompetence or bad luck.

\textbf{Follower} A \textbf{supporting character} under your control.
Either a \textbf{sidekick} or \textbf{retainer}

\textbf{Framing the contest} You and your GM agree on the \textbf{prize}
for the victor, and your tactic in trying to win it.

\textbf{Group Chained Contest} A \textbf{chained contest} in which more
than a pair of opponents contend for the \textbf{prize}

\textbf{Group Extended Contest} An \textbf{extended contest} in which
more than a pair of opponents contend for the \textbf{prize}

\textbf{Group Scored Contest} A \textbf{scored contest} in which more
than a pair of opponents contend for the \textbf{prize}

\textbf{Group Simple Contest} A \textbf{simple contest} where one side
has multiple participants.

\textbf{Graduated Goals} When a contestant has a \textbf{primary} and
\textbf{secondary} goal, and may have to choose between them if their
\textbf{outcome} is not a \textbf{major} or \textbf{complete} victory.

\textbf{Handicap} In an \textbf{extended contest} subtracts from the
\textbf{APs} lost or transferred when you win an \textbf{exchange}.

\textbf{Hero Point} Allows you to alter fate for a player character,
either by a \textbf{bump} to their \textbf{result} or a \textbf{plot
edit}. If unused, becomes an \textbf{experience point} at the end of the
session.

\textbf{Hurt} A state of adversity, a flesh wound or injured pride,
heals at the end of a session.

\textbf{Keyword} A single \textbf{ability} that encompasses a range of
abilities within it, such as an \textbf{occupation} or culture. An
\textbf{ability} within an \textbf{umbrella keyword} is a
\textbf{break-out ability}, an \textbf{ability} within a \textbf{package
keyword} is a \textbf{stand-alone ability}.

\textbf{Long Contest} A \textbf{contest} where we drill-down to the
individual exchanges that resolve the conflict. We support
\textbf{scored}, \textbf{extended}, and \textbf{chained contests}

\textbf{Major Defeat} No, and. You have lost, and the impact is
long-lasting.

\textbf{Major Victory}, Yes, and. You have won, and the impact is
long-lasting.

\textbf{Marginal Defeat} No, but\ldots{} You don't get what you want,
but the damage may be mitigated.

\textbf{Marginal Victory} Yes, but\ldots{} You get what you want, but
you may have to make a hard choice.

\textbf{Modifiers} Adjustments to a \textbf{target number} due to
circumstance.

\textbf{Mastery} An \textbf{ability} \textbf{rating} that rises above 20
is said to have a \textbf{mastery}. \textbf{Masteries} cancel each other
out in \textbf{contests}. \textbf{Masteries} that are not cancelled
provide a \textbf{bump}.

\textbf{Minor Defeat} No\ldots{} you don't get the agreed
\textbf{prize}.

\textbf{Minor Victory} Yes\ldots{} you get the agreed \textbf{prize}.

\textbf{Mismatched Goals} When the opposing sides in a \textbf{contest}
want different \textbf{prizes}.

\textbf{Occupation} An \textbf{ability} that indicates the profession,
or primary area of expertise, of your character.

\textbf{Outcome} A \textbf{contest} has an \textbf{outcome}, described
as a \textbf{victory} or \textbf{defeat} in obtaining the \textbf{prize}
that was agreed in \textbf{contest framing} for any PCs involved.

\textbf{Outcome Point} A point scored in favor one side in a
\textbf{group simple contest}

\textbf{Parting Shot} An attempt to make your opponent's \textbf{defeat}
worse in a \textbf{long contest} (\textbf{scored} or \textbf{extended}),
by `finishing them off'.

\textbf{Patron} A \textbf{supporting character} with superior assets.

\textbf{Penalty} A negative modifier.

\textbf{Prize} What is at stake in the \textbf{contest}, decided during
\textbf{framing}.

\textbf{Rating} An ability has a rating, indicating how likely a
character is to succeed at using it.

\textbf{Resistance} The forces opposing the PC in a conflict, or
concealing a secret that must be overcome by using an \textbf{ability}
in a \textbf{contest}. One of: \textbf{Nearly Impossible}, \textbf{Very
High}, \textbf{High}, \textbf{Moderate}, \textbf{Low}, \textbf{Very
Low}.

\textbf{Resistance Class} The \textbf{bonus} or \textbf{penalty} to the
\textbf{resistance} \textbf{TN}, depending on the GM's interpretation of
how \emph{dramatically} hard the \textbf{story obstacle} is.

\textbf{Resolution Point (RP)} In a \textbf{scored contest} an
\textbf{RP} tracks the advantage one contestant has over the other.

\textbf{Resource} A community \textbf{ability} that your PC may draw on.

\textbf{Result} The \textbf{outcome} of a die roll against a
\textbf{TN}. One of \textbf{critical}, \textbf{success},
\textbf{failure}, and \textbf{fumble}.

\textbf{Retainer} A \textbf{follower} of your PC who is not `fleshed
out' and cannot act independently.

\textbf{Rising Action} A \textbf{scored contest} where the \textbf{story
obstacle} is a step towards the final \textbf{story obstacle} of this
story.

\textbf{Risky Gambit} In a \textbf{long contest} you can take an action
that puts you at more risk on defeat, but enhances victory.

\textbf{Round} A \textbf{long contest} is broken into a series of
rounds, each of which is an attempt to obtain the \textbf{prize}. In an
\textbf{extended contest} a round is further broken into a number of
\textbf{exchanges} in which all participants have the chance to act.

\textbf{Scored Contest} A \textbf{long contest} where we track the
relative advantage one contestant has over another using
\textbf{resolution points}

\textbf{Sidekick} A fleshed out \textbf{follower} of your PC who can act
independently.

\textbf{Supporting Characters} Additional characters under the player's
control that play a supporting role to their PC.

\textbf{Simple Contest} A one roll resolution method, the default
\textbf{contest} type, used when learning the \textbf{outcome} matters
more than the breakdown of how you achieved it.

\textbf{Stand Alone Ability} An \textbf{ability} raised separately to a
\textbf{keyword}. It may have been added to the character as part of a
\textbf{package keyword}, or on its own.

\textbf{State of Adversity} How `banged up' a PC is, physically or
metaphorically, following a \textbf{defeat}: \textbf{Hurt},
\textbf{Injured}, \textbf{Impaired}, \textbf{Dying} and \textbf{Dead}

\textbf{State of Fortune} A `boost' to the PC which may be physical or
metaphorical.

\textbf{Story Obstacle} Something that prevents you from getting what
you want, the \textbf{prize}. A \textbf{story obstacle} is the trigger
for a \textbf{contest}.

\textbf{Stretch} A \textbf{penalty} applied to an \textbf{ability}
because it is stretches credibility that it is a reasonable
\textbf{tactic}.

\textbf{Success} Rolling under your \textbf{target number}. It can be a
\textbf{critical} or just a plain \textbf{success}.

\textbf{Target Number (TN)} The number, either an \textbf{ability}
\textbf{rating}, or a \textbf{resistance}, to roll under or equal to in
order to \textbf{succeed}.

\textbf{TN} Abbreviation for \textbf{Target Number}

\textbf{Unrelated Action} An action when you are disengaged in a
\textbf{long contest} that does not relate to your attempt to win the
\textbf{prize}.

\textbf{Victory} Your \textbf{result} is a better roll than the
\textbf{resistance}.

\hypertarget{version-changes}{%
\subsection{9.2 Version Changes}\label{version-changes}}

\hypertarget{version-2.2}{%
\subsubsection{Version 2.2}\label{version-2.2}}

\begin{itemize}
\tightlist
\item
  Uses of he/she changed to they/their
\item
  Flagged some rules that are not required, as optional to allow those
  using the SRD to safely omit them if not required.
\item
  Clarified that contest results are only reciprocal between PCs. When
  the contest is against a resistance set by the GM, the results
  indicate whether the PC gains the prize, and the GM narrates the
  result for the resistance based on this.
\item
  Rephrased the contest results to emphasize: Yes, No, And\ldots,
  But\ldots, This change is designed to dissuade GMs from
  misunderstanding that the prize is obtained on a marginal victory, one
  of the most common result types, and instead encourage GMs to allow
  PCs to fail forward on such a result by introducing downstream
  complications.
\item
  Provided clarity that consequences of defeat and benefit of victory
  are optional and the GM should focus on using the prize to narrate the
  outcome of a contest, only applying mechanical benefits if they make
  sense.
\item
  Specific Ability Bonuses are dropped. They were hard for the GM to
  adjudicate and the same intent is better served by using a stretch on
  a broad ability when contesting against a PC with a more specific
  ability.
\item
  A winning group in a Group Simple Contest does not suffer a
  Consequence of Defeat as a result of a low RP difference victory any
  more, the GM should narrate consequences from the level of victory, if
  appropriate.
\item
  Dropped the negative consequences for the winner in an Extended
  Contest during the Rising Action. If the winner is a PC the degree of
  success already suggests consequences in addition to the prize on a
  marginal victory. So this rule is over-complication.
\item
  Made it clear that only a PC should use a parting shot, not the
  resistance.
\item
  Switched to addressing you the player, using your GM for the Games
  Master, and we for the game authors
\item
  Long contests include both extended contest and scored contests.
  Between 1 and 2 extended contests switched to scored contests, this
  approach restores both variants, but requires changing the generic
  name to a long contest.
\item
  Dropped edges and handicaps from extended contests - we use a
  resistance not stats, so makes no sense to have edges and handicaps
\item
  Added alternate mechanisms for determining if resistance advances and
  when
\item
  Added story-based resistance mechanics
\item
  Added story-based improvements
\item
  Added States of Fortune to mirror States of Adversity. Overall
  mirrored benefits and consequences more closely
\item
  Added Escalating Contests
\item
  Added Mythic Russia's Plot Edits
\item
  Added Mythic Russia's Pyrrhic Victories for Extended Contests but as
  Climatic Contests
\item
  Changed degree of success and failure, to degree of victory and
  defeat, as success and failure are for individual rolls, victory and
  defeat once compared.
\item
  Simplified how multiple opponents are handled
\item
  Clarified contest outcomes for long contests, and how to determine the
  overall winner in a long contest
\item
  Do not allow transfers in an extended contest where the abilities
  differ by 6 or more. Consistent with benefits of victory and prevents
  `loading up on mooks' as a strategy.
\item
  Added new resistance classes to preserve the +3, +6, +9 model used
  elsewhere. Some classes will now have a different value from those
  that appeared in prior versions.
\end{itemize}

\end{document}
